\documentclass[sigplan,screen,nonacm]{acmart}
\settopmatter{printfolios=true}

\usepackage[conor]{agda}
\DisableLigatures[-]{encoding = T1, family = tt* }
% agda-custom.sty

% A package for use when generating highlighted PDFs from Agda code.

% Disclaimer: This is not an official Agda package.

% This package should be used after the agda and agda-unicode packages.
% It is intended for overriding definitions, loading further packages, etc.

% N.B. The package agda-unicode.sty defines 100s of Unicode characters. 

% Example (not to be used together with agda-unicode.sty):
%
% \newunicodechar{ℕ}{\ensuremath{\mathbb N}}

%%%%%%%%%%%%%%%%%%%%%%%%%%%%%%%%%%%%%%%%%%%%%%%%%%%%%%%%%%%%%%%%%%%%%%%%%%%%%%%
% Typesetting of individual symbols:
\renewcommand{\AgdaSymbol}[1]
  {\AgdaNoSpaceMath{\textcolor{AgdaSymbol}{\AgdaFontStyle{\AgdaFormat{#1}{#1}}}}}

% Typeset '->' as \to, and '\' as \lambda:
\DeclareRobustCommand{\AgdaFormat}[2]{%
  \ifthenelse{\equal{#1}{->}}{\ensuremath\to}{%
     \ifthenelse{\equal{#1}{\textbackslash{}}}{\ensuremath\lambda}{%
      #2}}}

% Add thin spaces around '_':
\renewcommand{\AgdaUnderscore}{\,\_\,}
% N.B. This affects the alignment of infix and postfix operator declarations.
% Ideally, spaces would be inserted only between '_' and non-space characters.

%%%%%%%%%%%%%%%%%%%%%%%%%%%%%%%%%%%%%%%%%%%%%%%%%%%%%%%%%%%%%%%%%%%%%%%%%%%%%%%
% Font styles

% agda.sty:
% \newcommand{\AgdaFontStyle}[1]{\textsf{#1}}
% \ifthenelse{\equal{\AgdaColourScheme}{bw}}{
%     \newcommand{\AgdaKeywordFontStyle}[1]{\underline{#1}}
% }{
%     \newcommand{\AgdaKeywordFontStyle}[1]{\textsf{#1}}
% }
% \newcommand{\AgdaStringFontStyle}[1]{\texttt{#1}}
% \newcommand{\AgdaCommentFontStyle}[1]{\texttt{#1}}
% \newcommand{\AgdaBoundFontStyle}[1]{\textit{#1}}

% The author's personal preferences:
\renewcommand{\AgdaStringFontStyle}[1]{\textrm{#1}}
\renewcommand{\AgdaCommentFontStyle}[1]{\texttt{\small#1}}
\renewcommand{\AgdaBoundFontStyle}[1]{\textsf{#1}}

\begin{document}

\title{A Compositional Semantics for \emph{eval} in Scheme}

\subtitle{Listing of a Lightweight Agda Formalization}

\author{Peter D. Mosses}
\orcid{0000-0002-5826-7520}
\affiliation{%
  \institution{Delft University of Technology}
  \city{Delft}
  \country{Netherlands}
}
\affiliation{%
  \institution{Swansea University}
  \city{Swansea}
  \country{United Kingdom}
}
\email{P.D.Mosses@tudelft.nl}

\begin{abstract}

\textsc{Scm} is a simple sublanguage of \textsc{Scheme};
\textsc{ScmQ} adds quotations to \textsc{Scm}; and
\textsc{ScmQE} adds \texttt{eval} expressions to \textsc{ScmQ}.
An accompanying paper presents a denotational semantics of \textsc{Scm},
and the additions and changes to define the semantics of \textsc{ScmQ}
and \textsc{ScmQE}.

This document provides a highlighted listing of the \textsc{Agda} source code
of a lightweight formalization of the complete denotational semantics of
\textsc{ScmQE}, and illustrates how soundness tests can be formulated and verified.
For explanatory comments, see §6 of the accompanying paper.

\textsc{Agda} generated the \LaTeX\ sources for the highlighted listing;
a map from \textsc{Unicode} characters to similar-looking math symbols was
manually coded. The \LaTeX\ sources for the illustrative fragments presented
in the body of the accompanying paper were copied from the \textsc{Agda}-generated
sources, but may have been edited to adjust layout and alignment.

\end{abstract}

%% The code below is generated by the tool at http://dl.acm.org/ccs.cfm.
\begin{CCSXML}
<ccs2012>
   <concept>
       <concept_id>10003752.10010124.10010131.10010133</concept_id>
       <concept_desc>Theory of computation~Denotational semantics</concept_desc>
       <concept_significance>500</concept_significance>
       </concept>
  <concept>
       <concept_id>10011007.10011006.10011039.10011311</concept_id>
       <concept_desc>Software and its engineering~Semantics</concept_desc>
       <concept_significance>500</concept_significance>
       </concept>
   <concept>
       <concept_id>10011007.10011006.10011008.10011009.10011012</concept_id>
       <concept_desc>Software and its engineering~Functional languages</concept_desc>
       <concept_significance>500</concept_significance>
       </concept>
 </ccs2012>
\end{CCSXML}

\ccsdesc[500]{Theory of computation~Denotational semantics}
\ccsdesc[500]{Software and its engineering~Semantics}
\ccsdesc[500]{Software and its engineering~Functional languages}%%

\keywords{Scheme, denotational semantics, compositional semantics,
  quote and eval, Lisp, formalization, Agda}

\maketitle

\nocite{Mosses2025CSE,Mosses2025CSE-artifact}

\bibliographystyle{ACM-Reference-Format}
\bibliography{ScmQE}

\pagebreak

\tableofcontents

\section*{Modules}

\begin{code}%
\>[0]\AgdaKeyword{module}\AgdaSpace{}%
\AgdaModule{ScmQE.All}\AgdaSpace{}%
\AgdaKeyword{where}\<%
\\
%
\\[\AgdaEmptyExtraSkip]%
\>[0]\AgdaKeyword{import}\AgdaSpace{}%
\AgdaModule{Notation}\<%
\\
\>[0]\AgdaKeyword{import}\AgdaSpace{}%
\AgdaModule{ScmQE.Abstract-Syntax}\<%
\\
\>[0]\AgdaKeyword{import}\AgdaSpace{}%
\AgdaModule{ScmQE.Domain-Equations}\<%
\\
\>[0]\AgdaKeyword{import}\AgdaSpace{}%
\AgdaModule{ScmQE.Semantic-Functions}\<%
\\
\>[0]\AgdaKeyword{import}\AgdaSpace{}%
\AgdaModule{ScmQE.Auxiliary-Functions}\<%
\end{code}

\clearpage
\onecolumn

\section{Abstract Syntax}
\begin{code}%
\>[0]\<%
\\
\>[0]\AgdaKeyword{module}\AgdaSpace{}%
\AgdaModule{PrimSchemeQED.Abstract-Syntax}\AgdaSpace{}%
\AgdaKeyword{where}\<%
\\
%
\\[\AgdaEmptyExtraSkip]%
\>[0]\AgdaKeyword{open}\AgdaSpace{}%
\AgdaKeyword{import}\AgdaSpace{}%
\AgdaModule{PrimSchemeQED.Domain-Notation}\<%
\\
\>[0][@{}l@{\AgdaIndent{0}}]%
\>[2]\AgdaKeyword{using}\AgdaSpace{}%
\AgdaSymbol{(}\AgdaOperator{\AgdaFunction{\AgdaUnderscore{}⋆′}}\AgdaSymbol{)}\<%
\\
%
\\[\AgdaEmptyExtraSkip]%
\>[0]\AgdaKeyword{open}\AgdaSpace{}%
\AgdaKeyword{import}\AgdaSpace{}%
\AgdaModule{Data.Bool.Base}\<%
\\
\>[0][@{}l@{\AgdaIndent{0}}]%
\>[2]\AgdaKeyword{using}\AgdaSpace{}%
\AgdaSymbol{(}\AgdaDatatype{Bool}\AgdaSymbol{)}\<%
\\
\>[0]\AgdaKeyword{open}\AgdaSpace{}%
\AgdaKeyword{import}\AgdaSpace{}%
\AgdaModule{Data.Integer.Base}\<%
\\
\>[0][@{}l@{\AgdaIndent{0}}]%
\>[2]\AgdaKeyword{renaming}\AgdaSpace{}%
\AgdaSymbol{(}\AgdaDatatype{ℤ}\AgdaSpace{}%
\AgdaSymbol{to}\AgdaSpace{}%
\AgdaDatatype{Int}\AgdaSymbol{)}\<%
\\
\>[0]\AgdaKeyword{open}\AgdaSpace{}%
\AgdaKeyword{import}\AgdaSpace{}%
\AgdaModule{Data.String.Base}\<%
\\
\>[0][@{}l@{\AgdaIndent{0}}]%
\>[2]\AgdaKeyword{using}\AgdaSpace{}%
\AgdaSymbol{(}\AgdaPostulate{String}\AgdaSymbol{)}\<%
\\
%
\\[\AgdaEmptyExtraSkip]%
\>[0]\AgdaComment{--\ 7.2.1.\ Abstract\ syntax}\<%
\\
%
\\[\AgdaEmptyExtraSkip]%
\>[0]\AgdaKeyword{data}\AgdaSpace{}%
\AgdaDatatype{Con}%
\>[10]\AgdaSymbol{:}\AgdaSpace{}%
\AgdaPrimitive{Set}%
\>[20]\AgdaComment{--\ constants,\ *excluding*\ quotations}\<%
\\
\>[0]\AgdaFunction{Ide}%
\>[10]\AgdaSymbol{=}\AgdaSpace{}%
\AgdaPostulate{String}%
\>[20]\AgdaComment{--\ identifiers\ (variables)}\<%
\\
\>[0]\AgdaKeyword{data}\AgdaSpace{}%
\AgdaDatatype{Key}%
\>[10]\AgdaSymbol{:}\AgdaSpace{}%
\AgdaPrimitive{Set}%
\>[20]\AgdaComment{--\ keywords}\<%
\\
\>[0]\AgdaKeyword{data}\AgdaSpace{}%
\AgdaDatatype{Dat}%
\>[10]\AgdaSymbol{:}\AgdaSpace{}%
\AgdaPrimitive{Set}%
\>[20]\AgdaComment{--\ external\ representations}\<%
\\
\>[0]\AgdaKeyword{data}\AgdaSpace{}%
\AgdaDatatype{Exp}%
\>[10]\AgdaSymbol{:}\AgdaSpace{}%
\AgdaPrimitive{Set}%
\>[20]\AgdaComment{--\ expressions}\<%
\\
%
\\[\AgdaEmptyExtraSkip]%
\>[0]\AgdaKeyword{data}\AgdaSpace{}%
\AgdaDatatype{Con}\AgdaSpace{}%
\AgdaKeyword{where}\<%
\\
\>[0][@{}l@{\AgdaIndent{0}}]%
\>[2]\AgdaInductiveConstructor{int}\AgdaSpace{}%
\AgdaSymbol{:}\AgdaSpace{}%
\AgdaDatatype{Int}\AgdaSpace{}%
\AgdaSymbol{→}\AgdaSpace{}%
\AgdaDatatype{Con}\<%
\\
%
\>[2]\AgdaInductiveConstructor{\#t}%
\>[6]\AgdaSymbol{:}\AgdaSpace{}%
\AgdaDatatype{Con}\<%
\\
%
\>[2]\AgdaInductiveConstructor{\#f}%
\>[6]\AgdaSymbol{:}\AgdaSpace{}%
\AgdaDatatype{Con}\<%
\\
%
\\[\AgdaEmptyExtraSkip]%
\>[0]\AgdaKeyword{data}\AgdaSpace{}%
\AgdaDatatype{Key}\AgdaSpace{}%
\AgdaKeyword{where}\<%
\\
\>[0][@{}l@{\AgdaIndent{0}}]%
\>[2]\AgdaInductiveConstructor{quote′}%
\>[10]\AgdaSymbol{:}\AgdaSpace{}%
\AgdaDatatype{Key}\<%
\\
%
\>[2]\AgdaInductiveConstructor{lambda}%
\>[10]\AgdaSymbol{:}\AgdaSpace{}%
\AgdaDatatype{Key}\<%
\\
%
\>[2]\AgdaInductiveConstructor{if}%
\>[10]\AgdaSymbol{:}\AgdaSpace{}%
\AgdaDatatype{Key}\<%
\\
%
\>[2]\AgdaInductiveConstructor{set!}%
\>[10]\AgdaSymbol{:}\AgdaSpace{}%
\AgdaDatatype{Key}\<%
\\
%
\>[2]\AgdaInductiveConstructor{eval}%
\>[10]\AgdaSymbol{:}\AgdaSpace{}%
\AgdaDatatype{Key}\<%
\\
%
\\[\AgdaEmptyExtraSkip]%
\>[0]\AgdaKeyword{data}\AgdaSpace{}%
\AgdaDatatype{Dat}\AgdaSpace{}%
\AgdaKeyword{where}\<%
\\
\>[0][@{}l@{\AgdaIndent{0}}]%
\>[2]\AgdaInductiveConstructor{con}%
\>[9]\AgdaSymbol{:}\AgdaSpace{}%
\AgdaDatatype{Con}\AgdaSpace{}%
\AgdaSymbol{→}\AgdaSpace{}%
\AgdaDatatype{Dat}%
\>[31]\AgdaComment{--\ constants}\<%
\\
%
\>[2]\AgdaInductiveConstructor{ide}%
\>[9]\AgdaSymbol{:}\AgdaSpace{}%
\AgdaFunction{Ide}\AgdaSpace{}%
\AgdaSymbol{→}\AgdaSpace{}%
\AgdaDatatype{Dat}%
\>[31]\AgdaComment{--\ symbols}\<%
\\
%
\>[2]\AgdaInductiveConstructor{key}%
\>[9]\AgdaSymbol{:}\AgdaSpace{}%
\AgdaDatatype{Key}\AgdaSpace{}%
\AgdaSymbol{→}\AgdaSpace{}%
\AgdaDatatype{Dat}%
\>[31]\AgdaComment{--\ keyword}\<%
\\
%
\>[2]\AgdaInductiveConstructor{′}%
\>[9]\AgdaSymbol{:}\AgdaSpace{}%
\AgdaDatatype{Dat}\AgdaSpace{}%
\AgdaSymbol{→}\AgdaSpace{}%
\AgdaDatatype{Dat}%
\>[31]\AgdaComment{--\ 'Δ}\<%
\\
%
\>[2]\AgdaOperator{\AgdaInductiveConstructor{⦅\AgdaUnderscore{}⦆}}%
\>[9]\AgdaSymbol{:}\AgdaSpace{}%
\AgdaDatatype{Dat}\AgdaSpace{}%
\AgdaOperator{\AgdaFunction{⋆′}}\AgdaSpace{}%
\AgdaSymbol{→}\AgdaSpace{}%
\AgdaDatatype{Dat}%
\>[31]\AgdaComment{--\ lists\ (Δ⋆)}\<%
\\
%
\>[2]\AgdaOperator{\AgdaInductiveConstructor{⦅\AgdaUnderscore{}·\AgdaUnderscore{}⦆}}%
\>[9]\AgdaSymbol{:}\AgdaSpace{}%
\AgdaDatatype{Dat}\AgdaSpace{}%
\AgdaOperator{\AgdaFunction{⋆′}}\AgdaSpace{}%
\AgdaSymbol{→}\AgdaSpace{}%
\AgdaDatatype{Dat}\AgdaSpace{}%
\AgdaSymbol{→}\AgdaSpace{}%
\AgdaDatatype{Dat}%
\>[31]\AgdaComment{--\ pairs\ (Δ⋆.Δ)}\<%
\\
%
\\[\AgdaEmptyExtraSkip]%
\>[0]\AgdaKeyword{data}\AgdaSpace{}%
\AgdaDatatype{Exp}\AgdaSpace{}%
\AgdaKeyword{where}\<%
\\
\>[0][@{}l@{\AgdaIndent{0}}]%
\>[2]\AgdaInductiveConstructor{con}%
\>[17]\AgdaSymbol{:}\AgdaSpace{}%
\AgdaDatatype{Con}\AgdaSpace{}%
\AgdaSymbol{→}\AgdaSpace{}%
\AgdaDatatype{Exp}%
\>[42]\AgdaComment{--\ K}\<%
\\
%
\>[2]\AgdaInductiveConstructor{ide}%
\>[17]\AgdaSymbol{:}\AgdaSpace{}%
\AgdaFunction{Ide}\AgdaSpace{}%
\AgdaSymbol{→}\AgdaSpace{}%
\AgdaDatatype{Exp}%
\>[42]\AgdaComment{--\ I}\<%
\\
%
\>[2]\AgdaOperator{\AgdaInductiveConstructor{⦅\AgdaUnderscore{}␣\AgdaUnderscore{}⦆}}%
\>[17]\AgdaSymbol{:}\AgdaSpace{}%
\AgdaDatatype{Exp}\AgdaSpace{}%
\AgdaSymbol{→}\AgdaSpace{}%
\AgdaDatatype{Exp}\AgdaSpace{}%
\AgdaOperator{\AgdaFunction{⋆′}}\AgdaSpace{}%
\AgdaSymbol{→}\AgdaSpace{}%
\AgdaDatatype{Exp}%
\>[42]\AgdaComment{--\ (E₀\ E⋆′)}\<%
\\
%
\>[2]\AgdaOperator{\AgdaInductiveConstructor{⦅lambda␣⦅\AgdaUnderscore{}⦆\AgdaUnderscore{}⦆}}%
\>[17]\AgdaSymbol{:}\AgdaSpace{}%
\AgdaFunction{Ide}\AgdaSpace{}%
\AgdaOperator{\AgdaFunction{⋆′}}\AgdaSpace{}%
\AgdaSymbol{→}\AgdaSpace{}%
\AgdaDatatype{Exp}\AgdaSpace{}%
\AgdaSymbol{→}\AgdaSpace{}%
\AgdaDatatype{Exp}%
\>[42]\AgdaComment{--\ (lambda\ (I⋆′)\ E₀)}\<%
\\
%
\>[2]\AgdaOperator{\AgdaInductiveConstructor{⦅if\AgdaUnderscore{}␣\AgdaUnderscore{}␣\AgdaUnderscore{}⦆}}%
\>[17]\AgdaSymbol{:}\AgdaSpace{}%
\AgdaDatatype{Exp}\AgdaSpace{}%
\AgdaSymbol{→}\AgdaSpace{}%
\AgdaDatatype{Exp}\AgdaSpace{}%
\AgdaSymbol{→}\AgdaSpace{}%
\AgdaDatatype{Exp}\AgdaSpace{}%
\AgdaSymbol{→}\AgdaSpace{}%
\AgdaDatatype{Exp}%
\>[42]\AgdaComment{--\ (if\ E₀\ E₁\ E₂)}\<%
\\
%
\>[2]\AgdaOperator{\AgdaInductiveConstructor{⦅set!\AgdaUnderscore{}␣\AgdaUnderscore{}⦆}}%
\>[17]\AgdaSymbol{:}\AgdaSpace{}%
\AgdaFunction{Ide}\AgdaSpace{}%
\AgdaSymbol{→}\AgdaSpace{}%
\AgdaDatatype{Exp}\AgdaSpace{}%
\AgdaSymbol{→}\AgdaSpace{}%
\AgdaDatatype{Exp}%
\>[42]\AgdaComment{--\ (set!\ I\ E)}\<%
\\
%
\>[2]\AgdaOperator{\AgdaInductiveConstructor{⦅quote\AgdaUnderscore{}⦆}}%
\>[17]\AgdaSymbol{:}\AgdaSpace{}%
\AgdaDatatype{Dat}\AgdaSpace{}%
\AgdaSymbol{→}\AgdaSpace{}%
\AgdaDatatype{Exp}%
\>[42]\AgdaComment{--\ (quote\ Δ)}\<%
\\
%
\>[2]\AgdaInductiveConstructor{′}%
\>[17]\AgdaSymbol{:}\AgdaSpace{}%
\AgdaDatatype{Dat}\AgdaSpace{}%
\AgdaSymbol{→}\AgdaSpace{}%
\AgdaDatatype{Exp}%
\>[42]\AgdaComment{--\ ′\ Δ}\<%
\\
%
\>[2]\AgdaOperator{\AgdaInductiveConstructor{⦅eval\AgdaUnderscore{}⦆}}%
\>[17]\AgdaSymbol{:}\AgdaSpace{}%
\AgdaDatatype{Exp}\AgdaSpace{}%
\AgdaSymbol{→}\AgdaSpace{}%
\AgdaDatatype{Exp}%
\>[42]\AgdaComment{--\ (eval\ E)}\<%
\\
%
\>[2]\AgdaInductiveConstructor{⦅␣⦆}%
\>[17]\AgdaSymbol{:}\AgdaSpace{}%
\AgdaDatatype{Exp}%
\>[42]\AgdaComment{--\ illegal}\<%
\\
%
\\[\AgdaEmptyExtraSkip]%
\>[0]\AgdaKeyword{variable}\<%
\\
\>[0][@{}l@{\AgdaIndent{0}}]%
\>[2]\AgdaGeneralizable{Z}%
\>[6]\AgdaSymbol{:}\AgdaSpace{}%
\AgdaDatatype{Int}\<%
\\
%
\>[2]\AgdaGeneralizable{K}%
\>[6]\AgdaSymbol{:}\AgdaSpace{}%
\AgdaDatatype{Con}\<%
\\
%
\>[2]\AgdaGeneralizable{I}%
\>[6]\AgdaSymbol{:}\AgdaSpace{}%
\AgdaFunction{Ide}\<%
\\
%
\>[2]\AgdaGeneralizable{I⋆}%
\>[6]\AgdaSymbol{:}\AgdaSpace{}%
\AgdaFunction{Ide}\AgdaSpace{}%
\AgdaOperator{\AgdaFunction{⋆′}}\<%
\\
%
\>[2]\AgdaGeneralizable{X}%
\>[6]\AgdaSymbol{:}\AgdaSpace{}%
\AgdaDatatype{Key}\<%
\\
%
\>[2]\AgdaGeneralizable{E}%
\>[6]\AgdaSymbol{:}\AgdaSpace{}%
\AgdaDatatype{Exp}\<%
\\
%
\>[2]\AgdaGeneralizable{E⋆}%
\>[6]\AgdaSymbol{:}\AgdaSpace{}%
\AgdaDatatype{Exp}\AgdaSpace{}%
\AgdaOperator{\AgdaFunction{⋆′}}\<%
\\
%
\>[2]\AgdaGeneralizable{Δ}%
\>[6]\AgdaSymbol{:}\AgdaSpace{}%
\AgdaDatatype{Dat}\<%
\\
%
\>[2]\AgdaGeneralizable{Δ⋆}%
\>[6]\AgdaSymbol{:}\AgdaSpace{}%
\AgdaDatatype{Dat}\AgdaSpace{}%
\AgdaOperator{\AgdaFunction{⋆′}}\<%
\\
\>[0]\<%
\end{code}

\clearpage

\section{Domain Equations}
\begin{code}%
\>[0]\AgdaKeyword{module}\AgdaSpace{}%
\AgdaModule{ScmQE.Domain-Equations}\AgdaSpace{}%
\AgdaKeyword{where}\<%
\\
%
\\[\AgdaEmptyExtraSkip]%
\>[0]\AgdaKeyword{open}\AgdaSpace{}%
\AgdaKeyword{import}\AgdaSpace{}%
\AgdaModule{Notation}\<%
\\
\>[0]\AgdaKeyword{open}\AgdaSpace{}%
\AgdaKeyword{import}\AgdaSpace{}%
\AgdaModule{ScmQE.Abstract-Syntax}\AgdaSpace{}%
\AgdaKeyword{using}\AgdaSpace{}%
\AgdaSymbol{(}\AgdaFunction{Ide}\AgdaSymbol{;}\AgdaSpace{}%
\AgdaDatatype{Key}\AgdaSymbol{;}\AgdaSpace{}%
\AgdaDatatype{Dat}\AgdaSymbol{;}\AgdaSpace{}%
\AgdaDatatype{Int}\AgdaSymbol{)}\<%
\\
%
\\[\AgdaEmptyExtraSkip]%
\>[0]\AgdaComment{--\ Domain\ declarations}\<%
\\
%
\\[\AgdaEmptyExtraSkip]%
\>[0]\AgdaKeyword{postulate}%
\>[11]\AgdaPostulate{𝐋}%
\>[15]\AgdaSymbol{:}%
\>[18]\AgdaFunction{Domain}%
\>[26]\AgdaComment{--\ locations}\<%
\\
\>[0]\AgdaKeyword{variable}%
\>[11]\AgdaGeneralizable{α}%
\>[15]\AgdaSymbol{:}%
\>[18]\AgdaPostulate{𝐋}\<%
\\
\>[0]\AgdaFunction{𝐍}%
\>[15]\AgdaSymbol{:}%
\>[18]\AgdaFunction{Domain}%
\>[26]\AgdaComment{--\ natural\ numbers}\<%
\\
\>[0]\AgdaFunction{𝐓}%
\>[15]\AgdaSymbol{:}%
\>[18]\AgdaFunction{Domain}%
\>[26]\AgdaComment{--\ booleans}\<%
\\
\>[0]\AgdaFunction{𝐑}%
\>[15]\AgdaSymbol{:}%
\>[18]\AgdaFunction{Domain}%
\>[26]\AgdaComment{--\ numbers}\<%
\\
\>[0]\AgdaFunction{𝐏}%
\>[15]\AgdaSymbol{:}%
\>[18]\AgdaFunction{Domain}%
\>[26]\AgdaComment{--\ pairs}\<%
\\
\>[0]\AgdaFunction{𝐌}%
\>[15]\AgdaSymbol{:}%
\>[18]\AgdaFunction{Domain}%
\>[26]\AgdaComment{--\ miscellaneous}\<%
\\
\>[0]\AgdaFunction{𝐅}%
\>[15]\AgdaSymbol{:}%
\>[18]\AgdaFunction{Domain}%
\>[26]\AgdaComment{--\ procedure\ values}\<%
\\
\>[0]\AgdaFunction{𝐐}%
\>[15]\AgdaSymbol{:}%
\>[18]\AgdaFunction{Domain}%
\>[26]\AgdaComment{--\ symbols}\<%
\\
\>[0]\AgdaFunction{𝐗}%
\>[15]\AgdaSymbol{:}%
\>[18]\AgdaFunction{Domain}%
\>[26]\AgdaComment{--\ keyword\ values}\<%
\\
\>[0]\AgdaKeyword{postulate}%
\>[11]\AgdaPostulate{𝐄}%
\>[15]\AgdaSymbol{:}%
\>[18]\AgdaFunction{Domain}%
\>[26]\AgdaComment{--\ expressed\ values}\<%
\\
\>[0]\AgdaKeyword{variable}%
\>[11]\AgdaGeneralizable{ϵ}%
\>[15]\AgdaSymbol{:}%
\>[18]\AgdaPostulate{𝐄}\<%
\\
\>[0]\AgdaFunction{𝐒}%
\>[15]\AgdaSymbol{:}%
\>[18]\AgdaFunction{Domain}%
\>[26]\AgdaComment{--\ stores}\<%
\\
\>[0]\AgdaKeyword{variable}%
\>[11]\AgdaGeneralizable{σ}%
\>[15]\AgdaSymbol{:}%
\>[18]\AgdaFunction{𝐒}\<%
\\
\>[0]\AgdaFunction{𝐔}%
\>[15]\AgdaSymbol{:}%
\>[18]\AgdaFunction{Domain}%
\>[26]\AgdaComment{--\ environments}\<%
\\
\>[0]\AgdaKeyword{variable}%
\>[11]\AgdaGeneralizable{ρ}%
\>[15]\AgdaSymbol{:}%
\>[18]\AgdaFunction{𝐔}\<%
\\
\>[0]\AgdaFunction{𝐂}%
\>[15]\AgdaSymbol{:}%
\>[18]\AgdaFunction{Domain}%
\>[26]\AgdaComment{--\ command\ continuations}\<%
\\
\>[0]\AgdaKeyword{variable}%
\>[11]\AgdaGeneralizable{θ}%
\>[15]\AgdaSymbol{:}%
\>[18]\AgdaFunction{𝐂}\<%
\\
\>[0]\AgdaKeyword{postulate}%
\>[11]\AgdaPostulate{𝐀}%
\>[15]\AgdaSymbol{:}%
\>[18]\AgdaFunction{Domain}%
\>[26]\AgdaComment{--\ answers}\<%
\\
%
\\[\AgdaEmptyExtraSkip]%
\>[0]\AgdaFunction{𝐄⋆}%
\>[15]\AgdaSymbol{=}%
\>[18]\AgdaPostulate{𝐄}\AgdaSpace{}%
\AgdaOperator{\AgdaPostulate{⋆}}\<%
\\
\>[0]\AgdaKeyword{variable}%
\>[11]\AgdaGeneralizable{ϵ⋆}%
\>[15]\AgdaSymbol{:}%
\>[18]\AgdaFunction{𝐄⋆}\<%
\\
%
\\[\AgdaEmptyExtraSkip]%
\>[0]\AgdaComment{--\ Domain\ equations}\<%
\\
%
\\[\AgdaEmptyExtraSkip]%
\>[0]\AgdaKeyword{data}\AgdaSpace{}%
\AgdaDatatype{Misc}\AgdaSpace{}%
\AgdaSymbol{:}\AgdaSpace{}%
\AgdaPrimitive{Set}\AgdaSpace{}%
\AgdaKeyword{where}\AgdaSpace{}%
\AgdaInductiveConstructor{null}\AgdaSpace{}%
\AgdaInductiveConstructor{unallocated}\AgdaSpace{}%
\AgdaInductiveConstructor{undefined}\AgdaSpace{}%
\AgdaInductiveConstructor{unspecified}\AgdaSpace{}%
\AgdaSymbol{:}\AgdaSpace{}%
\AgdaDatatype{Misc}\<%
\\
%
\\[\AgdaEmptyExtraSkip]%
\>[0]\AgdaFunction{𝐍}%
\>[6]\AgdaSymbol{=}%
\>[9]\AgdaFunction{Nat⊥}\<%
\\
\>[0]\AgdaFunction{𝐓}%
\>[6]\AgdaSymbol{=}%
\>[9]\AgdaFunction{Bool⊥}\<%
\\
\>[0]\AgdaFunction{𝐑}%
\>[6]\AgdaSymbol{=}%
\>[9]\AgdaDatatype{Int}\AgdaSpace{}%
\AgdaOperator{\AgdaPostulate{+⊥}}\<%
\\
\>[0]\AgdaFunction{𝐏}%
\>[6]\AgdaSymbol{=}%
\>[9]\AgdaPostulate{𝐋}\AgdaSpace{}%
\AgdaOperator{\AgdaPostulate{×}}\AgdaSpace{}%
\AgdaPostulate{𝐋}\<%
\\
\>[0]\AgdaFunction{𝐌}%
\>[6]\AgdaSymbol{=}%
\>[9]\AgdaDatatype{Misc}\AgdaSpace{}%
\AgdaOperator{\AgdaPostulate{+⊥}}\<%
\\
\>[0]\AgdaFunction{𝐅}%
\>[6]\AgdaSymbol{=}%
\>[9]\AgdaFunction{𝐄⋆}\AgdaSpace{}%
\AgdaSymbol{→}\AgdaSpace{}%
\AgdaSymbol{(}\AgdaPostulate{𝐄}\AgdaSpace{}%
\AgdaSymbol{→}\AgdaSpace{}%
\AgdaFunction{𝐂}\AgdaSymbol{)}\AgdaSpace{}%
\AgdaSymbol{→}\AgdaSpace{}%
\AgdaFunction{𝐂}\<%
\\
\>[0]\AgdaFunction{𝐐}%
\>[6]\AgdaSymbol{=}%
\>[9]\AgdaFunction{Ide}\AgdaSpace{}%
\AgdaOperator{\AgdaPostulate{+⊥}}\<%
\\
\>[0]\AgdaFunction{𝐗}%
\>[6]\AgdaSymbol{=}%
\>[9]\AgdaDatatype{Key}\AgdaSpace{}%
\AgdaOperator{\AgdaPostulate{+⊥}}\<%
\\
\>[0]\AgdaComment{--\ 𝐄\ \ =\ \ 𝐓\ +\ 𝐑\ +\ 𝐏\ +\ 𝐌\ +\ 𝐅\ +\ 𝐐\ +\ 𝐗}\<%
\\
\>[0]\AgdaFunction{𝐒}%
\>[6]\AgdaSymbol{=}%
\>[9]\AgdaPostulate{𝐋}\AgdaSpace{}%
\AgdaSymbol{→}\AgdaSpace{}%
\AgdaPostulate{𝐄}\<%
\\
\>[0]\AgdaFunction{𝐔}%
\>[6]\AgdaSymbol{=}%
\>[9]\AgdaFunction{Ide}\AgdaSpace{}%
\AgdaSymbol{→}\AgdaSpace{}%
\AgdaPostulate{𝐋}\<%
\\
\>[0]\AgdaFunction{𝐂}%
\>[6]\AgdaSymbol{=}%
\>[9]\AgdaFunction{𝐒}\AgdaSpace{}%
\AgdaSymbol{→}\AgdaSpace{}%
\AgdaPostulate{𝐀}\<%
\end{code}
\clearpage
\begin{code}%
\>[0]\AgdaComment{--\ Injections,\ tests,\ and\ projections}\<%
\\
%
\\[\AgdaEmptyExtraSkip]%
\>[0]\AgdaKeyword{postulate}\<%
\\
\>[0][@{}l@{\AgdaIndent{0}}]%
\>[2]\AgdaOperator{\AgdaPostulate{\AgdaUnderscore{}𝐓-in-𝐄}}%
\>[13]\AgdaSymbol{:}\AgdaSpace{}%
\AgdaFunction{𝐓}%
\>[19]\AgdaSymbol{→}\AgdaSpace{}%
\AgdaPostulate{𝐄}\<%
\\
%
\>[2]\AgdaOperator{\AgdaPostulate{\AgdaUnderscore{}∈-𝐓}}%
\>[13]\AgdaSymbol{:}\AgdaSpace{}%
\AgdaPostulate{𝐄}%
\>[19]\AgdaSymbol{→}\AgdaSpace{}%
\AgdaDatatype{Bool}\AgdaSpace{}%
\AgdaOperator{\AgdaPostulate{+⊥}}\<%
\\
%
\>[2]\AgdaOperator{\AgdaPostulate{\AgdaUnderscore{}|-𝐓}}%
\>[13]\AgdaSymbol{:}\AgdaSpace{}%
\AgdaPostulate{𝐄}%
\>[19]\AgdaSymbol{→}\AgdaSpace{}%
\AgdaFunction{𝐓}\<%
\\
%
\\[\AgdaEmptyExtraSkip]%
%
\>[2]\AgdaOperator{\AgdaPostulate{\AgdaUnderscore{}𝐑-in-𝐄}}%
\>[13]\AgdaSymbol{:}\AgdaSpace{}%
\AgdaFunction{𝐑}%
\>[19]\AgdaSymbol{→}\AgdaSpace{}%
\AgdaPostulate{𝐄}\<%
\\
%
\>[2]\AgdaOperator{\AgdaPostulate{\AgdaUnderscore{}∈-𝐑}}%
\>[13]\AgdaSymbol{:}\AgdaSpace{}%
\AgdaPostulate{𝐄}%
\>[19]\AgdaSymbol{→}\AgdaSpace{}%
\AgdaDatatype{Bool}\AgdaSpace{}%
\AgdaOperator{\AgdaPostulate{+⊥}}\<%
\\
%
\>[2]\AgdaOperator{\AgdaPostulate{\AgdaUnderscore{}|-𝐑}}%
\>[13]\AgdaSymbol{:}\AgdaSpace{}%
\AgdaPostulate{𝐄}%
\>[19]\AgdaSymbol{→}\AgdaSpace{}%
\AgdaFunction{𝐑}\<%
\\
%
\\[\AgdaEmptyExtraSkip]%
%
\>[2]\AgdaOperator{\AgdaPostulate{\AgdaUnderscore{}𝐏-in-𝐄}}%
\>[13]\AgdaSymbol{:}\AgdaSpace{}%
\AgdaFunction{𝐏}%
\>[18]\AgdaSymbol{→}\AgdaSpace{}%
\AgdaPostulate{𝐄}\<%
\\
%
\>[2]\AgdaOperator{\AgdaPostulate{\AgdaUnderscore{}∈-𝐏}}%
\>[13]\AgdaSymbol{:}\AgdaSpace{}%
\AgdaPostulate{𝐄}%
\>[19]\AgdaSymbol{→}\AgdaSpace{}%
\AgdaDatatype{Bool}\AgdaSpace{}%
\AgdaOperator{\AgdaPostulate{+⊥}}\<%
\\
%
\>[2]\AgdaOperator{\AgdaPostulate{\AgdaUnderscore{}|-𝐏}}%
\>[13]\AgdaSymbol{:}\AgdaSpace{}%
\AgdaPostulate{𝐄}%
\>[19]\AgdaSymbol{→}\AgdaSpace{}%
\AgdaFunction{𝐏}\<%
\\
%
\\[\AgdaEmptyExtraSkip]%
%
\>[2]\AgdaOperator{\AgdaPostulate{\AgdaUnderscore{}𝐌-in-𝐄}}%
\>[13]\AgdaSymbol{:}\AgdaSpace{}%
\AgdaFunction{𝐌}%
\>[19]\AgdaSymbol{→}\AgdaSpace{}%
\AgdaPostulate{𝐄}\<%
\\
%
\>[2]\AgdaOperator{\AgdaPostulate{\AgdaUnderscore{}∈-𝐌}}%
\>[13]\AgdaSymbol{:}\AgdaSpace{}%
\AgdaPostulate{𝐄}%
\>[19]\AgdaSymbol{→}\AgdaSpace{}%
\AgdaDatatype{Bool}\AgdaSpace{}%
\AgdaOperator{\AgdaPostulate{+⊥}}\<%
\\
%
\>[2]\AgdaOperator{\AgdaPostulate{\AgdaUnderscore{}|-𝐌}}%
\>[13]\AgdaSymbol{:}\AgdaSpace{}%
\AgdaPostulate{𝐄}%
\>[19]\AgdaSymbol{→}\AgdaSpace{}%
\AgdaFunction{𝐌}\<%
\\
%
\\[\AgdaEmptyExtraSkip]%
%
\>[2]\AgdaOperator{\AgdaPostulate{\AgdaUnderscore{}𝐅-in-𝐄}}%
\>[13]\AgdaSymbol{:}\AgdaSpace{}%
\AgdaFunction{𝐅}%
\>[19]\AgdaSymbol{→}\AgdaSpace{}%
\AgdaPostulate{𝐄}\<%
\\
%
\>[2]\AgdaOperator{\AgdaPostulate{\AgdaUnderscore{}∈-𝐅}}%
\>[13]\AgdaSymbol{:}\AgdaSpace{}%
\AgdaPostulate{𝐄}%
\>[19]\AgdaSymbol{→}\AgdaSpace{}%
\AgdaDatatype{Bool}\AgdaSpace{}%
\AgdaOperator{\AgdaPostulate{+⊥}}\<%
\\
%
\>[2]\AgdaOperator{\AgdaPostulate{\AgdaUnderscore{}|-𝐅}}%
\>[13]\AgdaSymbol{:}\AgdaSpace{}%
\AgdaPostulate{𝐄}%
\>[19]\AgdaSymbol{→}\AgdaSpace{}%
\AgdaFunction{𝐅}\<%
\\
%
\\[\AgdaEmptyExtraSkip]%
%
\>[2]\AgdaOperator{\AgdaPostulate{\AgdaUnderscore{}𝐐-in-𝐄}}%
\>[13]\AgdaSymbol{:}\AgdaSpace{}%
\AgdaFunction{𝐐}%
\>[19]\AgdaSymbol{→}\AgdaSpace{}%
\AgdaPostulate{𝐄}\<%
\\
%
\>[2]\AgdaOperator{\AgdaPostulate{\AgdaUnderscore{}∈-𝐐}}%
\>[13]\AgdaSymbol{:}\AgdaSpace{}%
\AgdaPostulate{𝐄}%
\>[19]\AgdaSymbol{→}\AgdaSpace{}%
\AgdaDatatype{Bool}\AgdaSpace{}%
\AgdaOperator{\AgdaPostulate{+⊥}}\<%
\\
%
\>[2]\AgdaOperator{\AgdaPostulate{\AgdaUnderscore{}|-𝐐}}%
\>[13]\AgdaSymbol{:}\AgdaSpace{}%
\AgdaPostulate{𝐄}%
\>[19]\AgdaSymbol{→}\AgdaSpace{}%
\AgdaFunction{𝐐}\<%
\\
%
\\[\AgdaEmptyExtraSkip]%
%
\>[2]\AgdaOperator{\AgdaPostulate{\AgdaUnderscore{}𝐗-in-𝐄}}%
\>[13]\AgdaSymbol{:}\AgdaSpace{}%
\AgdaFunction{𝐗}%
\>[19]\AgdaSymbol{→}\AgdaSpace{}%
\AgdaPostulate{𝐄}\<%
\\
%
\>[2]\AgdaOperator{\AgdaPostulate{\AgdaUnderscore{}∈-𝐗}}%
\>[13]\AgdaSymbol{:}\AgdaSpace{}%
\AgdaPostulate{𝐄}%
\>[19]\AgdaSymbol{→}\AgdaSpace{}%
\AgdaDatatype{Bool}\AgdaSpace{}%
\AgdaOperator{\AgdaPostulate{+⊥}}\<%
\\
%
\>[2]\AgdaOperator{\AgdaPostulate{\AgdaUnderscore{}|-𝐗}}%
\>[13]\AgdaSymbol{:}\AgdaSpace{}%
\AgdaPostulate{𝐄}%
\>[19]\AgdaSymbol{→}\AgdaSpace{}%
\AgdaFunction{𝐗}\<%
\\
%
\\[\AgdaEmptyExtraSkip]%
\>[0]\AgdaComment{--\ Operations\ on\ flat\ domains}\<%
\\
%
\\[\AgdaEmptyExtraSkip]%
\>[0]\AgdaKeyword{postulate}\<%
\\
\>[0][@{}l@{\AgdaIndent{0}}]%
\>[2]\AgdaOperator{\AgdaPostulate{\AgdaUnderscore{}==ᴸ\AgdaUnderscore{}}}%
\>[9]\AgdaSymbol{:}\AgdaSpace{}%
\AgdaPostulate{𝐋}\AgdaSpace{}%
\AgdaSymbol{→}\AgdaSpace{}%
\AgdaPostulate{𝐋}\AgdaSpace{}%
\AgdaSymbol{→}\AgdaSpace{}%
\AgdaFunction{𝐓}\<%
\\
%
\>[2]\AgdaOperator{\AgdaPostulate{\AgdaUnderscore{}==ᴹ\AgdaUnderscore{}}}%
\>[9]\AgdaSymbol{:}\AgdaSpace{}%
\AgdaFunction{𝐌}\AgdaSpace{}%
\AgdaSymbol{→}\AgdaSpace{}%
\AgdaFunction{𝐌}\AgdaSpace{}%
\AgdaSymbol{→}\AgdaSpace{}%
\AgdaFunction{𝐓}\<%
\\
%
\>[2]\AgdaOperator{\AgdaPostulate{\AgdaUnderscore{}==ᵀ\AgdaUnderscore{}}}%
\>[9]\AgdaSymbol{:}\AgdaSpace{}%
\AgdaFunction{𝐓}\AgdaSpace{}%
\AgdaSymbol{→}\AgdaSpace{}%
\AgdaFunction{𝐓}\AgdaSpace{}%
\AgdaSymbol{→}\AgdaSpace{}%
\AgdaFunction{𝐓}\<%
\end{code}

\clearpage

\section{Semantic Functions}
\begin{code}%
\>[0]\AgdaKeyword{module}\AgdaSpace{}%
\AgdaModule{ScmQE.Semantic-Functions}\AgdaSpace{}%
\AgdaKeyword{where}\<%
\\
%
\\[\AgdaEmptyExtraSkip]%
\>[0]\AgdaKeyword{open}\AgdaSpace{}%
\AgdaKeyword{import}\AgdaSpace{}%
\AgdaModule{Notation}\<%
\\
\>[0]\AgdaKeyword{open}\AgdaSpace{}%
\AgdaKeyword{import}\AgdaSpace{}%
\AgdaModule{ScmQE.Abstract-Syntax}\<%
\\
\>[0]\AgdaKeyword{open}\AgdaSpace{}%
\AgdaKeyword{import}\AgdaSpace{}%
\AgdaModule{ScmQE.Domain-Equations}\<%
\\
\>[0]\AgdaKeyword{open}\AgdaSpace{}%
\AgdaKeyword{import}\AgdaSpace{}%
\AgdaModule{ScmQE.Auxiliary-Functions}\<%
\\
%
\\[\AgdaEmptyExtraSkip]%
\>[0]\AgdaOperator{\AgdaFunction{𝒦⟦\AgdaUnderscore{}⟧}}%
\>[8]\AgdaSymbol{:}\AgdaSpace{}%
\AgdaDatatype{Con}\AgdaSpace{}%
\AgdaSymbol{→}\AgdaSpace{}%
\AgdaPostulate{𝐄}\<%
\\
\>[0]\AgdaOperator{\AgdaFunction{𝒟⟦\AgdaUnderscore{}⟧}}%
\>[8]\AgdaSymbol{:}\AgdaSpace{}%
\AgdaDatatype{Dat}\AgdaSpace{}%
\AgdaSymbol{→}\AgdaSpace{}%
\AgdaSymbol{(}\AgdaPostulate{𝐄}\AgdaSpace{}%
\AgdaSymbol{→}\AgdaSpace{}%
\AgdaFunction{𝐂}\AgdaSymbol{)}\AgdaSpace{}%
\AgdaSymbol{→}\AgdaSpace{}%
\AgdaFunction{𝐂}\<%
\\
\>[0]\AgdaOperator{\AgdaFunction{𝒟⋆⟦\AgdaUnderscore{}⟧}}%
\>[8]\AgdaSymbol{:}\AgdaSpace{}%
\AgdaDatatype{Dat⋆}\AgdaSpace{}%
\AgdaSymbol{→}\AgdaSpace{}%
\AgdaSymbol{(}\AgdaPostulate{𝐄}\AgdaSpace{}%
\AgdaSymbol{→}\AgdaSpace{}%
\AgdaFunction{𝐂}\AgdaSymbol{)}\AgdaSpace{}%
\AgdaSymbol{→}\AgdaSpace{}%
\AgdaFunction{𝐂}\<%
\\
\>[0]\AgdaOperator{\AgdaFunction{𝒟⁺⟦\AgdaUnderscore{}⟧}}%
\>[8]\AgdaSymbol{:}\AgdaSpace{}%
\AgdaDatatype{Dat⁺}\AgdaSpace{}%
\AgdaSymbol{→}\AgdaSpace{}%
\AgdaPostulate{𝐄}\AgdaSpace{}%
\AgdaSymbol{→}\AgdaSpace{}%
\AgdaSymbol{(}\AgdaPostulate{𝐄}\AgdaSpace{}%
\AgdaSymbol{→}\AgdaSpace{}%
\AgdaFunction{𝐂}\AgdaSymbol{)}\AgdaSpace{}%
\AgdaSymbol{→}\AgdaSpace{}%
\AgdaFunction{𝐂}\<%
\\
%
\\[\AgdaEmptyExtraSkip]%
\>[0]\AgdaOperator{\AgdaFunction{ℰ⟦\AgdaUnderscore{}⟧}}%
\>[8]\AgdaSymbol{:}\AgdaSpace{}%
\AgdaDatatype{Exp}\AgdaSpace{}%
\AgdaSymbol{→}\AgdaSpace{}%
\AgdaFunction{𝐔}\AgdaSpace{}%
\AgdaSymbol{→}\AgdaSpace{}%
\AgdaSymbol{(}\AgdaPostulate{𝐄}\AgdaSpace{}%
\AgdaSymbol{→}\AgdaSpace{}%
\AgdaFunction{𝐂}\AgdaSymbol{)}\AgdaSpace{}%
\AgdaSymbol{→}\AgdaSpace{}%
\AgdaFunction{𝐂}\<%
\\
\>[0]\AgdaOperator{\AgdaFunction{ℰ⋆⟦\AgdaUnderscore{}⟧}}%
\>[8]\AgdaSymbol{:}\AgdaSpace{}%
\AgdaDatatype{Exp⋆}\AgdaSpace{}%
\AgdaSymbol{→}\AgdaSpace{}%
\AgdaFunction{𝐔}\AgdaSpace{}%
\AgdaSymbol{→}\AgdaSpace{}%
\AgdaSymbol{(}\AgdaFunction{𝐄⋆}\AgdaSpace{}%
\AgdaSymbol{→}\AgdaSpace{}%
\AgdaFunction{𝐂}\AgdaSymbol{)}\AgdaSpace{}%
\AgdaSymbol{→}\AgdaSpace{}%
\AgdaFunction{𝐂}\<%
\\
\>[0]\AgdaOperator{\AgdaFunction{ℱ\AgdaUnderscore{}⟦\AgdaUnderscore{}⟧}}%
\>[8]\AgdaSymbol{:}\AgdaSpace{}%
\AgdaSymbol{(}\AgdaDatatype{Exp}\AgdaSpace{}%
\AgdaSymbol{→}\AgdaSpace{}%
\AgdaFunction{𝐔}\AgdaSpace{}%
\AgdaSymbol{→}\AgdaSpace{}%
\AgdaSymbol{(}\AgdaPostulate{𝐄}\AgdaSpace{}%
\AgdaSymbol{→}\AgdaSpace{}%
\AgdaFunction{𝐂}\AgdaSymbol{)}\AgdaSpace{}%
\AgdaSymbol{→}\AgdaSpace{}%
\AgdaFunction{𝐂}\AgdaSymbol{)}\AgdaSpace{}%
\AgdaSymbol{→}\AgdaSpace{}%
\AgdaDatatype{Exp}\AgdaSpace{}%
\AgdaSymbol{→}\AgdaSpace{}%
\AgdaFunction{𝐔}\AgdaSpace{}%
\AgdaSymbol{→}\AgdaSpace{}%
\AgdaSymbol{(}\AgdaPostulate{𝐄}\AgdaSpace{}%
\AgdaSymbol{→}\AgdaSpace{}%
\AgdaFunction{𝐂}\AgdaSymbol{)}\AgdaSpace{}%
\AgdaSymbol{→}\AgdaSpace{}%
\AgdaFunction{𝐂}\<%
\\
\>[0]\AgdaOperator{\AgdaFunction{ℱ⋆\AgdaUnderscore{}⟦\AgdaUnderscore{}⟧}}%
\>[8]\AgdaSymbol{:}\AgdaSpace{}%
\AgdaSymbol{(}\AgdaDatatype{Exp}\AgdaSpace{}%
\AgdaSymbol{→}\AgdaSpace{}%
\AgdaFunction{𝐔}\AgdaSpace{}%
\AgdaSymbol{→}\AgdaSpace{}%
\AgdaSymbol{(}\AgdaPostulate{𝐄}\AgdaSpace{}%
\AgdaSymbol{→}\AgdaSpace{}%
\AgdaFunction{𝐂}\AgdaSymbol{)}\AgdaSpace{}%
\AgdaSymbol{→}\AgdaSpace{}%
\AgdaFunction{𝐂}\AgdaSymbol{)}\AgdaSpace{}%
\AgdaSymbol{→}\AgdaSpace{}%
\AgdaDatatype{Exp⋆}\AgdaSpace{}%
\AgdaSymbol{→}\AgdaSpace{}%
\AgdaFunction{𝐔}\AgdaSpace{}%
\AgdaSymbol{→}\AgdaSpace{}%
\AgdaSymbol{(}\AgdaPostulate{𝐄}\AgdaSpace{}%
\AgdaOperator{\AgdaPostulate{⋆}}\AgdaSpace{}%
\AgdaSymbol{→}\AgdaSpace{}%
\AgdaFunction{𝐂}\AgdaSymbol{)}\AgdaSpace{}%
\AgdaSymbol{→}\AgdaSpace{}%
\AgdaFunction{𝐂}\<%
\\
%
\\[\AgdaEmptyExtraSkip]%
\>[0]\AgdaOperator{\AgdaFunction{ℬ⟦\AgdaUnderscore{}⟧}}%
\>[8]\AgdaSymbol{:}\AgdaSpace{}%
\AgdaDatatype{Body}\AgdaSpace{}%
\AgdaSymbol{→}\AgdaSpace{}%
\AgdaFunction{𝐔}\AgdaSpace{}%
\AgdaSymbol{→}\AgdaSpace{}%
\AgdaSymbol{(}\AgdaFunction{𝐔}\AgdaSpace{}%
\AgdaSymbol{→}\AgdaSpace{}%
\AgdaFunction{𝐂}\AgdaSymbol{)}\AgdaSpace{}%
\AgdaSymbol{→}\AgdaSpace{}%
\AgdaFunction{𝐂}\<%
\\
\>[0]\AgdaOperator{\AgdaFunction{ℬ⁺⟦\AgdaUnderscore{}⟧}}%
\>[8]\AgdaSymbol{:}\AgdaSpace{}%
\AgdaDatatype{Body⁺}\AgdaSpace{}%
\AgdaSymbol{→}\AgdaSpace{}%
\AgdaFunction{𝐔}\AgdaSpace{}%
\AgdaSymbol{→}\AgdaSpace{}%
\AgdaSymbol{(}\AgdaFunction{𝐔}\AgdaSpace{}%
\AgdaSymbol{→}\AgdaSpace{}%
\AgdaFunction{𝐂}\AgdaSymbol{)}\AgdaSpace{}%
\AgdaSymbol{→}\AgdaSpace{}%
\AgdaFunction{𝐂}\<%
\\
\>[0]\AgdaOperator{\AgdaFunction{𝒫⟦\AgdaUnderscore{}⟧}}%
\>[8]\AgdaSymbol{:}\AgdaSpace{}%
\AgdaDatatype{Prog}\AgdaSpace{}%
\AgdaSymbol{→}\AgdaSpace{}%
\AgdaPostulate{𝐀}\<%
\\
%
\\[\AgdaEmptyExtraSkip]%
\>[0]\AgdaComment{--\ Constant\ denotations\ 𝒦⟦\ K\ ⟧\ :\ 𝐄}\<%
\\
%
\\[\AgdaEmptyExtraSkip]%
\>[0]\AgdaOperator{\AgdaFunction{𝒦⟦}}\AgdaSpace{}%
\AgdaInductiveConstructor{int}\AgdaSpace{}%
\AgdaBound{Z}\AgdaSpace{}%
\AgdaOperator{\AgdaFunction{⟧}}%
\>[12]\AgdaSymbol{=}\AgdaSpace{}%
\AgdaPostulate{η}\AgdaSpace{}%
\AgdaBound{Z}\AgdaSpace{}%
\AgdaOperator{\AgdaPostulate{𝐑-in-𝐄}}\<%
\\
\>[0]\AgdaOperator{\AgdaFunction{𝒦⟦}}\AgdaSpace{}%
\AgdaInductiveConstructor{\#t}\AgdaSpace{}%
\AgdaOperator{\AgdaFunction{⟧}}%
\>[12]\AgdaSymbol{=}\AgdaSpace{}%
\AgdaPostulate{η}\AgdaSpace{}%
\AgdaInductiveConstructor{true}\AgdaSpace{}%
\AgdaOperator{\AgdaPostulate{𝐓-in-𝐄}}\<%
\\
\>[0]\AgdaOperator{\AgdaFunction{𝒦⟦}}\AgdaSpace{}%
\AgdaInductiveConstructor{\#f}\AgdaSpace{}%
\AgdaOperator{\AgdaFunction{⟧}}%
\>[12]\AgdaSymbol{=}\AgdaSpace{}%
\AgdaPostulate{η}\AgdaSpace{}%
\AgdaInductiveConstructor{false}\AgdaSpace{}%
\AgdaOperator{\AgdaPostulate{𝐓-in-𝐄}}\<%
\\
%
\\[\AgdaEmptyExtraSkip]%
\>[0]\AgdaComment{--\ Datum\ denotations\ 𝒟⟦\ Δ\ ⟧\ :\ (𝐄\ →\ 𝐂)\ →\ 𝐂}\<%
\\
%
\\[\AgdaEmptyExtraSkip]%
\>[0]\AgdaOperator{\AgdaFunction{𝒟⟦}}\AgdaSpace{}%
\AgdaInductiveConstructor{con}\AgdaSpace{}%
\AgdaBound{K}\AgdaSpace{}%
\AgdaOperator{\AgdaFunction{⟧}}\AgdaSpace{}%
\AgdaBound{κ}%
\>[20]\AgdaSymbol{=}\AgdaSpace{}%
\AgdaBound{κ}\AgdaSymbol{(}\AgdaOperator{\AgdaFunction{𝒦⟦}}\AgdaSpace{}%
\AgdaBound{K}\AgdaSpace{}%
\AgdaOperator{\AgdaFunction{⟧}}\AgdaSymbol{)}\<%
\\
\>[0]\AgdaOperator{\AgdaFunction{𝒟⟦}}\AgdaSpace{}%
\AgdaInductiveConstructor{ide}\AgdaSpace{}%
\AgdaBound{I}\AgdaSpace{}%
\AgdaOperator{\AgdaFunction{⟧}}\AgdaSpace{}%
\AgdaBound{κ}%
\>[20]\AgdaSymbol{=}\AgdaSpace{}%
\AgdaBound{κ}\AgdaSymbol{(}\AgdaPostulate{η}\AgdaSpace{}%
\AgdaBound{I}\AgdaSpace{}%
\AgdaOperator{\AgdaPostulate{𝐐-in-𝐄}}\AgdaSymbol{)}\<%
\\
\>[0]\AgdaOperator{\AgdaFunction{𝒟⟦}}\AgdaSpace{}%
\AgdaInductiveConstructor{key}\AgdaSpace{}%
\AgdaBound{X}\AgdaSpace{}%
\AgdaOperator{\AgdaFunction{⟧}}\AgdaSpace{}%
\AgdaBound{κ}%
\>[20]\AgdaSymbol{=}\AgdaSpace{}%
\AgdaBound{κ}\AgdaSymbol{(}\AgdaPostulate{η}\AgdaSpace{}%
\AgdaBound{X}\AgdaSpace{}%
\AgdaOperator{\AgdaPostulate{𝐗-in-𝐄}}\AgdaSymbol{)}\<%
\\
\>[0]\AgdaOperator{\AgdaFunction{𝒟⟦}}\AgdaSpace{}%
\AgdaInductiveConstructor{′}\AgdaSpace{}%
\AgdaBound{Δ}\AgdaSpace{}%
\AgdaOperator{\AgdaFunction{⟧}}\AgdaSpace{}%
\AgdaBound{κ}%
\>[20]\AgdaSymbol{=}\AgdaSpace{}%
\AgdaOperator{\AgdaFunction{𝒟⟦}}\AgdaSpace{}%
\AgdaBound{Δ}\AgdaSpace{}%
\AgdaOperator{\AgdaFunction{⟧}}\AgdaSpace{}%
\AgdaBound{κ}\<%
\\
\>[0]\AgdaOperator{\AgdaFunction{𝒟⟦}}\AgdaSpace{}%
\AgdaOperator{\AgdaInductiveConstructor{⦅}}\AgdaSpace{}%
\AgdaBound{Δ⋆}\AgdaSpace{}%
\AgdaOperator{\AgdaInductiveConstructor{⦆}}\AgdaSpace{}%
\AgdaOperator{\AgdaFunction{⟧}}\AgdaSpace{}%
\AgdaBound{κ}%
\>[20]\AgdaSymbol{=}\AgdaSpace{}%
\AgdaOperator{\AgdaFunction{𝒟⋆⟦}}\AgdaSpace{}%
\AgdaBound{Δ⋆}\AgdaSpace{}%
\AgdaOperator{\AgdaFunction{⟧}}\AgdaSpace{}%
\AgdaBound{κ}\<%
\\
\>[0]\AgdaOperator{\AgdaFunction{𝒟⟦}}\AgdaSpace{}%
\AgdaOperator{\AgdaInductiveConstructor{⦅}}\AgdaSpace{}%
\AgdaBound{Δ⁺}\AgdaSpace{}%
\AgdaOperator{\AgdaInductiveConstructor{·}}\AgdaSpace{}%
\AgdaBound{Δ}\AgdaSpace{}%
\AgdaOperator{\AgdaInductiveConstructor{⦆}}\AgdaSpace{}%
\AgdaOperator{\AgdaFunction{⟧}}\AgdaSpace{}%
\AgdaBound{κ}%
\>[20]\AgdaSymbol{=}\AgdaSpace{}%
\AgdaOperator{\AgdaFunction{𝒟⟦}}\AgdaSpace{}%
\AgdaBound{Δ}\AgdaSpace{}%
\AgdaOperator{\AgdaFunction{⟧}}\AgdaSpace{}%
\AgdaSymbol{(λ}\AgdaSpace{}%
\AgdaBound{ϵ}\AgdaSpace{}%
\AgdaSymbol{→}\AgdaSpace{}%
\AgdaOperator{\AgdaFunction{𝒟⁺⟦}}\AgdaSpace{}%
\AgdaBound{Δ⁺}\AgdaSpace{}%
\AgdaOperator{\AgdaFunction{⟧}}\AgdaSpace{}%
\AgdaBound{ϵ}\AgdaSpace{}%
\AgdaBound{κ}\AgdaSymbol{)}\<%
\\
\>[0]\AgdaOperator{\AgdaFunction{𝒟⟦}}\AgdaSpace{}%
\AgdaInductiveConstructor{\#proc}\AgdaSpace{}%
\AgdaOperator{\AgdaFunction{⟧}}\AgdaSpace{}%
\AgdaBound{κ}%
\>[20]\AgdaSymbol{=}\AgdaSpace{}%
\AgdaPostulate{⊥}\<%
\\
%
\\[\AgdaEmptyExtraSkip]%
\>[0]\AgdaComment{--\ Datum\ sequence\ denotations\ 𝒟⋆⟦\ Δ⋆\ ⟧\ :\ (𝐄\ →\ 𝐂)\ →\ 𝐂}\<%
\\
%
\\[\AgdaEmptyExtraSkip]%
\>[0]\AgdaOperator{\AgdaFunction{𝒟⋆⟦}}\AgdaSpace{}%
\AgdaInductiveConstructor{␣␣␣}\AgdaSpace{}%
\AgdaOperator{\AgdaFunction{⟧}}\AgdaSpace{}%
\AgdaBound{κ}\AgdaSpace{}%
\AgdaSymbol{=}\AgdaSpace{}%
\AgdaBound{κ}\AgdaSpace{}%
\AgdaSymbol{(}\AgdaPostulate{η}\AgdaSpace{}%
\AgdaInductiveConstructor{null}\AgdaSpace{}%
\AgdaOperator{\AgdaPostulate{𝐌-in-𝐄}}\AgdaSymbol{)}\<%
\\
%
\\[\AgdaEmptyExtraSkip]%
\>[0]\AgdaOperator{\AgdaFunction{𝒟⋆⟦}}\AgdaSpace{}%
\AgdaBound{Δ₁}\AgdaSpace{}%
\AgdaOperator{\AgdaInductiveConstructor{␣␣}}\AgdaSpace{}%
\AgdaBound{Δ⋆}\AgdaSpace{}%
\AgdaOperator{\AgdaFunction{⟧}}\AgdaSpace{}%
\AgdaBound{κ}\AgdaSpace{}%
\AgdaSymbol{=}\<%
\\
\>[0][@{}l@{\AgdaIndent{0}}]%
\>[2]\AgdaOperator{\AgdaFunction{𝒟⟦}}\AgdaSpace{}%
\AgdaBound{Δ₁}\AgdaSpace{}%
\AgdaOperator{\AgdaFunction{⟧}}\AgdaSpace{}%
\AgdaSymbol{(λ}\AgdaSpace{}%
\AgdaBound{ϵ₁}\AgdaSpace{}%
\AgdaSymbol{→}\<%
\\
\>[2][@{}l@{\AgdaIndent{0}}]%
\>[4]\AgdaOperator{\AgdaFunction{𝒟⋆⟦}}\AgdaSpace{}%
\AgdaBound{Δ⋆}\AgdaSpace{}%
\AgdaOperator{\AgdaFunction{⟧}}\AgdaSpace{}%
\AgdaSymbol{(λ}\AgdaSpace{}%
\AgdaBound{ϵ}\AgdaSpace{}%
\AgdaSymbol{→}\<%
\\
\>[4][@{}l@{\AgdaIndent{0}}]%
\>[6]\AgdaFunction{cons}\AgdaSpace{}%
\AgdaOperator{\AgdaPostulate{⟨}}\AgdaSpace{}%
\AgdaBound{ϵ₁}\AgdaSpace{}%
\AgdaOperator{\AgdaPostulate{,}}\AgdaSpace{}%
\AgdaBound{ϵ}\AgdaSpace{}%
\AgdaOperator{\AgdaPostulate{⟩}}\AgdaSpace{}%
\AgdaBound{κ}\AgdaSymbol{))}\<%
\\
%
\\[\AgdaEmptyExtraSkip]%
\>[0]\AgdaComment{--\ Datum\ prefix\ sequence\ denotations\ 𝒟⁺⟦\ Δ⁺\ ⟧\ :\ 𝐄\ →\ (𝐄\ →\ 𝐂)\ →\ 𝐂}\<%
\\
%
\\[\AgdaEmptyExtraSkip]%
\>[0]\AgdaOperator{\AgdaFunction{𝒟⁺⟦}}\AgdaSpace{}%
\AgdaOperator{\AgdaInductiveConstructor{␣␣}}\AgdaSpace{}%
\AgdaBound{Δ₁}\AgdaSpace{}%
\AgdaOperator{\AgdaFunction{⟧}}\AgdaSpace{}%
\AgdaBound{ϵ}\AgdaSpace{}%
\AgdaBound{κ}\AgdaSpace{}%
\AgdaSymbol{=}\<%
\\
\>[0][@{}l@{\AgdaIndent{0}}]%
\>[2]\AgdaOperator{\AgdaFunction{𝒟⟦}}\AgdaSpace{}%
\AgdaBound{Δ₁}\AgdaSpace{}%
\AgdaOperator{\AgdaFunction{⟧}}\AgdaSpace{}%
\AgdaSymbol{(λ}\AgdaSpace{}%
\AgdaBound{ϵ₁}\AgdaSpace{}%
\AgdaSymbol{→}\<%
\\
\>[2][@{}l@{\AgdaIndent{0}}]%
\>[4]\AgdaFunction{cons}\AgdaSpace{}%
\AgdaOperator{\AgdaPostulate{⟨}}\AgdaSpace{}%
\AgdaBound{ϵ₁}\AgdaSpace{}%
\AgdaOperator{\AgdaPostulate{,}}\AgdaSpace{}%
\AgdaBound{ϵ}\AgdaSpace{}%
\AgdaOperator{\AgdaPostulate{⟩}}\AgdaSpace{}%
\AgdaBound{κ}\AgdaSymbol{)}\<%
\\
%
\\[\AgdaEmptyExtraSkip]%
\>[0]\AgdaOperator{\AgdaFunction{𝒟⁺⟦}}\AgdaSpace{}%
\AgdaBound{Δ⁺}\AgdaSpace{}%
\AgdaOperator{\AgdaInductiveConstructor{␣␣}}\AgdaSpace{}%
\AgdaBound{Δ₁}\AgdaSpace{}%
\AgdaOperator{\AgdaFunction{⟧}}\AgdaSpace{}%
\AgdaBound{ϵ}\AgdaSpace{}%
\AgdaBound{κ}\AgdaSpace{}%
\AgdaSymbol{=}\<%
\\
\>[0][@{}l@{\AgdaIndent{0}}]%
\>[2]\AgdaOperator{\AgdaFunction{𝒟⟦}}\AgdaSpace{}%
\AgdaBound{Δ₁}\AgdaSpace{}%
\AgdaOperator{\AgdaFunction{⟧}}\AgdaSpace{}%
\AgdaSymbol{(λ}\AgdaSpace{}%
\AgdaBound{ϵ₁}\AgdaSpace{}%
\AgdaSymbol{→}\<%
\\
\>[2][@{}l@{\AgdaIndent{0}}]%
\>[4]\AgdaFunction{cons}\AgdaSpace{}%
\AgdaOperator{\AgdaPostulate{⟨}}\AgdaSpace{}%
\AgdaBound{ϵ₁}\AgdaSpace{}%
\AgdaOperator{\AgdaPostulate{,}}\AgdaSpace{}%
\AgdaBound{ϵ}\AgdaSpace{}%
\AgdaOperator{\AgdaPostulate{⟩}}\AgdaSpace{}%
\AgdaSymbol{(λ}\AgdaSpace{}%
\AgdaBound{ϵ′}\AgdaSpace{}%
\AgdaSymbol{→}\<%
\\
%
\>[4]\AgdaOperator{\AgdaFunction{𝒟⁺⟦}}\AgdaSpace{}%
\AgdaBound{Δ⁺}\AgdaSpace{}%
\AgdaOperator{\AgdaFunction{⟧}}\AgdaSpace{}%
\AgdaBound{ϵ′}\AgdaSpace{}%
\AgdaBound{κ}\AgdaSymbol{))}\<%
\end{code}
\clearpage
\begin{code}%
\>[0]\AgdaComment{--\ Fixed\ expression\ denotations\ ℰ⟦\ E\ ⟧\ :\ 𝐔\ →\ (𝐄\ →\ 𝐂)\ →\ 𝐂}\<%
\\
%
\\[\AgdaEmptyExtraSkip]%
\>[0]\AgdaOperator{\AgdaFunction{ℰ⟦}}\AgdaSpace{}%
\AgdaBound{E}\AgdaSpace{}%
\AgdaOperator{\AgdaFunction{⟧}}\AgdaSpace{}%
\AgdaSymbol{=}\AgdaSpace{}%
\AgdaOperator{\AgdaFunction{ℱ}}\AgdaSpace{}%
\AgdaSymbol{(}\AgdaPostulate{fix}\AgdaSpace{}%
\AgdaOperator{\AgdaFunction{ℱ\AgdaUnderscore{}⟦\AgdaUnderscore{}⟧}}\AgdaSymbol{)}\AgdaSpace{}%
\AgdaOperator{\AgdaFunction{⟦}}\AgdaSpace{}%
\AgdaBound{E}\AgdaSpace{}%
\AgdaOperator{\AgdaFunction{⟧}}\<%
\\
%
\\[\AgdaEmptyExtraSkip]%
\>[0]\AgdaComment{--\ Fixed\ expression\ sequence\ denotations\ ℰ⋆⟦\AgdaUnderscore{}⟧\ \ :\ Exp⋆\ →\ 𝐔\ →\ (𝐄⋆\ →\ 𝐂)\ →\ 𝐂}\<%
\\
%
\\[\AgdaEmptyExtraSkip]%
\>[0]\AgdaOperator{\AgdaFunction{ℰ⋆⟦}}\AgdaSpace{}%
\AgdaBound{E⋆}\AgdaSpace{}%
\AgdaOperator{\AgdaFunction{⟧}}\AgdaSpace{}%
\AgdaSymbol{=}\AgdaSpace{}%
\AgdaOperator{\AgdaFunction{ℱ⋆}}\AgdaSpace{}%
\AgdaSymbol{(}\AgdaPostulate{fix}\AgdaSpace{}%
\AgdaOperator{\AgdaFunction{ℱ\AgdaUnderscore{}⟦\AgdaUnderscore{}⟧}}\AgdaSymbol{)}\AgdaSpace{}%
\AgdaOperator{\AgdaFunction{⟦}}\AgdaSpace{}%
\AgdaBound{E⋆}\AgdaSpace{}%
\AgdaOperator{\AgdaFunction{⟧}}\<%
\\
%
\\[\AgdaEmptyExtraSkip]%
\>[0]\AgdaComment{--\ Expression\ denotations\ ℱ\ ℰ′\ ⟦\ E\ ⟧\ :\ 𝐔\ →\ (𝐄\ →\ 𝐂)\ →\ 𝐂}\<%
\\
%
\\[\AgdaEmptyExtraSkip]%
\>[0]\AgdaOperator{\AgdaFunction{ℱ}}\AgdaSpace{}%
\AgdaBound{ℰ′}\AgdaSpace{}%
\AgdaOperator{\AgdaFunction{⟦}}\AgdaSpace{}%
\AgdaInductiveConstructor{con}\AgdaSpace{}%
\AgdaBound{K}\AgdaSpace{}%
\AgdaOperator{\AgdaFunction{⟧}}\AgdaSpace{}%
\AgdaBound{ρ}\AgdaSpace{}%
\AgdaBound{κ}\AgdaSpace{}%
\AgdaSymbol{=}\AgdaSpace{}%
\AgdaBound{κ}\AgdaSpace{}%
\AgdaSymbol{(}\AgdaOperator{\AgdaFunction{𝒦⟦}}\AgdaSpace{}%
\AgdaBound{K}\AgdaSpace{}%
\AgdaOperator{\AgdaFunction{⟧}}\AgdaSymbol{)}\<%
\\
%
\\[\AgdaEmptyExtraSkip]%
\>[0]\AgdaOperator{\AgdaFunction{ℱ}}\AgdaSpace{}%
\AgdaBound{ℰ′}\AgdaSpace{}%
\AgdaOperator{\AgdaFunction{⟦}}\AgdaSpace{}%
\AgdaInductiveConstructor{ide}\AgdaSpace{}%
\AgdaBound{I}\AgdaSpace{}%
\AgdaOperator{\AgdaFunction{⟧}}\AgdaSpace{}%
\AgdaBound{ρ}\AgdaSpace{}%
\AgdaBound{κ}\AgdaSpace{}%
\AgdaSymbol{=}\AgdaSpace{}%
\AgdaFunction{hold}\AgdaSpace{}%
\AgdaSymbol{(}\AgdaBound{ρ}\AgdaSpace{}%
\AgdaBound{I}\AgdaSymbol{)}\AgdaSpace{}%
\AgdaBound{κ}\<%
\\
%
\\[\AgdaEmptyExtraSkip]%
\>[0]\AgdaOperator{\AgdaFunction{ℱ}}%
\>[303I]\AgdaBound{ℰ′}\AgdaSpace{}%
\AgdaOperator{\AgdaFunction{⟦}}\AgdaSpace{}%
\AgdaOperator{\AgdaInductiveConstructor{⦅}}\AgdaSpace{}%
\AgdaBound{E}\AgdaSpace{}%
\AgdaOperator{\AgdaInductiveConstructor{␣}}\AgdaSpace{}%
\AgdaBound{E⋆}\AgdaSpace{}%
\AgdaOperator{\AgdaInductiveConstructor{⦆}}\AgdaSpace{}%
\AgdaOperator{\AgdaFunction{⟧}}\AgdaSpace{}%
\AgdaBound{ρ}\AgdaSpace{}%
\AgdaBound{κ}\AgdaSpace{}%
\AgdaSymbol{=}\<%
\\
\>[.][@{}l@{}]\<[303I]%
\>[2]\AgdaOperator{\AgdaFunction{ℱ}}%
\>[314I]\AgdaBound{ℰ′}\AgdaSpace{}%
\AgdaOperator{\AgdaFunction{⟦}}\AgdaSpace{}%
\AgdaBound{E}\AgdaSpace{}%
\AgdaOperator{\AgdaFunction{⟧}}\AgdaSpace{}%
\AgdaBound{ρ}\AgdaSpace{}%
\AgdaSymbol{(λ}\AgdaSpace{}%
\AgdaBound{ϵ}\AgdaSpace{}%
\AgdaSymbol{→}\<%
\\
\>[.][@{}l@{}]\<[314I]%
\>[4]\AgdaOperator{\AgdaFunction{ℱ⋆}}\AgdaSpace{}%
\AgdaBound{ℰ′}\AgdaSpace{}%
\AgdaOperator{\AgdaFunction{⟦}}\AgdaSpace{}%
\AgdaBound{E⋆}\AgdaSpace{}%
\AgdaOperator{\AgdaFunction{⟧}}\AgdaSpace{}%
\AgdaBound{ρ}\AgdaSpace{}%
\AgdaSymbol{(λ}\AgdaSpace{}%
\AgdaBound{ϵ⋆}\AgdaSpace{}%
\AgdaSymbol{→}\<%
\\
\>[4][@{}l@{\AgdaIndent{0}}]%
\>[6]\AgdaSymbol{(}\AgdaBound{ϵ}\AgdaSpace{}%
\AgdaOperator{\AgdaPostulate{|-𝐅}}\AgdaSymbol{)}\AgdaSpace{}%
\AgdaBound{ϵ⋆}\AgdaSpace{}%
\AgdaBound{κ}\AgdaSymbol{))}\<%
\\
%
\\[\AgdaEmptyExtraSkip]%
\>[0]\AgdaOperator{\AgdaFunction{ℱ}}%
\>[333I]\AgdaBound{ℰ′}\AgdaSpace{}%
\AgdaOperator{\AgdaFunction{⟦}}\AgdaSpace{}%
\AgdaOperator{\AgdaInductiveConstructor{⦅lambda}}\AgdaSpace{}%
\AgdaBound{I}\AgdaSpace{}%
\AgdaOperator{\AgdaInductiveConstructor{␣}}\AgdaSpace{}%
\AgdaBound{E}\AgdaSpace{}%
\AgdaOperator{\AgdaInductiveConstructor{⦆}}\AgdaSpace{}%
\AgdaOperator{\AgdaFunction{⟧}}\AgdaSpace{}%
\AgdaBound{ρ}\AgdaSpace{}%
\AgdaBound{κ}\AgdaSpace{}%
\AgdaSymbol{=}\<%
\\
\>[.][@{}l@{}]\<[333I]%
\>[2]\AgdaBound{κ}\AgdaSpace{}%
\AgdaSymbol{(}%
\>[7]\AgdaSymbol{(λ}%
\>[345I]\AgdaBound{ϵ⋆}\AgdaSpace{}%
\AgdaBound{κ′}\AgdaSpace{}%
\AgdaSymbol{→}\<%
\\
\>[.][@{}l@{}]\<[345I]%
\>[10]\AgdaFunction{list}\AgdaSpace{}%
\AgdaBound{ϵ⋆}\AgdaSpace{}%
\AgdaSymbol{(λ}\AgdaSpace{}%
\AgdaBound{ϵ}\AgdaSpace{}%
\AgdaSymbol{→}\<%
\\
\>[10][@{}l@{\AgdaIndent{0}}]%
\>[12]\AgdaFunction{alloc}\AgdaSpace{}%
\AgdaBound{ϵ}\AgdaSpace{}%
\AgdaSymbol{(λ}\AgdaSpace{}%
\AgdaBound{α}\AgdaSpace{}%
\AgdaSymbol{→}\<%
\\
\>[12][@{}l@{\AgdaIndent{0}}]%
\>[14]\AgdaOperator{\AgdaFunction{ℱ}}\AgdaSpace{}%
\AgdaBound{ℰ′}\AgdaSpace{}%
\AgdaOperator{\AgdaFunction{⟦}}\AgdaSpace{}%
\AgdaBound{E}\AgdaSpace{}%
\AgdaOperator{\AgdaFunction{⟧}}\AgdaSpace{}%
\AgdaSymbol{(}\AgdaBound{ρ}\AgdaSpace{}%
\AgdaOperator{\AgdaFunction{[}}\AgdaSpace{}%
\AgdaBound{α}\AgdaSpace{}%
\AgdaOperator{\AgdaFunction{/}}\AgdaSpace{}%
\AgdaBound{I}\AgdaSpace{}%
\AgdaOperator{\AgdaFunction{]}}\AgdaSymbol{)}\AgdaSpace{}%
\AgdaBound{κ′}\AgdaSymbol{))}\<%
\\
%
\>[7]\AgdaSymbol{)}\AgdaSpace{}%
\AgdaOperator{\AgdaPostulate{𝐅-in-𝐄}}\AgdaSymbol{)}\<%
\\
%
\\[\AgdaEmptyExtraSkip]%
\>[0]\AgdaOperator{\AgdaFunction{ℱ}}%
\>[368I]\AgdaBound{ℰ′}\AgdaSpace{}%
\AgdaOperator{\AgdaFunction{⟦}}\AgdaSpace{}%
\AgdaOperator{\AgdaInductiveConstructor{⦅if}}\AgdaSpace{}%
\AgdaBound{E}\AgdaSpace{}%
\AgdaOperator{\AgdaInductiveConstructor{␣}}\AgdaSpace{}%
\AgdaBound{E₁}\AgdaSpace{}%
\AgdaOperator{\AgdaInductiveConstructor{␣}}\AgdaSpace{}%
\AgdaBound{E₂}\AgdaSpace{}%
\AgdaOperator{\AgdaInductiveConstructor{⦆}}\AgdaSpace{}%
\AgdaOperator{\AgdaFunction{⟧}}\AgdaSpace{}%
\AgdaBound{ρ}\AgdaSpace{}%
\AgdaBound{κ}\AgdaSpace{}%
\AgdaSymbol{=}\<%
\\
\>[.][@{}l@{}]\<[368I]%
\>[2]\AgdaOperator{\AgdaFunction{ℱ}}%
\>[381I]\AgdaBound{ℰ′}\AgdaSpace{}%
\AgdaOperator{\AgdaFunction{⟦}}\AgdaSpace{}%
\AgdaBound{E}\AgdaSpace{}%
\AgdaOperator{\AgdaFunction{⟧}}\AgdaSpace{}%
\AgdaBound{ρ}\AgdaSpace{}%
\AgdaSymbol{(λ}\AgdaSpace{}%
\AgdaBound{ϵ}\AgdaSpace{}%
\AgdaSymbol{→}\<%
\\
\>[.][@{}l@{}]\<[381I]%
\>[4]\AgdaFunction{truish}\AgdaSpace{}%
\AgdaBound{ϵ}\AgdaSpace{}%
\AgdaOperator{\AgdaPostulate{⟶}}\AgdaSpace{}%
\AgdaOperator{\AgdaFunction{ℱ}}\AgdaSpace{}%
\AgdaBound{ℰ′}\AgdaSpace{}%
\AgdaOperator{\AgdaFunction{⟦}}\AgdaSpace{}%
\AgdaBound{E₁}\AgdaSpace{}%
\AgdaOperator{\AgdaFunction{⟧}}\AgdaSpace{}%
\AgdaBound{ρ}\AgdaSpace{}%
\AgdaBound{κ}\AgdaSpace{}%
\AgdaOperator{\AgdaPostulate{,}}\AgdaSpace{}%
\AgdaOperator{\AgdaFunction{ℱ}}\AgdaSpace{}%
\AgdaBound{ℰ′}\AgdaSpace{}%
\AgdaOperator{\AgdaFunction{⟦}}\AgdaSpace{}%
\AgdaBound{E₂}\AgdaSpace{}%
\AgdaOperator{\AgdaFunction{⟧}}\AgdaSpace{}%
\AgdaBound{ρ}\AgdaSpace{}%
\AgdaBound{κ}\AgdaSymbol{)}\<%
\\
%
\\[\AgdaEmptyExtraSkip]%
\>[0]\AgdaOperator{\AgdaFunction{ℱ}}%
\>[406I]\AgdaBound{ℰ′}\AgdaSpace{}%
\AgdaOperator{\AgdaFunction{⟦}}\AgdaSpace{}%
\AgdaOperator{\AgdaInductiveConstructor{⦅set!}}\AgdaSpace{}%
\AgdaBound{I}\AgdaSpace{}%
\AgdaOperator{\AgdaInductiveConstructor{␣}}\AgdaSpace{}%
\AgdaBound{E}\AgdaSpace{}%
\AgdaOperator{\AgdaInductiveConstructor{⦆}}\AgdaSpace{}%
\AgdaOperator{\AgdaFunction{⟧}}\AgdaSpace{}%
\AgdaBound{ρ}\AgdaSpace{}%
\AgdaBound{κ}\AgdaSpace{}%
\AgdaSymbol{=}\<%
\\
\>[.][@{}l@{}]\<[406I]%
\>[2]\AgdaOperator{\AgdaFunction{ℱ}}%
\>[417I]\AgdaBound{ℰ′}\AgdaSpace{}%
\AgdaOperator{\AgdaFunction{⟦}}\AgdaSpace{}%
\AgdaBound{E}\AgdaSpace{}%
\AgdaOperator{\AgdaFunction{⟧}}\AgdaSpace{}%
\AgdaBound{ρ}\AgdaSpace{}%
\AgdaSymbol{(λ}\AgdaSpace{}%
\AgdaBound{ϵ}\AgdaSpace{}%
\AgdaSymbol{→}\<%
\\
\>[.][@{}l@{}]\<[417I]%
\>[4]\AgdaFunction{assign}\AgdaSpace{}%
\AgdaSymbol{(}\AgdaBound{ρ}\AgdaSpace{}%
\AgdaBound{I}\AgdaSymbol{)}\AgdaSpace{}%
\AgdaBound{ϵ}\AgdaSpace{}%
\AgdaSymbol{(}\<%
\\
\>[4][@{}l@{\AgdaIndent{0}}]%
\>[6]\AgdaBound{κ}\AgdaSpace{}%
\AgdaSymbol{(}\AgdaPostulate{η}\AgdaSpace{}%
\AgdaInductiveConstructor{unspecified}\AgdaSpace{}%
\AgdaOperator{\AgdaPostulate{𝐌-in-𝐄}}\AgdaSymbol{)))}\<%
\\
%
\\[\AgdaEmptyExtraSkip]%
\>[0]\AgdaOperator{\AgdaFunction{ℱ}}\AgdaSpace{}%
\AgdaBound{ℰ′}\AgdaSpace{}%
\AgdaOperator{\AgdaFunction{⟦}}\AgdaSpace{}%
\AgdaOperator{\AgdaInductiveConstructor{⦅quote}}\AgdaSpace{}%
\AgdaBound{Δ}\AgdaSpace{}%
\AgdaOperator{\AgdaInductiveConstructor{⦆}}\AgdaSpace{}%
\AgdaOperator{\AgdaFunction{⟧}}\AgdaSpace{}%
\AgdaBound{ρ}\AgdaSpace{}%
\AgdaBound{κ}\AgdaSpace{}%
\AgdaSymbol{=}\AgdaSpace{}%
\AgdaOperator{\AgdaFunction{𝒟⟦}}\AgdaSpace{}%
\AgdaBound{Δ}\AgdaSpace{}%
\AgdaOperator{\AgdaFunction{⟧}}\AgdaSpace{}%
\AgdaBound{κ}\<%
\\
%
\\[\AgdaEmptyExtraSkip]%
\>[0]\AgdaOperator{\AgdaFunction{ℱ}}%
\>[445I]\AgdaBound{ℰ′}\AgdaSpace{}%
\AgdaOperator{\AgdaFunction{⟦}}\AgdaSpace{}%
\AgdaOperator{\AgdaInductiveConstructor{⦅eval}}\AgdaSpace{}%
\AgdaBound{E}\AgdaSpace{}%
\AgdaOperator{\AgdaInductiveConstructor{⦆}}\AgdaSpace{}%
\AgdaOperator{\AgdaFunction{⟧}}\AgdaSpace{}%
\AgdaBound{ρ}\AgdaSpace{}%
\AgdaBound{κ}\AgdaSpace{}%
\AgdaSymbol{=}\<%
\\
\>[.][@{}l@{}]\<[445I]%
\>[2]\AgdaOperator{\AgdaFunction{ℱ}}%
\>[454I]\AgdaBound{ℰ′}\AgdaSpace{}%
\AgdaOperator{\AgdaFunction{⟦}}\AgdaSpace{}%
\AgdaBound{E}\AgdaSpace{}%
\AgdaOperator{\AgdaFunction{⟧}}\AgdaSpace{}%
\AgdaBound{ρ}\AgdaSpace{}%
\AgdaSymbol{(λ}\AgdaSpace{}%
\AgdaBound{ϵ}\AgdaSpace{}%
\AgdaSymbol{→}\<%
\\
\>[.][@{}l@{}]\<[454I]%
\>[4]\AgdaFunction{datum}\AgdaSpace{}%
\AgdaBound{ϵ}\AgdaSpace{}%
\AgdaSymbol{(λ}\AgdaSpace{}%
\AgdaBound{Δ}\AgdaSpace{}%
\AgdaSymbol{→}\AgdaSpace{}%
\AgdaBound{ℰ′}\AgdaSpace{}%
\AgdaSymbol{(}\AgdaOperator{\AgdaFunction{exp⟦}}\AgdaSpace{}%
\AgdaBound{Δ}\AgdaSpace{}%
\AgdaOperator{\AgdaFunction{⟧}}\AgdaSymbol{)}\AgdaSpace{}%
\AgdaPostulate{nullenv}\AgdaSpace{}%
\AgdaBound{κ}\AgdaSymbol{))}\<%
\\
%
\\[\AgdaEmptyExtraSkip]%
\>[0]\AgdaOperator{\AgdaFunction{ℱ}}\AgdaSpace{}%
\AgdaBound{ℰ′}\AgdaSpace{}%
\AgdaOperator{\AgdaFunction{⟦}}\AgdaSpace{}%
\AgdaInductiveConstructor{⦅␣⦆}\AgdaSpace{}%
\AgdaOperator{\AgdaFunction{⟧}}\AgdaSpace{}%
\AgdaBound{ρ}\AgdaSpace{}%
\AgdaBound{κ}\AgdaSpace{}%
\AgdaSymbol{=}\AgdaSpace{}%
\AgdaPostulate{⊥}\<%
\\
%
\\[\AgdaEmptyExtraSkip]%
\>[0]\AgdaComment{--\ Expression\ sequence\ denotations\ ℱ⋆\ ℰ′\ ⟦\ E⋆\ ⟧\ :\ 𝐔\ →\ (𝐄⋆\ →\ 𝐂)\ →\ 𝐂}\<%
\\
%
\\[\AgdaEmptyExtraSkip]%
\>[0]\AgdaOperator{\AgdaFunction{ℱ⋆}}\AgdaSpace{}%
\AgdaBound{ℰ′}\AgdaSpace{}%
\AgdaOperator{\AgdaFunction{⟦}}\AgdaSpace{}%
\AgdaInductiveConstructor{␣␣␣}\AgdaSpace{}%
\AgdaOperator{\AgdaFunction{⟧}}\AgdaSpace{}%
\AgdaBound{ρ}\AgdaSpace{}%
\AgdaBound{κ}\AgdaSpace{}%
\AgdaSymbol{=}\AgdaSpace{}%
\AgdaBound{κ}\AgdaSpace{}%
\AgdaPostulate{⟨⟩}\<%
\\
%
\\[\AgdaEmptyExtraSkip]%
\>[0]\AgdaOperator{\AgdaFunction{ℱ⋆}}\AgdaSpace{}%
\AgdaBound{ℰ′}\AgdaSpace{}%
\AgdaOperator{\AgdaFunction{⟦}}\AgdaSpace{}%
\AgdaBound{E}\AgdaSpace{}%
\AgdaOperator{\AgdaInductiveConstructor{␣␣}}\AgdaSpace{}%
\AgdaBound{E⋆}\AgdaSpace{}%
\AgdaOperator{\AgdaFunction{⟧}}\AgdaSpace{}%
\AgdaBound{ρ}\AgdaSpace{}%
\AgdaBound{κ}\AgdaSpace{}%
\AgdaSymbol{=}\<%
\\
\>[0][@{}l@{\AgdaIndent{0}}]%
\>[2]\AgdaOperator{\AgdaFunction{ℱ}}%
\>[498I]\AgdaBound{ℰ′}\AgdaSpace{}%
\AgdaOperator{\AgdaFunction{⟦}}\AgdaSpace{}%
\AgdaBound{E}\AgdaSpace{}%
\AgdaOperator{\AgdaFunction{⟧}}\AgdaSpace{}%
\AgdaBound{ρ}\AgdaSpace{}%
\AgdaSymbol{(λ}\AgdaSpace{}%
\AgdaBound{ϵ}\AgdaSpace{}%
\AgdaSymbol{→}\<%
\\
\>[.][@{}l@{}]\<[498I]%
\>[4]\AgdaOperator{\AgdaFunction{ℱ⋆}}\AgdaSpace{}%
\AgdaBound{ℰ′}\AgdaSpace{}%
\AgdaOperator{\AgdaFunction{⟦}}\AgdaSpace{}%
\AgdaBound{E⋆}\AgdaSpace{}%
\AgdaOperator{\AgdaFunction{⟧}}\AgdaSpace{}%
\AgdaBound{ρ}\AgdaSpace{}%
\AgdaSymbol{(λ}\AgdaSpace{}%
\AgdaBound{ϵ⋆}\AgdaSpace{}%
\AgdaSymbol{→}\<%
\\
\>[4][@{}l@{\AgdaIndent{0}}]%
\>[6]\AgdaBound{κ}\AgdaSpace{}%
\AgdaSymbol{(}\AgdaOperator{\AgdaPostulate{⟨}}\AgdaSpace{}%
\AgdaBound{ϵ}\AgdaSpace{}%
\AgdaOperator{\AgdaPostulate{⟩}}\AgdaSpace{}%
\AgdaOperator{\AgdaPostulate{§}}\AgdaSpace{}%
\AgdaBound{ϵ⋆}\AgdaSymbol{)))}\<%
\end{code}
\clearpage
\begin{code}%
\>[0]\AgdaComment{--\ Body\ denotations\ ℬ⟦\ B\ ⟧\ :\ 𝐔\ →\ (𝐔\ →\ 𝐂)\ →\ 𝐂}\<%
\\
%
\\[\AgdaEmptyExtraSkip]%
\>[0]\AgdaOperator{\AgdaFunction{ℬ⟦}}\AgdaSpace{}%
\AgdaOperator{\AgdaInductiveConstructor{␣␣}}\AgdaSpace{}%
\AgdaBound{E}\AgdaSpace{}%
\AgdaOperator{\AgdaFunction{⟧}}\AgdaSpace{}%
\AgdaBound{ρ}\AgdaSpace{}%
\AgdaBound{κ}\AgdaSpace{}%
\AgdaSymbol{=}\AgdaSpace{}%
\AgdaOperator{\AgdaFunction{ℰ⟦}}\AgdaSpace{}%
\AgdaBound{E}\AgdaSpace{}%
\AgdaOperator{\AgdaFunction{⟧}}\AgdaSpace{}%
\AgdaBound{ρ}\AgdaSpace{}%
\AgdaSymbol{(λ}\AgdaSpace{}%
\AgdaBound{ϵ}\AgdaSpace{}%
\AgdaSymbol{→}\AgdaSpace{}%
\AgdaBound{κ}\AgdaSpace{}%
\AgdaBound{ρ}\AgdaSymbol{)}\<%
\\
%
\\[\AgdaEmptyExtraSkip]%
\>[0]\AgdaOperator{\AgdaFunction{ℬ⟦}}\AgdaSpace{}%
\AgdaOperator{\AgdaInductiveConstructor{⦅define}}\AgdaSpace{}%
\AgdaBound{I}\AgdaSpace{}%
\AgdaOperator{\AgdaInductiveConstructor{␣}}\AgdaSpace{}%
\AgdaBound{E}\AgdaSpace{}%
\AgdaOperator{\AgdaInductiveConstructor{⦆}}\AgdaSpace{}%
\AgdaOperator{\AgdaFunction{⟧}}\AgdaSpace{}%
\AgdaBound{ρ}\AgdaSpace{}%
\AgdaBound{κ}\AgdaSpace{}%
\AgdaSymbol{=}\<%
\\
\>[0][@{}l@{\AgdaIndent{0}}]%
\>[2]\AgdaOperator{\AgdaFunction{ℰ⟦}}\AgdaSpace{}%
\AgdaBound{E}\AgdaSpace{}%
\AgdaOperator{\AgdaFunction{⟧}}\AgdaSpace{}%
\AgdaBound{ρ}\AgdaSpace{}%
\AgdaSymbol{(λ}\AgdaSpace{}%
\AgdaBound{ϵ}\AgdaSpace{}%
\AgdaSymbol{→}%
\>[549I]\AgdaSymbol{(}\AgdaBound{ρ}%
\>[550I]\AgdaBound{I}\AgdaSpace{}%
\AgdaOperator{\AgdaPostulate{==ᴸ}}\AgdaSpace{}%
\AgdaPostulate{unknown}\AgdaSymbol{)}\AgdaSpace{}%
\AgdaOperator{\AgdaPostulate{⟶}}\<%
\\
\>[550I][@{}l@{\AgdaIndent{0}}]%
\>[22]\AgdaFunction{alloc}\AgdaSpace{}%
\AgdaBound{ϵ}\AgdaSpace{}%
\AgdaSymbol{(λ}\AgdaSpace{}%
\AgdaBound{α}\AgdaSpace{}%
\AgdaSymbol{→}\AgdaSpace{}%
\AgdaBound{κ}\AgdaSpace{}%
\AgdaSymbol{(}\AgdaBound{ρ}\AgdaSpace{}%
\AgdaOperator{\AgdaFunction{[}}\AgdaSpace{}%
\AgdaBound{α}\AgdaSpace{}%
\AgdaOperator{\AgdaFunction{/}}\AgdaSpace{}%
\AgdaBound{I}\AgdaSpace{}%
\AgdaOperator{\AgdaFunction{]}}\AgdaSymbol{))}\AgdaOperator{\AgdaPostulate{,}}\<%
\\
\>[549I][@{}l@{\AgdaIndent{0}}]%
\>[20]\AgdaFunction{assign}\AgdaSpace{}%
\AgdaSymbol{(}\AgdaBound{ρ}\AgdaSpace{}%
\AgdaBound{I}\AgdaSymbol{)}\AgdaSpace{}%
\AgdaBound{ϵ}\AgdaSpace{}%
\AgdaSymbol{(}\AgdaBound{κ}\AgdaSpace{}%
\AgdaBound{ρ}\AgdaSymbol{))}\<%
\\
%
\\[\AgdaEmptyExtraSkip]%
\>[0]\AgdaOperator{\AgdaFunction{ℬ⟦}}\AgdaSpace{}%
\AgdaOperator{\AgdaInductiveConstructor{⦅begin}}\AgdaSpace{}%
\AgdaBound{B⁺}\AgdaSpace{}%
\AgdaOperator{\AgdaInductiveConstructor{⦆}}\AgdaSpace{}%
\AgdaOperator{\AgdaFunction{⟧}}\AgdaSpace{}%
\AgdaBound{ρ}\AgdaSpace{}%
\AgdaBound{κ}\AgdaSpace{}%
\AgdaSymbol{=}\AgdaSpace{}%
\AgdaOperator{\AgdaFunction{ℬ⁺⟦}}\AgdaSpace{}%
\AgdaBound{B⁺}\AgdaSpace{}%
\AgdaOperator{\AgdaFunction{⟧}}\AgdaSpace{}%
\AgdaBound{ρ}\AgdaSpace{}%
\AgdaBound{κ}\<%
\\
%
\\[\AgdaEmptyExtraSkip]%
\>[0]\AgdaComment{--\ Body\ sequence\ denotations\ ℬ⁺⟦\ B⁺\ ⟧\ :\ 𝐔\ →\ (𝐔\ →\ 𝐂)\ →\ 𝐂}\<%
\\
%
\\[\AgdaEmptyExtraSkip]%
\>[0]\AgdaOperator{\AgdaFunction{ℬ⁺⟦}}\AgdaSpace{}%
\AgdaOperator{\AgdaInductiveConstructor{␣␣}}\AgdaSpace{}%
\AgdaBound{B}\AgdaSpace{}%
\AgdaOperator{\AgdaFunction{⟧}}\AgdaSpace{}%
\AgdaBound{ρ}\AgdaSpace{}%
\AgdaBound{κ}\AgdaSpace{}%
\AgdaSymbol{=}\AgdaSpace{}%
\AgdaOperator{\AgdaFunction{ℬ⟦}}\AgdaSpace{}%
\AgdaBound{B}\AgdaSpace{}%
\AgdaOperator{\AgdaFunction{⟧}}\AgdaSpace{}%
\AgdaBound{ρ}\AgdaSpace{}%
\AgdaBound{κ}\<%
\\
%
\\[\AgdaEmptyExtraSkip]%
\>[0]\AgdaOperator{\AgdaFunction{ℬ⁺⟦}}\AgdaSpace{}%
\AgdaBound{B}\AgdaSpace{}%
\AgdaOperator{\AgdaInductiveConstructor{␣␣}}\AgdaSpace{}%
\AgdaBound{B⁺}\AgdaSpace{}%
\AgdaOperator{\AgdaFunction{⟧}}\AgdaSpace{}%
\AgdaBound{ρ}\AgdaSpace{}%
\AgdaBound{κ}\AgdaSpace{}%
\AgdaSymbol{=}\AgdaSpace{}%
\AgdaOperator{\AgdaFunction{ℬ⟦}}\AgdaSpace{}%
\AgdaBound{B}\AgdaSpace{}%
\AgdaOperator{\AgdaFunction{⟧}}\AgdaSpace{}%
\AgdaBound{ρ}\AgdaSpace{}%
\AgdaSymbol{(λ}\AgdaSpace{}%
\AgdaBound{ρ′}\AgdaSpace{}%
\AgdaSymbol{→}\AgdaSpace{}%
\AgdaOperator{\AgdaFunction{ℬ⁺⟦}}\AgdaSpace{}%
\AgdaBound{B⁺}\AgdaSpace{}%
\AgdaOperator{\AgdaFunction{⟧}}\AgdaSpace{}%
\AgdaBound{ρ′}\AgdaSpace{}%
\AgdaBound{κ}\AgdaSymbol{)}\<%
\\
%
\\[\AgdaEmptyExtraSkip]%
\>[0]\AgdaComment{--\ Program\ denotations\ 𝒫⟦\ Π\ ⟧\ :\ 𝐀}\<%
\\
%
\\[\AgdaEmptyExtraSkip]%
\>[0]\AgdaOperator{\AgdaFunction{𝒫⟦}}\AgdaSpace{}%
\AgdaInductiveConstructor{␣␣␣}\AgdaSpace{}%
\AgdaOperator{\AgdaFunction{⟧}}\AgdaSpace{}%
\AgdaSymbol{=}\AgdaSpace{}%
\AgdaPostulate{finished}\AgdaSpace{}%
\AgdaFunction{initial-store}\<%
\\
%
\\[\AgdaEmptyExtraSkip]%
\>[0]\AgdaOperator{\AgdaFunction{𝒫⟦}}\AgdaSpace{}%
\AgdaOperator{\AgdaInductiveConstructor{␣␣}}\AgdaSpace{}%
\AgdaBound{B⁺}\AgdaSpace{}%
\AgdaOperator{\AgdaFunction{⟧}}\AgdaSpace{}%
\AgdaSymbol{=}\AgdaSpace{}%
\AgdaOperator{\AgdaFunction{ℬ⁺⟦}}\AgdaSpace{}%
\AgdaBound{B⁺}\AgdaSpace{}%
\AgdaOperator{\AgdaFunction{⟧}}\AgdaSpace{}%
\AgdaPostulate{nullenv}\AgdaSpace{}%
\AgdaSymbol{(λ}\AgdaSpace{}%
\AgdaBound{ρ}\AgdaSpace{}%
\AgdaSymbol{→}\AgdaSpace{}%
\AgdaPostulate{finished}\AgdaSymbol{)}\AgdaSpace{}%
\AgdaFunction{initial-store}\<%
\end{code}

\clearpage

\section{Auxiliary Functions}
\begin{code}%
\>[0]\AgdaKeyword{module}\AgdaSpace{}%
\AgdaModule{ScmQE.Auxiliary-Functions}\AgdaSpace{}%
\AgdaKeyword{where}\<%
\\
%
\\[\AgdaEmptyExtraSkip]%
\>[0]\AgdaKeyword{open}\AgdaSpace{}%
\AgdaKeyword{import}\AgdaSpace{}%
\AgdaModule{Notation}\<%
\\
\>[0]\AgdaKeyword{open}\AgdaSpace{}%
\AgdaKeyword{import}\AgdaSpace{}%
\AgdaModule{ScmQE.Abstract-Syntax}\<%
\\
\>[0]\AgdaKeyword{open}\AgdaSpace{}%
\AgdaKeyword{import}\AgdaSpace{}%
\AgdaModule{ScmQE.Domain-Equations}\<%
\\
%
\\[\AgdaEmptyExtraSkip]%
\>[0]\AgdaComment{--\ Environments\ ρ\ :\ 𝐔\ =\ Ide\ →\ 𝐋}\<%
\\
%
\\[\AgdaEmptyExtraSkip]%
\>[0]\AgdaKeyword{postulate}\AgdaSpace{}%
\AgdaOperator{\AgdaPostulate{\AgdaUnderscore{}==\AgdaUnderscore{}}}\AgdaSpace{}%
\AgdaSymbol{:}\AgdaSpace{}%
\AgdaFunction{Ide}\AgdaSpace{}%
\AgdaSymbol{→}\AgdaSpace{}%
\AgdaFunction{Ide}\AgdaSpace{}%
\AgdaSymbol{→}\AgdaSpace{}%
\AgdaDatatype{Bool}\<%
\\
%
\\[\AgdaEmptyExtraSkip]%
\>[0]\AgdaOperator{\AgdaFunction{\AgdaUnderscore{}[\AgdaUnderscore{}/\AgdaUnderscore{}]}}\AgdaSpace{}%
\AgdaSymbol{:}\AgdaSpace{}%
\AgdaFunction{𝐔}\AgdaSpace{}%
\AgdaSymbol{→}\AgdaSpace{}%
\AgdaPostulate{𝐋}\AgdaSpace{}%
\AgdaSymbol{→}\AgdaSpace{}%
\AgdaFunction{Ide}\AgdaSpace{}%
\AgdaSymbol{→}\AgdaSpace{}%
\AgdaFunction{𝐔}\<%
\\
\>[0]\AgdaBound{ρ}\AgdaSpace{}%
\AgdaOperator{\AgdaFunction{[}}\AgdaSpace{}%
\AgdaBound{α}\AgdaSpace{}%
\AgdaOperator{\AgdaFunction{/}}\AgdaSpace{}%
\AgdaBound{I}\AgdaSpace{}%
\AgdaOperator{\AgdaFunction{]}}\AgdaSpace{}%
\AgdaSymbol{=}\AgdaSpace{}%
\AgdaSymbol{λ}\AgdaSpace{}%
\AgdaBound{I′}\AgdaSpace{}%
\AgdaSymbol{→}\AgdaSpace{}%
\AgdaPostulate{η}\AgdaSpace{}%
\AgdaSymbol{(}\AgdaBound{I}\AgdaSpace{}%
\AgdaOperator{\AgdaPostulate{==}}\AgdaSpace{}%
\AgdaBound{I′}\AgdaSymbol{)}\AgdaSpace{}%
\AgdaOperator{\AgdaPostulate{⟶}}\AgdaSpace{}%
\AgdaBound{α}\AgdaSpace{}%
\AgdaOperator{\AgdaPostulate{,}}\AgdaSpace{}%
\AgdaBound{ρ}\AgdaSpace{}%
\AgdaBound{I′}\<%
\\
%
\\[\AgdaEmptyExtraSkip]%
\>[0]\AgdaKeyword{postulate}\AgdaSpace{}%
\AgdaPostulate{unknown}\AgdaSpace{}%
\AgdaSymbol{:}\AgdaSpace{}%
\AgdaPostulate{𝐋}\<%
\\
\>[0]\AgdaComment{--\ ρ\ I\ =\ unknown\ represents\ the\ lack\ of\ a\ binding\ for\ I\ in\ ρ}\<%
\\
%
\\[\AgdaEmptyExtraSkip]%
\>[0]\AgdaKeyword{postulate}\AgdaSpace{}%
\AgdaPostulate{nullenv}\AgdaSpace{}%
\AgdaSymbol{:}\AgdaSpace{}%
\AgdaFunction{𝐔}\<%
\\
\>[0]\AgdaComment{--\ nullenv\ shoud\ include\ various\ procedures\ and\ values}\<%
\\
%
\\[\AgdaEmptyExtraSkip]%
\>[0]\AgdaComment{--\ Stores\ σ\ :\ 𝐒\ =\ 𝐋\ →\ 𝐄}\<%
\\
%
\\[\AgdaEmptyExtraSkip]%
\>[0]\AgdaOperator{\AgdaFunction{\AgdaUnderscore{}[\AgdaUnderscore{}/\AgdaUnderscore{}]′}}\AgdaSpace{}%
\AgdaSymbol{:}\AgdaSpace{}%
\AgdaFunction{𝐒}\AgdaSpace{}%
\AgdaSymbol{→}\AgdaSpace{}%
\AgdaPostulate{𝐄}\AgdaSpace{}%
\AgdaSymbol{→}\AgdaSpace{}%
\AgdaPostulate{𝐋}\AgdaSpace{}%
\AgdaSymbol{→}\AgdaSpace{}%
\AgdaFunction{𝐒}\<%
\\
\>[0]\AgdaBound{σ}\AgdaSpace{}%
\AgdaOperator{\AgdaFunction{[}}\AgdaSpace{}%
\AgdaBound{ϵ}\AgdaSpace{}%
\AgdaOperator{\AgdaFunction{/}}\AgdaSpace{}%
\AgdaBound{α}\AgdaSpace{}%
\AgdaOperator{\AgdaFunction{]′}}\AgdaSpace{}%
\AgdaSymbol{=}\AgdaSpace{}%
\AgdaSymbol{λ}\AgdaSpace{}%
\AgdaBound{α′}\AgdaSpace{}%
\AgdaSymbol{→}\AgdaSpace{}%
\AgdaSymbol{(}\AgdaBound{α}\AgdaSpace{}%
\AgdaOperator{\AgdaPostulate{==ᴸ}}\AgdaSpace{}%
\AgdaBound{α′}\AgdaSymbol{)}\AgdaSpace{}%
\AgdaOperator{\AgdaPostulate{⟶}}\AgdaSpace{}%
\AgdaBound{ϵ}\AgdaSpace{}%
\AgdaOperator{\AgdaPostulate{,}}\AgdaSpace{}%
\AgdaBound{σ}\AgdaSpace{}%
\AgdaBound{α′}\<%
\\
%
\\[\AgdaEmptyExtraSkip]%
\>[0]\AgdaFunction{assign}\AgdaSpace{}%
\AgdaSymbol{:}\AgdaSpace{}%
\AgdaPostulate{𝐋}\AgdaSpace{}%
\AgdaSymbol{→}\AgdaSpace{}%
\AgdaPostulate{𝐄}\AgdaSpace{}%
\AgdaSymbol{→}\AgdaSpace{}%
\AgdaFunction{𝐂}\AgdaSpace{}%
\AgdaSymbol{→}\AgdaSpace{}%
\AgdaFunction{𝐂}\<%
\\
\>[0]\AgdaFunction{assign}\AgdaSpace{}%
\AgdaSymbol{=}\AgdaSpace{}%
\AgdaSymbol{λ}\AgdaSpace{}%
\AgdaBound{α}\AgdaSpace{}%
\AgdaBound{ϵ}\AgdaSpace{}%
\AgdaBound{θ}\AgdaSpace{}%
\AgdaBound{σ}\AgdaSpace{}%
\AgdaSymbol{→}\AgdaSpace{}%
\AgdaBound{θ}\AgdaSpace{}%
\AgdaSymbol{(}\AgdaBound{σ}\AgdaSpace{}%
\AgdaOperator{\AgdaFunction{[}}\AgdaSpace{}%
\AgdaBound{ϵ}\AgdaSpace{}%
\AgdaOperator{\AgdaFunction{/}}\AgdaSpace{}%
\AgdaBound{α}\AgdaSpace{}%
\AgdaOperator{\AgdaFunction{]′}}\AgdaSymbol{)}\<%
\\
%
\\[\AgdaEmptyExtraSkip]%
\>[0]\AgdaFunction{hold}\AgdaSpace{}%
\AgdaSymbol{:}\AgdaSpace{}%
\AgdaPostulate{𝐋}\AgdaSpace{}%
\AgdaSymbol{→}\AgdaSpace{}%
\AgdaSymbol{(}\AgdaPostulate{𝐄}\AgdaSpace{}%
\AgdaSymbol{→}\AgdaSpace{}%
\AgdaFunction{𝐂}\AgdaSymbol{)}\AgdaSpace{}%
\AgdaSymbol{→}\AgdaSpace{}%
\AgdaFunction{𝐂}\<%
\\
\>[0]\AgdaFunction{hold}\AgdaSpace{}%
\AgdaSymbol{=}\AgdaSpace{}%
\AgdaSymbol{λ}\AgdaSpace{}%
\AgdaBound{α}\AgdaSpace{}%
\AgdaBound{κ}\AgdaSpace{}%
\AgdaBound{σ}\AgdaSpace{}%
\AgdaSymbol{→}\AgdaSpace{}%
\AgdaBound{κ}\AgdaSpace{}%
\AgdaSymbol{(}\AgdaBound{σ}\AgdaSpace{}%
\AgdaBound{α}\AgdaSymbol{)}\AgdaSpace{}%
\AgdaBound{σ}\<%
\\
%
\\[\AgdaEmptyExtraSkip]%
\>[0]\AgdaKeyword{postulate}\AgdaSpace{}%
\AgdaPostulate{new}\AgdaSpace{}%
\AgdaSymbol{:}\AgdaSpace{}%
\AgdaSymbol{(}\AgdaPostulate{𝐋}\AgdaSpace{}%
\AgdaSymbol{→}\AgdaSpace{}%
\AgdaFunction{𝐂}\AgdaSymbol{)}\AgdaSpace{}%
\AgdaSymbol{→}\AgdaSpace{}%
\AgdaFunction{𝐂}\<%
\\
\>[0]\AgdaComment{--\ new\ κ\ σ\ =\ κ\ α\ σ′\ where\ σ\ α\ =\ unallocated,\ σ′\ α\ ≠\ unallocated}\<%
\\
%
\\[\AgdaEmptyExtraSkip]%
\>[0]\AgdaFunction{alloc}\AgdaSpace{}%
\AgdaSymbol{:}\AgdaSpace{}%
\AgdaPostulate{𝐄}\AgdaSpace{}%
\AgdaSymbol{→}\AgdaSpace{}%
\AgdaSymbol{(}\AgdaPostulate{𝐋}\AgdaSpace{}%
\AgdaSymbol{→}\AgdaSpace{}%
\AgdaFunction{𝐂}\AgdaSymbol{)}\AgdaSpace{}%
\AgdaSymbol{→}\AgdaSpace{}%
\AgdaFunction{𝐂}\<%
\\
\>[0]\AgdaFunction{alloc}\AgdaSpace{}%
\AgdaSymbol{=}\AgdaSpace{}%
\AgdaSymbol{λ}\AgdaSpace{}%
\AgdaBound{ϵ}\AgdaSpace{}%
\AgdaBound{κ}\AgdaSpace{}%
\AgdaSymbol{→}\AgdaSpace{}%
\AgdaPostulate{new}\AgdaSpace{}%
\AgdaSymbol{(λ}\AgdaSpace{}%
\AgdaBound{α}\AgdaSpace{}%
\AgdaSymbol{→}\AgdaSpace{}%
\AgdaFunction{assign}\AgdaSpace{}%
\AgdaBound{α}\AgdaSpace{}%
\AgdaBound{ϵ}\AgdaSpace{}%
\AgdaSymbol{(}\AgdaBound{κ}\AgdaSpace{}%
\AgdaBound{α}\AgdaSymbol{))}\<%
\\
\>[0]\AgdaComment{--\ should\ be\ ⊥\ when\ ϵ\ |-𝐌\ ==\ unallocated}\<%
\\
%
\\[\AgdaEmptyExtraSkip]%
\>[0]\AgdaFunction{initial-store}\AgdaSpace{}%
\AgdaSymbol{:}\AgdaSpace{}%
\AgdaFunction{𝐒}\<%
\\
\>[0]\AgdaFunction{initial-store}\AgdaSpace{}%
\AgdaSymbol{=}\AgdaSpace{}%
\AgdaSymbol{λ}\AgdaSpace{}%
\AgdaBound{α}\AgdaSpace{}%
\AgdaSymbol{→}\AgdaSpace{}%
\AgdaPostulate{η}\AgdaSpace{}%
\AgdaInductiveConstructor{unallocated}\AgdaSpace{}%
\AgdaOperator{\AgdaPostulate{𝐌-in-𝐄}}\<%
\\
%
\\[\AgdaEmptyExtraSkip]%
\>[0]\AgdaKeyword{postulate}\AgdaSpace{}%
\AgdaPostulate{finished}\AgdaSpace{}%
\AgdaSymbol{:}\AgdaSpace{}%
\AgdaFunction{𝐂}\<%
\\
\>[0]\AgdaComment{--\ normal\ termination\ with\ answer\ depending\ on\ final\ store}\<%
\\
%
\\[\AgdaEmptyExtraSkip]%
\>[0]\AgdaFunction{truish}\AgdaSpace{}%
\AgdaSymbol{:}\AgdaSpace{}%
\AgdaPostulate{𝐄}\AgdaSpace{}%
\AgdaSymbol{→}\AgdaSpace{}%
\AgdaFunction{𝐓}\<%
\\
\>[0]\AgdaFunction{truish}\AgdaSpace{}%
\AgdaSymbol{=}\<%
\\
\>[0][@{}l@{\AgdaIndent{0}}]%
\>[2]\AgdaSymbol{λ}%
\>[158I]\AgdaBound{ϵ}%
\>[159I]\AgdaSymbol{→}\AgdaSpace{}%
\AgdaSymbol{(}\AgdaBound{ϵ}\AgdaSpace{}%
\AgdaOperator{\AgdaPostulate{∈-𝐓}}\AgdaSymbol{)}\AgdaSpace{}%
\AgdaOperator{\AgdaPostulate{⟶}}\<%
\\
\>[.][@{}l@{}]\<[159I]%
\>[6]\AgdaSymbol{(((}\AgdaBound{ϵ}\AgdaSpace{}%
\AgdaOperator{\AgdaPostulate{|-𝐓}}\AgdaSymbol{)}\AgdaSpace{}%
\AgdaOperator{\AgdaPostulate{==ᵀ}}\AgdaSpace{}%
\AgdaPostulate{η}\AgdaSpace{}%
\AgdaInductiveConstructor{false}\AgdaSymbol{)}\AgdaSpace{}%
\AgdaOperator{\AgdaPostulate{⟶}}\AgdaSpace{}%
\AgdaPostulate{η}\AgdaSpace{}%
\AgdaInductiveConstructor{false}\AgdaSpace{}%
\AgdaOperator{\AgdaPostulate{,}}\AgdaSpace{}%
\AgdaPostulate{η}\AgdaSpace{}%
\AgdaInductiveConstructor{true}\AgdaSymbol{)}\AgdaSpace{}%
\AgdaOperator{\AgdaPostulate{,}}\<%
\\
\>[.][@{}l@{}]\<[158I]%
\>[4]\AgdaPostulate{η}\AgdaSpace{}%
\AgdaInductiveConstructor{true}\<%
\end{code}
\clearpage
\begin{code}%
\>[0]\AgdaComment{--\ Lists}\<%
\\
%
\\[\AgdaEmptyExtraSkip]%
\>[0]\AgdaFunction{cons}\AgdaSpace{}%
\AgdaSymbol{:}\AgdaSpace{}%
\AgdaFunction{𝐅}\<%
\\
\>[0]\AgdaFunction{cons}\AgdaSpace{}%
\AgdaSymbol{=}\<%
\\
\>[0][@{}l@{\AgdaIndent{0}}]%
\>[2]\AgdaSymbol{λ}%
\>[178I]\AgdaBound{ϵ⋆}\AgdaSpace{}%
\AgdaBound{κ}\AgdaSpace{}%
\AgdaSymbol{→}\<%
\\
\>[178I][@{}l@{\AgdaIndent{0}}]%
\>[6]\AgdaSymbol{(}\AgdaPostulate{\#}\AgdaSpace{}%
\AgdaBound{ϵ⋆}\AgdaSpace{}%
\AgdaOperator{\AgdaPostulate{==⊥}}\AgdaSpace{}%
\AgdaNumber{2}\AgdaSymbol{)}\AgdaSpace{}%
\AgdaOperator{\AgdaPostulate{⟶}}%
\>[185I]\AgdaFunction{alloc}\AgdaSpace{}%
\AgdaSymbol{(}\AgdaBound{ϵ⋆}\AgdaSpace{}%
\AgdaOperator{\AgdaPostulate{↓}}\AgdaSpace{}%
\AgdaNumber{1}\AgdaSymbol{)}\AgdaSpace{}%
\AgdaSymbol{(λ}\AgdaSpace{}%
\AgdaBound{α₁}\AgdaSpace{}%
\AgdaSymbol{→}\<%
\\
\>[185I][@{}l@{\AgdaIndent{0}}]%
\>[24]\AgdaFunction{alloc}\AgdaSpace{}%
\AgdaSymbol{(}\AgdaBound{ϵ⋆}\AgdaSpace{}%
\AgdaOperator{\AgdaPostulate{↓}}\AgdaSpace{}%
\AgdaNumber{2}\AgdaSymbol{)}\AgdaSpace{}%
\AgdaSymbol{(λ}\AgdaSpace{}%
\AgdaBound{α₂}\AgdaSpace{}%
\AgdaSymbol{→}\<%
\\
\>[24][@{}l@{\AgdaIndent{0}}]%
\>[26]\AgdaBound{κ}\AgdaSpace{}%
\AgdaSymbol{((}\AgdaBound{α₁}\AgdaSpace{}%
\AgdaOperator{\AgdaPostulate{,}}\AgdaSpace{}%
\AgdaBound{α₂}\AgdaSymbol{)}\AgdaSpace{}%
\AgdaOperator{\AgdaPostulate{𝐏-in-𝐄}}\AgdaSymbol{)))}\AgdaSpace{}%
\AgdaOperator{\AgdaPostulate{,}}\<%
\\
\>[.][@{}l@{}]\<[178I]%
\>[4]\AgdaPostulate{⊥}\<%
\\
%
\\[\AgdaEmptyExtraSkip]%
\>[0]\AgdaFunction{list}\AgdaSpace{}%
\AgdaSymbol{:}\AgdaSpace{}%
\AgdaFunction{𝐅}\<%
\\
\>[0]\AgdaFunction{list}\AgdaSpace{}%
\AgdaSymbol{=}\AgdaSpace{}%
\AgdaPostulate{fix}\AgdaSpace{}%
\AgdaSymbol{λ}\AgdaSpace{}%
\AgdaBound{list′}\AgdaSpace{}%
\AgdaSymbol{→}\<%
\\
\>[0][@{}l@{\AgdaIndent{0}}]%
\>[2]\AgdaSymbol{λ}%
\>[210I]\AgdaBound{ϵ⋆}\AgdaSpace{}%
\AgdaBound{κ}\AgdaSpace{}%
\AgdaSymbol{→}\<%
\\
\>[.][@{}l@{}]\<[210I]%
\>[4]\AgdaSymbol{(}\AgdaPostulate{\#}\AgdaSpace{}%
\AgdaBound{ϵ⋆}\AgdaSpace{}%
\AgdaOperator{\AgdaPostulate{==⊥}}\AgdaSpace{}%
\AgdaNumber{0}\AgdaSymbol{)}\AgdaSpace{}%
\AgdaOperator{\AgdaPostulate{⟶}}\AgdaSpace{}%
\AgdaBound{κ}\AgdaSpace{}%
\AgdaSymbol{(}\AgdaPostulate{η}\AgdaSpace{}%
\AgdaInductiveConstructor{null}\AgdaSpace{}%
\AgdaOperator{\AgdaPostulate{𝐌-in-𝐄}}\AgdaSymbol{)}\AgdaSpace{}%
\AgdaOperator{\AgdaPostulate{,}}\<%
\\
\>[4][@{}l@{\AgdaIndent{0}}]%
\>[6]\AgdaBound{list′}\AgdaSpace{}%
\AgdaSymbol{(}\AgdaBound{ϵ⋆}\AgdaSpace{}%
\AgdaOperator{\AgdaPostulate{†}}\AgdaSpace{}%
\AgdaNumber{1}\AgdaSymbol{)}\AgdaSpace{}%
\AgdaSymbol{(λ}\AgdaSpace{}%
\AgdaBound{ϵ}\AgdaSpace{}%
\AgdaSymbol{→}\AgdaSpace{}%
\AgdaFunction{cons}\AgdaSpace{}%
\AgdaOperator{\AgdaPostulate{⟨}}\AgdaSpace{}%
\AgdaSymbol{(}\AgdaBound{ϵ⋆}\AgdaSpace{}%
\AgdaOperator{\AgdaPostulate{↓}}\AgdaSpace{}%
\AgdaNumber{1}\AgdaSymbol{)}\AgdaSpace{}%
\AgdaOperator{\AgdaPostulate{,}}\AgdaSpace{}%
\AgdaBound{ϵ}\AgdaSpace{}%
\AgdaOperator{\AgdaPostulate{⟩}}\AgdaSpace{}%
\AgdaBound{κ}\AgdaSymbol{)}\<%
\\
%
\\[\AgdaEmptyExtraSkip]%
\>[0]\AgdaFunction{car}\AgdaSpace{}%
\AgdaSymbol{:}\AgdaSpace{}%
\AgdaFunction{𝐅}\<%
\\
\>[0]\AgdaFunction{car}\AgdaSpace{}%
\AgdaSymbol{=}\<%
\\
\>[0][@{}l@{\AgdaIndent{0}}]%
\>[2]\AgdaSymbol{λ}\AgdaSpace{}%
\AgdaBound{ϵ⋆}\AgdaSpace{}%
\AgdaBound{κ}\AgdaSpace{}%
\AgdaSymbol{→}\AgdaSpace{}%
\AgdaSymbol{(}\AgdaPostulate{\#}\AgdaSpace{}%
\AgdaBound{ϵ⋆}\AgdaSpace{}%
\AgdaOperator{\AgdaPostulate{==⊥}}\AgdaSpace{}%
\AgdaNumber{1}\AgdaSymbol{)}\AgdaSpace{}%
\AgdaOperator{\AgdaPostulate{⟶}}\AgdaSpace{}%
\AgdaFunction{hold}\AgdaSpace{}%
\AgdaSymbol{((}\AgdaBound{ϵ⋆}\AgdaSpace{}%
\AgdaOperator{\AgdaPostulate{↓}}\AgdaSpace{}%
\AgdaNumber{1}\AgdaSymbol{)}\AgdaSpace{}%
\AgdaOperator{\AgdaPostulate{|-𝐏}}\AgdaSpace{}%
\AgdaOperator{\AgdaPostulate{↓1}}\AgdaSymbol{)}\AgdaSpace{}%
\AgdaBound{κ}\AgdaSpace{}%
\AgdaOperator{\AgdaPostulate{,}}\AgdaSpace{}%
\AgdaPostulate{⊥}\<%
\\
%
\\[\AgdaEmptyExtraSkip]%
\>[0]\AgdaFunction{cdr}\AgdaSpace{}%
\AgdaSymbol{:}\AgdaSpace{}%
\AgdaFunction{𝐅}\<%
\\
\>[0]\AgdaFunction{cdr}\AgdaSpace{}%
\AgdaSymbol{=}\<%
\\
\>[0][@{}l@{\AgdaIndent{0}}]%
\>[2]\AgdaSymbol{λ}\AgdaSpace{}%
\AgdaBound{ϵ⋆}\AgdaSpace{}%
\AgdaBound{κ}\AgdaSpace{}%
\AgdaSymbol{→}\AgdaSpace{}%
\AgdaSymbol{(}\AgdaPostulate{\#}\AgdaSpace{}%
\AgdaBound{ϵ⋆}\AgdaSpace{}%
\AgdaOperator{\AgdaPostulate{==⊥}}\AgdaSpace{}%
\AgdaNumber{1}\AgdaSymbol{)}\AgdaSpace{}%
\AgdaOperator{\AgdaPostulate{⟶}}\AgdaSpace{}%
\AgdaFunction{hold}\AgdaSpace{}%
\AgdaSymbol{((}\AgdaBound{ϵ⋆}\AgdaSpace{}%
\AgdaOperator{\AgdaPostulate{↓}}\AgdaSpace{}%
\AgdaNumber{1}\AgdaSymbol{)}\AgdaSpace{}%
\AgdaOperator{\AgdaPostulate{|-𝐏}}\AgdaSpace{}%
\AgdaOperator{\AgdaPostulate{↓2}}\AgdaSymbol{)}\AgdaSpace{}%
\AgdaBound{κ}\AgdaSpace{}%
\AgdaOperator{\AgdaPostulate{,}}\AgdaSpace{}%
\AgdaPostulate{⊥}\<%
\\
%
\\[\AgdaEmptyExtraSkip]%
\>[0]\AgdaFunction{setcar}\AgdaSpace{}%
\AgdaSymbol{:}\AgdaSpace{}%
\AgdaFunction{𝐅}\<%
\\
\>[0]\AgdaFunction{setcar}\AgdaSpace{}%
\AgdaSymbol{=}\<%
\\
\>[0][@{}l@{\AgdaIndent{0}}]%
\>[2]\AgdaSymbol{λ}%
\>[280I]\AgdaBound{ϵ⋆}\AgdaSpace{}%
\AgdaBound{κ}\AgdaSpace{}%
\AgdaSymbol{→}\<%
\\
\>[280I][@{}l@{\AgdaIndent{0}}]%
\>[6]\AgdaSymbol{(}\AgdaPostulate{\#}\AgdaSpace{}%
\AgdaBound{ϵ⋆}\AgdaSpace{}%
\AgdaOperator{\AgdaPostulate{==⊥}}\AgdaSpace{}%
\AgdaNumber{2}\AgdaSymbol{)}\AgdaSpace{}%
\AgdaOperator{\AgdaPostulate{⟶}}\AgdaSpace{}%
\AgdaFunction{assign}%
\>[29]\AgdaSymbol{((}\AgdaBound{ϵ⋆}\AgdaSpace{}%
\AgdaOperator{\AgdaPostulate{↓}}\AgdaSpace{}%
\AgdaNumber{1}\AgdaSymbol{)}\AgdaSpace{}%
\AgdaOperator{\AgdaPostulate{|-𝐏}}\AgdaSpace{}%
\AgdaOperator{\AgdaPostulate{↓1}}\AgdaSymbol{)}\<%
\\
%
\>[29]\AgdaSymbol{(}\AgdaBound{ϵ⋆}\AgdaSpace{}%
\AgdaOperator{\AgdaPostulate{↓}}\AgdaSpace{}%
\AgdaNumber{2}\AgdaSymbol{)}\<%
\\
%
\>[29]\AgdaSymbol{(}\AgdaBound{κ}\AgdaSpace{}%
\AgdaSymbol{(}\AgdaPostulate{η}\AgdaSpace{}%
\AgdaInductiveConstructor{unspecified}\AgdaSpace{}%
\AgdaOperator{\AgdaPostulate{𝐌-in-𝐄}}\AgdaSymbol{))}\AgdaSpace{}%
\AgdaOperator{\AgdaPostulate{,}}\<%
\\
\>[.][@{}l@{}]\<[280I]%
\>[4]\AgdaPostulate{⊥}\<%
\\
%
\\[\AgdaEmptyExtraSkip]%
\>[0]\AgdaFunction{setcdr}\AgdaSpace{}%
\AgdaSymbol{:}\AgdaSpace{}%
\AgdaFunction{𝐅}\<%
\\
\>[0]\AgdaFunction{setcdr}\AgdaSpace{}%
\AgdaSymbol{=}\<%
\\
\>[0][@{}l@{\AgdaIndent{0}}]%
\>[2]\AgdaSymbol{λ}%
\>[301I]\AgdaBound{ϵ⋆}\AgdaSpace{}%
\AgdaBound{κ}\AgdaSpace{}%
\AgdaSymbol{→}\<%
\\
\>[301I][@{}l@{\AgdaIndent{0}}]%
\>[6]\AgdaSymbol{(}\AgdaPostulate{\#}\AgdaSpace{}%
\AgdaBound{ϵ⋆}\AgdaSpace{}%
\AgdaOperator{\AgdaPostulate{==⊥}}\AgdaSpace{}%
\AgdaNumber{2}\AgdaSymbol{)}\AgdaSpace{}%
\AgdaOperator{\AgdaPostulate{⟶}}\AgdaSpace{}%
\AgdaFunction{assign}%
\>[29]\AgdaSymbol{((}\AgdaBound{ϵ⋆}\AgdaSpace{}%
\AgdaOperator{\AgdaPostulate{↓}}\AgdaSpace{}%
\AgdaNumber{1}\AgdaSymbol{)}\AgdaSpace{}%
\AgdaOperator{\AgdaPostulate{|-𝐏}}\AgdaSpace{}%
\AgdaOperator{\AgdaPostulate{↓2}}\AgdaSymbol{)}\<%
\\
%
\>[29]\AgdaSymbol{(}\AgdaBound{ϵ⋆}\AgdaSpace{}%
\AgdaOperator{\AgdaPostulate{↓}}\AgdaSpace{}%
\AgdaNumber{2}\AgdaSymbol{)}\<%
\\
%
\>[29]\AgdaSymbol{(}\AgdaBound{κ}\AgdaSpace{}%
\AgdaSymbol{(}\AgdaPostulate{η}\AgdaSpace{}%
\AgdaInductiveConstructor{unspecified}\AgdaSpace{}%
\AgdaOperator{\AgdaPostulate{𝐌-in-𝐄}}\AgdaSymbol{))}\AgdaSpace{}%
\AgdaOperator{\AgdaPostulate{,}}\<%
\\
\>[.][@{}l@{}]\<[301I]%
\>[4]\AgdaPostulate{⊥}\<%
\end{code}
\clearpage
\begin{code}%
\>[0]\AgdaComment{--\ datum\ prefix\ pre⟦\ Δ\ ⟧\ :\ Dat}\<%
\\
%
\\[\AgdaEmptyExtraSkip]%
\>[0]\AgdaOperator{\AgdaFunction{pre⟦\AgdaUnderscore{}⟧}}\AgdaSpace{}%
\AgdaSymbol{:}\AgdaSpace{}%
\AgdaDatatype{Dat}\AgdaSpace{}%
\AgdaSymbol{→}\AgdaSpace{}%
\AgdaDatatype{Dat}\<%
\\
%
\\[\AgdaEmptyExtraSkip]%
\>[0]\AgdaOperator{\AgdaFunction{pre⟦}}\AgdaSpace{}%
\AgdaOperator{\AgdaInductiveConstructor{⦅}}\AgdaSpace{}%
\AgdaOperator{\AgdaInductiveConstructor{␣␣}}\AgdaSpace{}%
\AgdaBound{Δ}\AgdaSpace{}%
\AgdaOperator{\AgdaInductiveConstructor{·}}\AgdaSpace{}%
\AgdaOperator{\AgdaInductiveConstructor{⦅}}\AgdaSpace{}%
\AgdaBound{Δ⋆}\AgdaSpace{}%
\AgdaOperator{\AgdaInductiveConstructor{⦆}}\AgdaSpace{}%
\AgdaOperator{\AgdaInductiveConstructor{⦆}}\AgdaSpace{}%
\AgdaOperator{\AgdaFunction{⟧}}\AgdaSpace{}%
\AgdaSymbol{=}\AgdaSpace{}%
\AgdaOperator{\AgdaFunction{⟦}}\AgdaSpace{}%
\AgdaOperator{\AgdaInductiveConstructor{⦅}}\AgdaSpace{}%
\AgdaBound{Δ}\AgdaSpace{}%
\AgdaOperator{\AgdaInductiveConstructor{␣␣}}\AgdaSpace{}%
\AgdaBound{Δ⋆}\AgdaSpace{}%
\AgdaOperator{\AgdaInductiveConstructor{⦆}}\AgdaSpace{}%
\AgdaOperator{\AgdaFunction{⟧}}\<%
\\
\>[0]\AgdaComment{--\ otherwise:}\<%
\\
\>[0]\AgdaCatchallClause{\AgdaOperator{\AgdaFunction{pre⟦}}}\AgdaSpace{}%
\AgdaCatchallClause{\AgdaBound{Δ}}\AgdaSpace{}%
\AgdaCatchallClause{\AgdaOperator{\AgdaFunction{⟧}}}\AgdaSpace{}%
\AgdaSymbol{=}\AgdaSpace{}%
\AgdaOperator{\AgdaFunction{⟦}}\AgdaSpace{}%
\AgdaBound{Δ}\AgdaSpace{}%
\AgdaOperator{\AgdaFunction{⟧}}\<%
\\
%
\\[\AgdaEmptyExtraSkip]%
\>[0]\AgdaComment{--\ datum\ ϵ\ κ\ applies\ κ\ to\ the\ datum\ represented\ by\ the\ value\ ϵ}\<%
\\
\>[0]\AgdaFunction{datum}\AgdaSpace{}%
\AgdaSymbol{:}\AgdaSpace{}%
\AgdaPostulate{𝐄}\AgdaSpace{}%
\AgdaSymbol{→}\AgdaSpace{}%
\AgdaSymbol{(}\AgdaDatatype{Dat}\AgdaSpace{}%
\AgdaSymbol{→}\AgdaSpace{}%
\AgdaFunction{𝐂}\AgdaSymbol{)}\AgdaSpace{}%
\AgdaSymbol{→}\AgdaSpace{}%
\AgdaFunction{𝐂}\<%
\\
\>[0]\AgdaFunction{datum}\AgdaSpace{}%
\AgdaSymbol{=}\AgdaSpace{}%
\AgdaPostulate{fix}\AgdaSpace{}%
\AgdaSymbol{λ}\AgdaSpace{}%
\AgdaBound{datum′}\AgdaSpace{}%
\AgdaSymbol{→}\<%
\\
\>[0][@{}l@{\AgdaIndent{0}}]%
\>[2]\AgdaSymbol{λ}%
\>[359I]\AgdaBound{ϵ}\AgdaSpace{}%
\AgdaBound{κ}\AgdaSpace{}%
\AgdaSymbol{→}\<%
\\
\>[.][@{}l@{}]\<[359I]%
\>[4]\AgdaSymbol{(}\AgdaBound{ϵ}\AgdaSpace{}%
\AgdaOperator{\AgdaPostulate{∈-𝐓}}\AgdaSymbol{)}\AgdaSpace{}%
\AgdaOperator{\AgdaPostulate{⟶}}\<%
\\
\>[4][@{}l@{\AgdaIndent{0}}]%
\>[6]\AgdaSymbol{((}\AgdaBound{ϵ}\AgdaSpace{}%
\AgdaOperator{\AgdaPostulate{|-𝐓}}\AgdaSymbol{)}\AgdaSpace{}%
\AgdaOperator{\AgdaPostulate{⟶}}\AgdaSpace{}%
\AgdaBound{κ}\AgdaSpace{}%
\AgdaOperator{\AgdaFunction{⟦}}\AgdaSpace{}%
\AgdaInductiveConstructor{con}\AgdaSpace{}%
\AgdaInductiveConstructor{\#t}\AgdaSpace{}%
\AgdaOperator{\AgdaFunction{⟧}}\AgdaSpace{}%
\AgdaOperator{\AgdaPostulate{,}}\AgdaSpace{}%
\AgdaBound{κ}\AgdaSpace{}%
\AgdaOperator{\AgdaFunction{⟦}}\AgdaSpace{}%
\AgdaInductiveConstructor{con}\AgdaSpace{}%
\AgdaInductiveConstructor{\#f}\AgdaSpace{}%
\AgdaOperator{\AgdaFunction{⟧}}\AgdaSymbol{)}\AgdaSpace{}%
\AgdaOperator{\AgdaPostulate{,}}\<%
\\
%
\>[4]\AgdaSymbol{(}\AgdaBound{ϵ}\AgdaSpace{}%
\AgdaOperator{\AgdaPostulate{∈-𝐑}}\AgdaSymbol{)}\AgdaSpace{}%
\AgdaOperator{\AgdaPostulate{⟶}}\<%
\\
\>[4][@{}l@{\AgdaIndent{0}}]%
\>[6]\AgdaSymbol{((λ}\AgdaSpace{}%
\AgdaBound{Z}\AgdaSpace{}%
\AgdaSymbol{→}\AgdaSpace{}%
\AgdaBound{κ}\AgdaSpace{}%
\AgdaOperator{\AgdaFunction{⟦}}\AgdaSpace{}%
\AgdaInductiveConstructor{con}\AgdaSpace{}%
\AgdaSymbol{(}\AgdaInductiveConstructor{int}\AgdaSpace{}%
\AgdaBound{Z}\AgdaSymbol{)}\AgdaSpace{}%
\AgdaOperator{\AgdaFunction{⟧}}\AgdaSymbol{)}\AgdaSpace{}%
\AgdaOperator{\AgdaPostulate{♯}}\AgdaSymbol{)}\AgdaSpace{}%
\AgdaSymbol{(}\AgdaBound{ϵ}\AgdaSpace{}%
\AgdaOperator{\AgdaPostulate{|-𝐑}}\AgdaSymbol{)}\AgdaSpace{}%
\AgdaOperator{\AgdaPostulate{,}}\<%
\\
%
\>[4]\AgdaSymbol{(}\AgdaBound{ϵ}\AgdaSpace{}%
\AgdaOperator{\AgdaPostulate{∈-𝐏}}\AgdaSymbol{)}\AgdaSpace{}%
\AgdaOperator{\AgdaPostulate{⟶}}\<%
\\
\>[4][@{}l@{\AgdaIndent{0}}]%
\>[6]\AgdaFunction{car}\AgdaSpace{}%
\AgdaOperator{\AgdaPostulate{⟨}}\AgdaSpace{}%
\AgdaBound{ϵ}\AgdaSpace{}%
\AgdaOperator{\AgdaPostulate{⟩}}\AgdaSpace{}%
\AgdaSymbol{(λ}\AgdaSpace{}%
\AgdaBound{ϵ₁}\AgdaSpace{}%
\AgdaSymbol{→}\AgdaSpace{}%
\AgdaFunction{cdr}\AgdaSpace{}%
\AgdaOperator{\AgdaPostulate{⟨}}\AgdaSpace{}%
\AgdaBound{ϵ}\AgdaSpace{}%
\AgdaOperator{\AgdaPostulate{⟩}}\AgdaSpace{}%
\AgdaSymbol{(λ}\AgdaSpace{}%
\AgdaBound{ϵ₂}\AgdaSpace{}%
\AgdaSymbol{→}\<%
\\
\>[6][@{}l@{\AgdaIndent{0}}]%
\>[8]\AgdaBound{datum′}\AgdaSpace{}%
\AgdaBound{ϵ₁}\AgdaSpace{}%
\AgdaSymbol{(λ}\AgdaSpace{}%
\AgdaBound{Δ₁}\AgdaSpace{}%
\AgdaSymbol{→}\AgdaSpace{}%
\AgdaBound{datum′}\AgdaSpace{}%
\AgdaBound{ϵ₂}\AgdaSpace{}%
\AgdaSymbol{(λ}\AgdaSpace{}%
\AgdaBound{Δ₂}\AgdaSpace{}%
\AgdaSymbol{→}\<%
\\
\>[8][@{}l@{\AgdaIndent{0}}]%
\>[10]\AgdaBound{κ}\AgdaSpace{}%
\AgdaOperator{\AgdaFunction{pre⟦}}\AgdaSpace{}%
\AgdaOperator{\AgdaInductiveConstructor{⦅}}\AgdaSpace{}%
\AgdaOperator{\AgdaInductiveConstructor{␣␣}}\AgdaSpace{}%
\AgdaBound{Δ₁}\AgdaSpace{}%
\AgdaOperator{\AgdaInductiveConstructor{·}}\AgdaSpace{}%
\AgdaBound{Δ₂}\AgdaSpace{}%
\AgdaOperator{\AgdaInductiveConstructor{⦆}}\AgdaSpace{}%
\AgdaOperator{\AgdaFunction{⟧}}\AgdaSymbol{))))}\AgdaSpace{}%
\AgdaOperator{\AgdaPostulate{,}}\<%
\\
%
\>[4]\AgdaSymbol{(}\AgdaBound{ϵ}\AgdaSpace{}%
\AgdaOperator{\AgdaPostulate{∈-𝐌}}\AgdaSymbol{)}\AgdaSpace{}%
\AgdaOperator{\AgdaPostulate{⟶}}\<%
\\
\>[4][@{}l@{\AgdaIndent{0}}]%
\>[6]\AgdaSymbol{(((}\AgdaBound{ϵ}\AgdaSpace{}%
\AgdaOperator{\AgdaPostulate{|-𝐌}}\AgdaSymbol{)}\AgdaSpace{}%
\AgdaOperator{\AgdaPostulate{==ᴹ}}\AgdaSpace{}%
\AgdaPostulate{η}\AgdaSpace{}%
\AgdaInductiveConstructor{null}\AgdaSymbol{)}\AgdaSpace{}%
\AgdaOperator{\AgdaPostulate{⟶}}\AgdaSpace{}%
\AgdaBound{κ}\AgdaSpace{}%
\AgdaOperator{\AgdaFunction{⟦}}\AgdaSpace{}%
\AgdaOperator{\AgdaInductiveConstructor{⦅}}\AgdaSpace{}%
\AgdaInductiveConstructor{␣␣␣}\AgdaSpace{}%
\AgdaOperator{\AgdaInductiveConstructor{⦆}}\AgdaSpace{}%
\AgdaOperator{\AgdaFunction{⟧}}%
\>[45]\AgdaOperator{\AgdaPostulate{,}}\AgdaSpace{}%
\AgdaPostulate{⊥}\AgdaSymbol{)}\AgdaSpace{}%
\AgdaOperator{\AgdaPostulate{,}}\<%
\\
%
\>[4]\AgdaSymbol{(}\AgdaBound{ϵ}\AgdaSpace{}%
\AgdaOperator{\AgdaPostulate{∈-𝐅}}\AgdaSymbol{)}\AgdaSpace{}%
\AgdaOperator{\AgdaPostulate{⟶}}\<%
\\
\>[4][@{}l@{\AgdaIndent{0}}]%
\>[6]\AgdaBound{κ}\AgdaSpace{}%
\AgdaOperator{\AgdaFunction{⟦}}\AgdaSpace{}%
\AgdaInductiveConstructor{\#proc}\AgdaSpace{}%
\AgdaOperator{\AgdaFunction{⟧}}\AgdaSpace{}%
\AgdaOperator{\AgdaPostulate{,}}\<%
\\
%
\>[4]\AgdaSymbol{(}\AgdaBound{ϵ}\AgdaSpace{}%
\AgdaOperator{\AgdaPostulate{∈-𝐐}}\AgdaSymbol{)}\AgdaSpace{}%
\AgdaOperator{\AgdaPostulate{⟶}}\<%
\\
\>[4][@{}l@{\AgdaIndent{0}}]%
\>[6]\AgdaSymbol{((λ}\AgdaSpace{}%
\AgdaBound{I}\AgdaSpace{}%
\AgdaSymbol{→}\AgdaSpace{}%
\AgdaBound{κ}\AgdaSpace{}%
\AgdaOperator{\AgdaFunction{⟦}}\AgdaSpace{}%
\AgdaInductiveConstructor{ide}\AgdaSpace{}%
\AgdaBound{I}\AgdaSpace{}%
\AgdaOperator{\AgdaFunction{⟧}}\AgdaSymbol{)}\AgdaSpace{}%
\AgdaOperator{\AgdaPostulate{♯}}\AgdaSymbol{)}\AgdaSpace{}%
\AgdaSymbol{(}\AgdaBound{ϵ}\AgdaSpace{}%
\AgdaOperator{\AgdaPostulate{|-𝐐}}\AgdaSymbol{)}\AgdaSpace{}%
\AgdaOperator{\AgdaPostulate{,}}\<%
\\
%
\>[4]\AgdaSymbol{(}\AgdaBound{ϵ}\AgdaSpace{}%
\AgdaOperator{\AgdaPostulate{∈-𝐗}}\AgdaSymbol{)}\AgdaSpace{}%
\AgdaOperator{\AgdaPostulate{⟶}}\<%
\\
\>[4][@{}l@{\AgdaIndent{0}}]%
\>[6]\AgdaSymbol{((λ}\AgdaSpace{}%
\AgdaBound{X}\AgdaSpace{}%
\AgdaSymbol{→}\AgdaSpace{}%
\AgdaBound{κ}\AgdaSpace{}%
\AgdaOperator{\AgdaFunction{⟦}}\AgdaSpace{}%
\AgdaInductiveConstructor{key}\AgdaSpace{}%
\AgdaBound{X}\AgdaSpace{}%
\AgdaOperator{\AgdaFunction{⟧}}\AgdaSymbol{)}\AgdaSpace{}%
\AgdaOperator{\AgdaPostulate{♯}}\AgdaSymbol{)}\AgdaSpace{}%
\AgdaSymbol{(}\AgdaBound{ϵ}\AgdaSpace{}%
\AgdaOperator{\AgdaPostulate{|-𝐗}}\AgdaSymbol{)}\AgdaSpace{}%
\AgdaOperator{\AgdaPostulate{,}}\<%
\\
%
\>[4]\AgdaPostulate{⊥}\<%
\end{code}
\clearpage
\begin{code}%
\>[0]\AgdaComment{--\ mapping\ datum\ terms\ to\ expressions}\<%
\\
%
\\[\AgdaEmptyExtraSkip]%
\>[0]\AgdaOperator{\AgdaFunction{exp⟦\AgdaUnderscore{}⟧}}%
\>[9]\AgdaSymbol{:}\AgdaSpace{}%
\AgdaDatatype{Dat}\AgdaSpace{}%
\AgdaSymbol{→}\AgdaSpace{}%
\AgdaDatatype{Exp}\<%
\\
\>[0]\AgdaOperator{\AgdaFunction{exp⋆⟦\AgdaUnderscore{}⟧}}%
\>[9]\AgdaSymbol{:}\AgdaSpace{}%
\AgdaDatatype{Dat⋆}\AgdaSpace{}%
\AgdaSymbol{→}\AgdaSpace{}%
\AgdaDatatype{Exp⋆}\<%
\\
%
\\[\AgdaEmptyExtraSkip]%
\>[0]\AgdaComment{--\ datum\ expressions\ exp⟦\ Δ\ ⟧\ :\ Exp}\<%
\\
%
\\[\AgdaEmptyExtraSkip]%
\>[0]\AgdaOperator{\AgdaFunction{exp⟦}}\AgdaSpace{}%
\AgdaInductiveConstructor{con}\AgdaSpace{}%
\AgdaBound{K}\AgdaSpace{}%
\AgdaOperator{\AgdaFunction{⟧}}%
\>[14]\AgdaSymbol{=}\AgdaSpace{}%
\AgdaOperator{\AgdaFunction{⟦}}\AgdaSpace{}%
\AgdaInductiveConstructor{con}\AgdaSpace{}%
\AgdaBound{K}\AgdaSpace{}%
\AgdaOperator{\AgdaFunction{⟧}}\<%
\\
%
\\[\AgdaEmptyExtraSkip]%
\>[0]\AgdaOperator{\AgdaFunction{exp⟦}}\AgdaSpace{}%
\AgdaInductiveConstructor{ide}\AgdaSpace{}%
\AgdaBound{I}\AgdaSpace{}%
\AgdaOperator{\AgdaFunction{⟧}}%
\>[14]\AgdaSymbol{=}\AgdaSpace{}%
\AgdaOperator{\AgdaFunction{⟦}}\AgdaSpace{}%
\AgdaInductiveConstructor{ide}\AgdaSpace{}%
\AgdaBound{I}\AgdaSpace{}%
\AgdaOperator{\AgdaFunction{⟧}}\<%
\\
%
\\[\AgdaEmptyExtraSkip]%
\>[0]\AgdaOperator{\AgdaFunction{exp⟦}}\AgdaSpace{}%
\AgdaInductiveConstructor{′}\AgdaSpace{}%
\AgdaBound{Δ}\AgdaSpace{}%
\AgdaOperator{\AgdaFunction{⟧}}%
\>[14]\AgdaSymbol{=}\AgdaSpace{}%
\AgdaOperator{\AgdaFunction{⟦}}\AgdaSpace{}%
\AgdaOperator{\AgdaInductiveConstructor{⦅quote}}\AgdaSpace{}%
\AgdaBound{Δ}\AgdaSpace{}%
\AgdaOperator{\AgdaInductiveConstructor{⦆}}\AgdaSpace{}%
\AgdaOperator{\AgdaFunction{⟧}}\<%
\\
%
\\[\AgdaEmptyExtraSkip]%
\>[0]\AgdaOperator{\AgdaFunction{exp⟦}}\AgdaSpace{}%
\AgdaOperator{\AgdaInductiveConstructor{⦅}}\AgdaSpace{}%
\AgdaInductiveConstructor{key}\AgdaSpace{}%
\AgdaInductiveConstructor{quote′}\AgdaSpace{}%
\AgdaOperator{\AgdaInductiveConstructor{␣␣}}\AgdaSpace{}%
\AgdaBound{Δ}\AgdaSpace{}%
\AgdaOperator{\AgdaInductiveConstructor{␣␣}}\AgdaSpace{}%
\AgdaInductiveConstructor{␣␣␣}\AgdaSpace{}%
\AgdaOperator{\AgdaInductiveConstructor{⦆}}\AgdaSpace{}%
\AgdaOperator{\AgdaFunction{⟧}}\AgdaSpace{}%
\AgdaSymbol{=}\AgdaSpace{}%
\AgdaOperator{\AgdaFunction{⟦}}\AgdaSpace{}%
\AgdaOperator{\AgdaInductiveConstructor{⦅quote}}\AgdaSpace{}%
\AgdaBound{Δ}\AgdaSpace{}%
\AgdaOperator{\AgdaInductiveConstructor{⦆}}\AgdaSpace{}%
\AgdaOperator{\AgdaFunction{⟧}}\<%
\\
%
\\[\AgdaEmptyExtraSkip]%
\>[0]\AgdaOperator{\AgdaFunction{exp⟦}}\AgdaSpace{}%
\AgdaOperator{\AgdaInductiveConstructor{⦅}}\AgdaSpace{}%
\AgdaInductiveConstructor{key}\AgdaSpace{}%
\AgdaInductiveConstructor{lambda}\AgdaSpace{}%
\AgdaOperator{\AgdaInductiveConstructor{␣␣}}\AgdaSpace{}%
\AgdaInductiveConstructor{ide}\AgdaSpace{}%
\AgdaBound{I}\AgdaSpace{}%
\AgdaOperator{\AgdaInductiveConstructor{␣␣}}\AgdaSpace{}%
\AgdaBound{Δ}\AgdaSpace{}%
\AgdaOperator{\AgdaInductiveConstructor{␣␣}}\AgdaSpace{}%
\AgdaInductiveConstructor{␣␣␣}\AgdaSpace{}%
\AgdaOperator{\AgdaInductiveConstructor{⦆}}\AgdaSpace{}%
\AgdaOperator{\AgdaFunction{⟧}}\AgdaSpace{}%
\AgdaSymbol{=}\<%
\\
\>[0][@{}l@{\AgdaIndent{0}}]%
\>[2]\AgdaOperator{\AgdaFunction{⟦}}\AgdaSpace{}%
\AgdaOperator{\AgdaInductiveConstructor{⦅lambda}}\AgdaSpace{}%
\AgdaBound{I}\AgdaSpace{}%
\AgdaOperator{\AgdaInductiveConstructor{␣}}\AgdaSpace{}%
\AgdaOperator{\AgdaFunction{exp⟦}}\AgdaSpace{}%
\AgdaBound{Δ}\AgdaSpace{}%
\AgdaOperator{\AgdaFunction{⟧}}\AgdaSpace{}%
\AgdaOperator{\AgdaInductiveConstructor{⦆}}\AgdaSpace{}%
\AgdaOperator{\AgdaFunction{⟧}}\<%
\\
%
\\[\AgdaEmptyExtraSkip]%
\>[0]\AgdaOperator{\AgdaFunction{exp⟦}}%
\>[536I]\AgdaOperator{\AgdaInductiveConstructor{⦅}}\AgdaSpace{}%
\AgdaInductiveConstructor{key}\AgdaSpace{}%
\AgdaInductiveConstructor{if}\AgdaSpace{}%
\AgdaOperator{\AgdaInductiveConstructor{␣␣}}\AgdaSpace{}%
\AgdaBound{Δ}\AgdaSpace{}%
\AgdaOperator{\AgdaInductiveConstructor{␣␣}}\AgdaSpace{}%
\AgdaBound{Δ₁}\AgdaSpace{}%
\AgdaOperator{\AgdaInductiveConstructor{␣␣}}\AgdaSpace{}%
\AgdaBound{Δ₂}\AgdaSpace{}%
\AgdaOperator{\AgdaInductiveConstructor{␣␣}}\AgdaSpace{}%
\AgdaInductiveConstructor{␣␣␣}\AgdaSpace{}%
\AgdaOperator{\AgdaInductiveConstructor{⦆}}\AgdaSpace{}%
\AgdaOperator{\AgdaFunction{⟧}}\AgdaSpace{}%
\AgdaSymbol{=}\<%
\\
\>[536I][@{}l@{\AgdaIndent{0}}]%
\>[6]\AgdaOperator{\AgdaFunction{⟦}}\AgdaSpace{}%
\AgdaOperator{\AgdaInductiveConstructor{⦅if}}\AgdaSpace{}%
\AgdaOperator{\AgdaFunction{exp⟦}}\AgdaSpace{}%
\AgdaBound{Δ}\AgdaSpace{}%
\AgdaOperator{\AgdaFunction{⟧}}\AgdaSpace{}%
\AgdaOperator{\AgdaInductiveConstructor{␣}}\AgdaSpace{}%
\AgdaOperator{\AgdaFunction{exp⟦}}\AgdaSpace{}%
\AgdaBound{Δ₁}\AgdaSpace{}%
\AgdaOperator{\AgdaFunction{⟧}}\AgdaSpace{}%
\AgdaOperator{\AgdaInductiveConstructor{␣}}\AgdaSpace{}%
\AgdaOperator{\AgdaFunction{exp⟦}}\AgdaSpace{}%
\AgdaBound{Δ₂}\AgdaSpace{}%
\AgdaOperator{\AgdaFunction{⟧}}\AgdaSpace{}%
\AgdaOperator{\AgdaInductiveConstructor{⦆}}\AgdaSpace{}%
\AgdaOperator{\AgdaFunction{⟧}}\<%
\\
%
\\[\AgdaEmptyExtraSkip]%
\>[0]\AgdaOperator{\AgdaFunction{exp⟦}}\AgdaSpace{}%
\AgdaOperator{\AgdaInductiveConstructor{⦅}}\AgdaSpace{}%
\AgdaInductiveConstructor{key}\AgdaSpace{}%
\AgdaInductiveConstructor{set!}\AgdaSpace{}%
\AgdaOperator{\AgdaInductiveConstructor{␣␣}}\AgdaSpace{}%
\AgdaInductiveConstructor{ide}\AgdaSpace{}%
\AgdaBound{I}\AgdaSpace{}%
\AgdaOperator{\AgdaInductiveConstructor{␣␣}}\AgdaSpace{}%
\AgdaBound{Δ}\AgdaSpace{}%
\AgdaOperator{\AgdaInductiveConstructor{␣␣}}\AgdaSpace{}%
\AgdaInductiveConstructor{␣␣␣}\AgdaSpace{}%
\AgdaOperator{\AgdaInductiveConstructor{⦆}}\AgdaSpace{}%
\AgdaOperator{\AgdaFunction{⟧}}\AgdaSpace{}%
\AgdaSymbol{=}\<%
\\
\>[0][@{}l@{\AgdaIndent{0}}]%
\>[2]\AgdaOperator{\AgdaFunction{⟦}}\AgdaSpace{}%
\AgdaOperator{\AgdaInductiveConstructor{⦅set!}}\AgdaSpace{}%
\AgdaBound{I}\AgdaSpace{}%
\AgdaOperator{\AgdaInductiveConstructor{␣}}\AgdaSpace{}%
\AgdaOperator{\AgdaFunction{exp⟦}}\AgdaSpace{}%
\AgdaBound{Δ}\AgdaSpace{}%
\AgdaOperator{\AgdaFunction{⟧}}\AgdaSpace{}%
\AgdaOperator{\AgdaInductiveConstructor{⦆}}\AgdaSpace{}%
\AgdaOperator{\AgdaFunction{⟧}}\<%
\\
%
\\[\AgdaEmptyExtraSkip]%
\>[0]\AgdaOperator{\AgdaFunction{exp⟦}}\AgdaSpace{}%
\AgdaOperator{\AgdaInductiveConstructor{⦅}}\AgdaSpace{}%
\AgdaInductiveConstructor{ide}\AgdaSpace{}%
\AgdaBound{I}\AgdaSpace{}%
\AgdaOperator{\AgdaInductiveConstructor{␣␣}}\AgdaSpace{}%
\AgdaBound{Δ⋆}\AgdaSpace{}%
\AgdaOperator{\AgdaInductiveConstructor{⦆}}\AgdaSpace{}%
\AgdaOperator{\AgdaFunction{⟧}}\AgdaSpace{}%
\AgdaSymbol{=}\<%
\\
\>[0][@{}l@{\AgdaIndent{0}}]%
\>[2]\AgdaOperator{\AgdaFunction{⟦}}\AgdaSpace{}%
\AgdaOperator{\AgdaInductiveConstructor{⦅}}\AgdaSpace{}%
\AgdaInductiveConstructor{ide}\AgdaSpace{}%
\AgdaBound{I}\AgdaSpace{}%
\AgdaOperator{\AgdaInductiveConstructor{␣}}\AgdaSpace{}%
\AgdaOperator{\AgdaFunction{exp⋆⟦}}\AgdaSpace{}%
\AgdaBound{Δ⋆}\AgdaSpace{}%
\AgdaOperator{\AgdaFunction{⟧}}\AgdaSpace{}%
\AgdaOperator{\AgdaInductiveConstructor{⦆}}\AgdaSpace{}%
\AgdaOperator{\AgdaFunction{⟧}}\<%
\\
%
\\[\AgdaEmptyExtraSkip]%
\>[0]\AgdaCatchallClause{\AgdaOperator{\AgdaFunction{exp⟦}}}\AgdaSpace{}%
\AgdaCatchallClause{\AgdaSymbol{\AgdaUnderscore{}}}\AgdaSpace{}%
\AgdaCatchallClause{\AgdaOperator{\AgdaFunction{⟧}}}\AgdaSpace{}%
\AgdaSymbol{=}\AgdaSpace{}%
\AgdaOperator{\AgdaFunction{⟦}}\AgdaSpace{}%
\AgdaInductiveConstructor{⦅␣⦆}\AgdaSpace{}%
\AgdaOperator{\AgdaFunction{⟧}}\<%
\\
%
\\[\AgdaEmptyExtraSkip]%
\>[0]\AgdaComment{--\ datum\ sequence\ expressions\ exp⋆⟦\ Δ⋆\ :\ Exp⋆}\<%
\\
%
\\[\AgdaEmptyExtraSkip]%
\>[0]\AgdaOperator{\AgdaFunction{exp⋆⟦}}\AgdaSpace{}%
\AgdaInductiveConstructor{␣␣␣}\AgdaSpace{}%
\AgdaOperator{\AgdaFunction{⟧}}\AgdaSpace{}%
\AgdaSymbol{=}\AgdaSpace{}%
\AgdaOperator{\AgdaFunction{⟦}}\AgdaSpace{}%
\AgdaInductiveConstructor{␣␣␣}\AgdaSpace{}%
\AgdaOperator{\AgdaFunction{⟧}}\<%
\\
%
\\[\AgdaEmptyExtraSkip]%
\>[0]\AgdaOperator{\AgdaFunction{exp⋆⟦}}\AgdaSpace{}%
\AgdaBound{Δ}\AgdaSpace{}%
\AgdaOperator{\AgdaInductiveConstructor{␣␣}}\AgdaSpace{}%
\AgdaBound{Δ⋆}\AgdaSpace{}%
\AgdaOperator{\AgdaFunction{⟧}}\AgdaSpace{}%
\AgdaSymbol{=}\AgdaSpace{}%
\AgdaOperator{\AgdaFunction{⟦}}\AgdaSpace{}%
\AgdaOperator{\AgdaFunction{exp⟦}}\AgdaSpace{}%
\AgdaBound{Δ}\AgdaSpace{}%
\AgdaOperator{\AgdaFunction{⟧}}\AgdaSpace{}%
\AgdaOperator{\AgdaInductiveConstructor{␣␣}}\AgdaSpace{}%
\AgdaOperator{\AgdaFunction{exp⋆⟦}}\AgdaSpace{}%
\AgdaBound{Δ⋆}\AgdaSpace{}%
\AgdaOperator{\AgdaFunction{⟧}}\AgdaSpace{}%
\AgdaOperator{\AgdaFunction{⟧}}\<%
\end{code}

\clearpage

\appendix 

\section{Notation}

\begin{code}%
\>[0]\AgdaKeyword{module}\AgdaSpace{}%
\AgdaModule{Notation}\AgdaSpace{}%
\AgdaKeyword{where}\<%
\\
%
\\[\AgdaEmptyExtraSkip]%
\>[0]\AgdaKeyword{open}\AgdaSpace{}%
\AgdaKeyword{import}\AgdaSpace{}%
\AgdaModule{Data.Bool.Base}%
\>[30]\AgdaKeyword{using}\AgdaSpace{}%
\AgdaSymbol{(}\AgdaDatatype{Bool}\AgdaSymbol{;}\AgdaSpace{}%
\AgdaInductiveConstructor{false}\AgdaSymbol{;}\AgdaSpace{}%
\AgdaInductiveConstructor{true}\AgdaSymbol{)}\AgdaSpace{}%
\AgdaKeyword{public}\<%
\\
\>[0]\AgdaKeyword{open}\AgdaSpace{}%
\AgdaKeyword{import}\AgdaSpace{}%
\AgdaModule{Data.Nat.Base}%
\>[30]\AgdaKeyword{renaming}\AgdaSpace{}%
\AgdaSymbol{(}\AgdaDatatype{ℕ}\AgdaSpace{}%
\AgdaSymbol{to}\AgdaSpace{}%
\AgdaDatatype{Nat}\AgdaSymbol{)}\AgdaSpace{}%
\AgdaKeyword{using}\AgdaSpace{}%
\AgdaSymbol{(}\AgdaInductiveConstructor{suc}\AgdaSymbol{)}\AgdaSpace{}%
\AgdaKeyword{public}\<%
\\
\>[0]\AgdaKeyword{open}\AgdaSpace{}%
\AgdaKeyword{import}\AgdaSpace{}%
\AgdaModule{Data.String.Base}%
\>[30]\AgdaKeyword{using}\AgdaSpace{}%
\AgdaSymbol{(}\AgdaPostulate{String}\AgdaSymbol{)}\AgdaSpace{}%
\AgdaKeyword{public}\<%
\\
\>[0]\AgdaKeyword{open}\AgdaSpace{}%
\AgdaKeyword{import}\AgdaSpace{}%
\AgdaModule{Data.Unit.Base}%
\>[30]\AgdaKeyword{using}\AgdaSpace{}%
\AgdaSymbol{(}\AgdaRecord{⊤}\AgdaSymbol{)}\<%
\\
\>[0]\AgdaKeyword{open}\AgdaSpace{}%
\AgdaKeyword{import}\AgdaSpace{}%
\AgdaModule{Function}%
\>[30]\AgdaKeyword{using}\AgdaSpace{}%
\AgdaSymbol{(}\AgdaFunction{id}\AgdaSymbol{;}\AgdaSpace{}%
\AgdaOperator{\AgdaFunction{\AgdaUnderscore{}∘\AgdaUnderscore{}}}\AgdaSymbol{)}\AgdaSpace{}%
\AgdaKeyword{public}\<%
\\
%
\\[\AgdaEmptyExtraSkip]%
\>[0]\AgdaFunction{Domain}\AgdaSpace{}%
\AgdaSymbol{=}\AgdaSpace{}%
\AgdaPrimitive{Set}\AgdaSpace{}%
\AgdaComment{--\ unsound!}\<%
\\
%
\\[\AgdaEmptyExtraSkip]%
\>[0]\AgdaKeyword{variable}\<%
\\
\>[0][@{}l@{\AgdaIndent{0}}]%
\>[2]\AgdaGeneralizable{A}\AgdaSpace{}%
\AgdaGeneralizable{B}\AgdaSpace{}%
\AgdaGeneralizable{C}%
\>[9]\AgdaSymbol{:}\AgdaSpace{}%
\AgdaPrimitive{Set}\<%
\\
%
\>[2]\AgdaGeneralizable{D}\AgdaSpace{}%
\AgdaGeneralizable{E}\AgdaSpace{}%
\AgdaGeneralizable{F}%
\>[9]\AgdaSymbol{:}\AgdaSpace{}%
\AgdaFunction{Domain}\<%
\\
%
\>[2]\AgdaGeneralizable{n}%
\>[9]\AgdaSymbol{:}\AgdaSpace{}%
\AgdaDatatype{Nat}\<%
\\
%
\\[\AgdaEmptyExtraSkip]%
\>[0]\AgdaComment{------------------------------------------------------------------------}\<%
\\
\>[0]\AgdaComment{--\ Domains}\<%
\\
%
\\[\AgdaEmptyExtraSkip]%
\>[0]\AgdaKeyword{postulate}\<%
\\
\>[0][@{}l@{\AgdaIndent{0}}]%
\>[2]\AgdaPostulate{⊥}\AgdaSpace{}%
\AgdaSymbol{:}\AgdaSpace{}%
\AgdaGeneralizable{D}%
\>[21]\AgdaComment{--\ bottom\ element}\<%
\\
%
\>[2]\AgdaPostulate{fix}\AgdaSpace{}%
\AgdaSymbol{:}\AgdaSpace{}%
\AgdaSymbol{(}\AgdaGeneralizable{D}\AgdaSpace{}%
\AgdaSymbol{→}\AgdaSpace{}%
\AgdaGeneralizable{D}\AgdaSymbol{)}\AgdaSpace{}%
\AgdaSymbol{→}\AgdaSpace{}%
\AgdaGeneralizable{D}%
\>[21]\AgdaComment{--\ fixed\ point\ of\ endofunction}\<%
\\
%
\\[\AgdaEmptyExtraSkip]%
\>[0]\AgdaComment{------------------------------------------------------------------------}\<%
\\
\>[0]\AgdaComment{--\ Flat\ domains}\<%
\\
%
\\[\AgdaEmptyExtraSkip]%
\>[0]\AgdaKeyword{postulate}\<%
\\
\>[0][@{}l@{\AgdaIndent{0}}]%
\>[2]\AgdaOperator{\AgdaPostulate{\AgdaUnderscore{}+⊥}}%
\>[9]\AgdaSymbol{:}\AgdaSpace{}%
\AgdaPrimitive{Set}\AgdaSpace{}%
\AgdaSymbol{→}\AgdaSpace{}%
\AgdaFunction{Domain}%
\>[33]\AgdaComment{--\ lifted\ set}\<%
\\
%
\>[2]\AgdaPostulate{η}%
\>[9]\AgdaSymbol{:}\AgdaSpace{}%
\AgdaGeneralizable{A}\AgdaSpace{}%
\AgdaSymbol{→}\AgdaSpace{}%
\AgdaGeneralizable{A}\AgdaSpace{}%
\AgdaOperator{\AgdaPostulate{+⊥}}%
\>[33]\AgdaComment{--\ inclusion}\<%
\\
%
\>[2]\AgdaOperator{\AgdaPostulate{\AgdaUnderscore{}♯}}%
\>[9]\AgdaSymbol{:}\AgdaSpace{}%
\AgdaSymbol{(}\AgdaGeneralizable{A}\AgdaSpace{}%
\AgdaSymbol{→}\AgdaSpace{}%
\AgdaGeneralizable{D}\AgdaSymbol{)}\AgdaSpace{}%
\AgdaSymbol{→}\AgdaSpace{}%
\AgdaSymbol{(}\AgdaGeneralizable{A}\AgdaSpace{}%
\AgdaOperator{\AgdaPostulate{+⊥}}\AgdaSpace{}%
\AgdaSymbol{→}\AgdaSpace{}%
\AgdaGeneralizable{D}\AgdaSymbol{)}%
\>[33]\AgdaComment{--\ Kleisli\ extension}\<%
\\
%
\\[\AgdaEmptyExtraSkip]%
\>[0]\AgdaFunction{Bool⊥}%
\>[9]\AgdaSymbol{=}\AgdaSpace{}%
\AgdaDatatype{Bool}\AgdaSpace{}%
\AgdaOperator{\AgdaPostulate{+⊥}}%
\>[33]\AgdaComment{--\ truth\ value\ domain}\<%
\\
\>[0]\AgdaFunction{Nat⊥}%
\>[9]\AgdaSymbol{=}\AgdaSpace{}%
\AgdaDatatype{Nat}\AgdaSpace{}%
\AgdaOperator{\AgdaPostulate{+⊥}}%
\>[33]\AgdaComment{--\ natural\ number\ domain}\<%
\\
\>[0]\AgdaFunction{String⊥}%
\>[9]\AgdaSymbol{=}\AgdaSpace{}%
\AgdaPostulate{String}\AgdaSpace{}%
\AgdaOperator{\AgdaPostulate{+⊥}}%
\>[33]\AgdaComment{--\ meta-string\ domain}\<%
\\
%
\\[\AgdaEmptyExtraSkip]%
\>[0]\AgdaKeyword{postulate}\<%
\\
\>[0][@{}l@{\AgdaIndent{0}}]%
\>[2]\AgdaOperator{\AgdaPostulate{\AgdaUnderscore{}==⊥\AgdaUnderscore{}}}%
\>[9]\AgdaSymbol{:}\AgdaSpace{}%
\AgdaFunction{Nat⊥}\AgdaSpace{}%
\AgdaSymbol{→}\AgdaSpace{}%
\AgdaDatatype{Nat}\AgdaSpace{}%
\AgdaSymbol{→}\AgdaSpace{}%
\AgdaFunction{Bool⊥}%
\>[33]\AgdaComment{--\ strict\ numerical\ equality}\<%
\\
%
\>[2]\AgdaOperator{\AgdaPostulate{\AgdaUnderscore{}⟶\AgdaUnderscore{},\AgdaUnderscore{}}}%
\>[9]\AgdaSymbol{:}\AgdaSpace{}%
\AgdaFunction{Bool⊥}\AgdaSpace{}%
\AgdaSymbol{→}\AgdaSpace{}%
\AgdaGeneralizable{D}\AgdaSpace{}%
\AgdaSymbol{→}\AgdaSpace{}%
\AgdaGeneralizable{D}\AgdaSpace{}%
\AgdaSymbol{→}\AgdaSpace{}%
\AgdaGeneralizable{D}%
\>[33]\AgdaComment{--\ McCarthy\ conditional}\<%
\\
%
\\[\AgdaEmptyExtraSkip]%
\>[0]\AgdaComment{------------------------------------------------------------------------}\<%
\\
\>[0]\AgdaComment{--\ Sum\ domains}\<%
\\
%
\\[\AgdaEmptyExtraSkip]%
\>[0]\AgdaKeyword{postulate}\<%
\\
\>[0][@{}l@{\AgdaIndent{0}}]%
\>[2]\AgdaOperator{\AgdaPostulate{\AgdaUnderscore{}+\AgdaUnderscore{}}}%
\>[9]\AgdaSymbol{:}\AgdaSpace{}%
\AgdaFunction{Domain}\AgdaSpace{}%
\AgdaSymbol{→}\AgdaSpace{}%
\AgdaFunction{Domain}\AgdaSpace{}%
\AgdaSymbol{→}\AgdaSpace{}%
\AgdaFunction{Domain}%
\>[44]\AgdaComment{--\ separated\ sum}\<%
\\
%
\>[2]\AgdaPostulate{inj₁}%
\>[9]\AgdaSymbol{:}\AgdaSpace{}%
\AgdaGeneralizable{D}\AgdaSpace{}%
\AgdaSymbol{→}\AgdaSpace{}%
\AgdaGeneralizable{D}\AgdaSpace{}%
\AgdaOperator{\AgdaPostulate{+}}\AgdaSpace{}%
\AgdaGeneralizable{E}%
\>[44]\AgdaComment{--\ injection}\<%
\\
%
\>[2]\AgdaPostulate{inj₂}%
\>[9]\AgdaSymbol{:}\AgdaSpace{}%
\AgdaGeneralizable{E}\AgdaSpace{}%
\AgdaSymbol{→}\AgdaSpace{}%
\AgdaGeneralizable{D}\AgdaSpace{}%
\AgdaOperator{\AgdaPostulate{+}}\AgdaSpace{}%
\AgdaGeneralizable{E}%
\>[44]\AgdaComment{--\ injection}\<%
\\
%
\>[2]\AgdaOperator{\AgdaPostulate{[\AgdaUnderscore{},\AgdaUnderscore{}]}}%
\>[9]\AgdaSymbol{:}\AgdaSpace{}%
\AgdaSymbol{(}\AgdaGeneralizable{D}\AgdaSpace{}%
\AgdaSymbol{→}\AgdaSpace{}%
\AgdaGeneralizable{F}\AgdaSymbol{)}\AgdaSpace{}%
\AgdaSymbol{→}\AgdaSpace{}%
\AgdaSymbol{(}\AgdaGeneralizable{E}\AgdaSpace{}%
\AgdaSymbol{→}\AgdaSpace{}%
\AgdaGeneralizable{F}\AgdaSymbol{)}\AgdaSpace{}%
\AgdaSymbol{→}\AgdaSpace{}%
\AgdaSymbol{(}\AgdaGeneralizable{D}\AgdaSpace{}%
\AgdaOperator{\AgdaPostulate{+}}\AgdaSpace{}%
\AgdaGeneralizable{E}\AgdaSpace{}%
\AgdaSymbol{→}\AgdaSpace{}%
\AgdaGeneralizable{F}\AgdaSymbol{)}%
\>[44]\AgdaComment{--\ case\ analysis}\<%
\\
%
\\[\AgdaEmptyExtraSkip]%
\>[0]\AgdaComment{------------------------------------------------------------------------}\<%
\\
\>[0]\AgdaComment{--\ Product\ domains}\<%
\\
%
\\[\AgdaEmptyExtraSkip]%
\>[0]\AgdaKeyword{postulate}\<%
\\
\>[0][@{}l@{\AgdaIndent{0}}]%
\>[2]\AgdaOperator{\AgdaPostulate{\AgdaUnderscore{}×\AgdaUnderscore{}}}%
\>[7]\AgdaSymbol{:}\AgdaSpace{}%
\AgdaFunction{Domain}\AgdaSpace{}%
\AgdaSymbol{→}\AgdaSpace{}%
\AgdaFunction{Domain}\AgdaSpace{}%
\AgdaSymbol{→}\AgdaSpace{}%
\AgdaFunction{Domain}%
\>[35]\AgdaComment{--\ cartesian\ product}\<%
\\
%
\>[2]\AgdaOperator{\AgdaPostulate{\AgdaUnderscore{},\AgdaUnderscore{}}}%
\>[7]\AgdaSymbol{:}\AgdaSpace{}%
\AgdaGeneralizable{D}\AgdaSpace{}%
\AgdaSymbol{→}\AgdaSpace{}%
\AgdaGeneralizable{E}\AgdaSpace{}%
\AgdaSymbol{→}\AgdaSpace{}%
\AgdaGeneralizable{D}\AgdaSpace{}%
\AgdaOperator{\AgdaPostulate{×}}\AgdaSpace{}%
\AgdaGeneralizable{E}%
\>[35]\AgdaComment{--\ pairing}\<%
\\
%
\>[2]\AgdaOperator{\AgdaPostulate{\AgdaUnderscore{}↓1}}%
\>[7]\AgdaSymbol{:}\AgdaSpace{}%
\AgdaGeneralizable{D}\AgdaSpace{}%
\AgdaOperator{\AgdaPostulate{×}}\AgdaSpace{}%
\AgdaGeneralizable{E}\AgdaSpace{}%
\AgdaSymbol{→}\AgdaSpace{}%
\AgdaGeneralizable{D}%
\>[35]\AgdaComment{--\ projection}\<%
\\
%
\>[2]\AgdaOperator{\AgdaPostulate{\AgdaUnderscore{}↓2}}%
\>[7]\AgdaSymbol{:}\AgdaSpace{}%
\AgdaGeneralizable{D}\AgdaSpace{}%
\AgdaOperator{\AgdaPostulate{×}}\AgdaSpace{}%
\AgdaGeneralizable{E}\AgdaSpace{}%
\AgdaSymbol{→}\AgdaSpace{}%
\AgdaGeneralizable{E}%
\>[35]\AgdaComment{--\ projection}\<%
\\
%
\\[\AgdaEmptyExtraSkip]%
\>[0]\AgdaComment{------------------------------------------------------------------------}\<%
\\
\>[0]\AgdaComment{--\ Tuple\ domains}\<%
\\
%
\\[\AgdaEmptyExtraSkip]%
\>[0]\AgdaOperator{\AgdaFunction{\AgdaUnderscore{}\textasciicircum{}\AgdaUnderscore{}}}\AgdaSpace{}%
\AgdaSymbol{:}\AgdaSpace{}%
\AgdaFunction{Domain}\AgdaSpace{}%
\AgdaSymbol{→}\AgdaSpace{}%
\AgdaDatatype{Nat}\AgdaSpace{}%
\AgdaSymbol{→}\AgdaSpace{}%
\AgdaFunction{Domain}%
\>[30]\AgdaComment{--\ D\ \textasciicircum{}\ n\ \ \ \ \ \ \ \ \ \ \ \ \ \ \ n-tuples}\<%
\\
\>[0]\AgdaBound{D}\AgdaSpace{}%
\AgdaOperator{\AgdaFunction{\textasciicircum{}}}\AgdaSpace{}%
\AgdaNumber{0}%
\>[17]\AgdaSymbol{=}\AgdaSpace{}%
\AgdaRecord{⊤}\<%
\\
\>[0]\AgdaBound{D}\AgdaSpace{}%
\AgdaOperator{\AgdaFunction{\textasciicircum{}}}\AgdaSpace{}%
\AgdaNumber{1}%
\>[17]\AgdaSymbol{=}\AgdaSpace{}%
\AgdaBound{D}\<%
\\
\>[0]\AgdaBound{D}\AgdaSpace{}%
\AgdaOperator{\AgdaFunction{\textasciicircum{}}}\AgdaSpace{}%
\AgdaInductiveConstructor{suc}\AgdaSpace{}%
\AgdaSymbol{(}\AgdaInductiveConstructor{suc}\AgdaSpace{}%
\AgdaBound{n}\AgdaSymbol{)}%
\>[17]\AgdaSymbol{=}\AgdaSpace{}%
\AgdaBound{D}\AgdaSpace{}%
\AgdaOperator{\AgdaPostulate{×}}\AgdaSpace{}%
\AgdaSymbol{(}\AgdaBound{D}\AgdaSpace{}%
\AgdaOperator{\AgdaFunction{\textasciicircum{}}}\AgdaSpace{}%
\AgdaInductiveConstructor{suc}\AgdaSpace{}%
\AgdaBound{n}\AgdaSymbol{)}\<%
\\
%
\\[\AgdaEmptyExtraSkip]%
\>[0]\AgdaComment{------------------------------------------------------------------------}\<%
\\
\>[0]\AgdaComment{--\ Finite\ sequence\ domains}\<%
\\
%
\\[\AgdaEmptyExtraSkip]%
\>[0]\AgdaKeyword{postulate}\<%
\\
\>[0][@{}l@{\AgdaIndent{0}}]%
\>[2]\AgdaOperator{\AgdaPostulate{\AgdaUnderscore{}⋆}}%
\>[9]\AgdaSymbol{:}\AgdaSpace{}%
\AgdaFunction{Domain}\AgdaSpace{}%
\AgdaSymbol{→}\AgdaSpace{}%
\AgdaFunction{Domain}%
\>[30]\AgdaComment{--\ D\ ⋆\ domain\ of\ finite\ sequences\ }\<%
\\
%
\>[2]\AgdaPostulate{⟨⟩}%
\>[9]\AgdaSymbol{:}\AgdaSpace{}%
\AgdaGeneralizable{D}\AgdaSpace{}%
\AgdaOperator{\AgdaPostulate{⋆}}%
\>[30]\AgdaComment{--\ empty\ sequence}\<%
\\
%
\>[2]\AgdaOperator{\AgdaPostulate{⟨\AgdaUnderscore{}⟩}}%
\>[9]\AgdaSymbol{:}\AgdaSpace{}%
\AgdaSymbol{(}\AgdaGeneralizable{D}\AgdaSpace{}%
\AgdaOperator{\AgdaFunction{\textasciicircum{}}}\AgdaSpace{}%
\AgdaInductiveConstructor{suc}\AgdaSpace{}%
\AgdaGeneralizable{n}\AgdaSymbol{)}\AgdaSpace{}%
\AgdaSymbol{→}\AgdaSpace{}%
\AgdaGeneralizable{D}\AgdaSpace{}%
\AgdaOperator{\AgdaPostulate{⋆}}%
\>[30]\AgdaComment{--\ ⟨\ d₁\ ,\ ...\ ,\ dₙ₊₁\ ⟩\ non-empty\ sequence}\<%
\\
%
\>[2]\AgdaPostulate{\#}%
\>[9]\AgdaSymbol{:}\AgdaSpace{}%
\AgdaGeneralizable{D}\AgdaSpace{}%
\AgdaOperator{\AgdaPostulate{⋆}}\AgdaSpace{}%
\AgdaSymbol{→}\AgdaSpace{}%
\AgdaFunction{Nat⊥}%
\>[30]\AgdaComment{--\ \#\ d⋆\ \ \ \ \ \ \ \ \ \ \ \ \ \ \ \ sequence\ length}\<%
\\
%
\>[2]\AgdaOperator{\AgdaPostulate{\AgdaUnderscore{}§\AgdaUnderscore{}}}%
\>[9]\AgdaSymbol{:}\AgdaSpace{}%
\AgdaGeneralizable{D}\AgdaSpace{}%
\AgdaOperator{\AgdaPostulate{⋆}}\AgdaSpace{}%
\AgdaSymbol{→}\AgdaSpace{}%
\AgdaGeneralizable{D}\AgdaSpace{}%
\AgdaOperator{\AgdaPostulate{⋆}}\AgdaSpace{}%
\AgdaSymbol{→}\AgdaSpace{}%
\AgdaGeneralizable{D}\AgdaSpace{}%
\AgdaOperator{\AgdaPostulate{⋆}}%
\>[30]\AgdaComment{--\ d⋆\ §\ d⋆\ \ \ \ \ \ \ \ \ \ \ \ \ concatenation}\<%
\\
%
\>[2]\AgdaOperator{\AgdaPostulate{\AgdaUnderscore{}↓\AgdaUnderscore{}}}%
\>[9]\AgdaSymbol{:}\AgdaSpace{}%
\AgdaGeneralizable{D}\AgdaSpace{}%
\AgdaOperator{\AgdaPostulate{⋆}}\AgdaSpace{}%
\AgdaSymbol{→}\AgdaSpace{}%
\AgdaDatatype{Nat}\AgdaSpace{}%
\AgdaSymbol{→}\AgdaSpace{}%
\AgdaGeneralizable{D}%
\>[30]\AgdaComment{--\ d⋆\ ↓\ n\ \ \ \ \ \ \ \ \ \ \ \ \ \ nth\ component}\<%
\\
%
\>[2]\AgdaOperator{\AgdaPostulate{\AgdaUnderscore{}†\AgdaUnderscore{}}}%
\>[9]\AgdaSymbol{:}\AgdaSpace{}%
\AgdaGeneralizable{D}\AgdaSpace{}%
\AgdaOperator{\AgdaPostulate{⋆}}\AgdaSpace{}%
\AgdaSymbol{→}\AgdaSpace{}%
\AgdaDatatype{Nat}\AgdaSpace{}%
\AgdaSymbol{→}\AgdaSpace{}%
\AgdaGeneralizable{D}\AgdaSpace{}%
\AgdaOperator{\AgdaPostulate{⋆}}%
\>[30]\AgdaComment{--\ d⋆\ †\ n\ \ \ \ \ \ \ \ \ \ \ \ \ \ nth\ tail}\<%
\\
%
\\[\AgdaEmptyExtraSkip]%
\>[0]\AgdaComment{------------------------------------------------------------------------}\<%
\\
\>[0]\AgdaComment{--\ Grouping\ precedence}\<%
\\
%
\\[\AgdaEmptyExtraSkip]%
\>[0]\AgdaKeyword{infixr}\AgdaSpace{}%
\AgdaNumber{1}%
\>[11]\AgdaOperator{\AgdaPostulate{\AgdaUnderscore{}+\AgdaUnderscore{}}}\<%
\\
\>[0]\AgdaKeyword{infixr}\AgdaSpace{}%
\AgdaNumber{2}%
\>[11]\AgdaOperator{\AgdaPostulate{\AgdaUnderscore{}×\AgdaUnderscore{}}}\<%
\\
\>[0]\AgdaKeyword{infixr}\AgdaSpace{}%
\AgdaNumber{4}%
\>[11]\AgdaOperator{\AgdaPostulate{\AgdaUnderscore{},\AgdaUnderscore{}}}\<%
\\
\>[0]\AgdaKeyword{infix}%
\>[7]\AgdaNumber{8}%
\>[11]\AgdaOperator{\AgdaFunction{\AgdaUnderscore{}\textasciicircum{}\AgdaUnderscore{}}}\<%
\\
\>[0]\AgdaKeyword{infixr}\AgdaSpace{}%
\AgdaNumber{20}%
\>[11]\AgdaOperator{\AgdaPostulate{\AgdaUnderscore{}⟶\AgdaUnderscore{},\AgdaUnderscore{}}}\<%
\\
%
\\[\AgdaEmptyExtraSkip]%
\>[0]\AgdaOperator{\AgdaFunction{⟦\AgdaUnderscore{}⟧}}\AgdaSpace{}%
\AgdaSymbol{=}\AgdaSpace{}%
\AgdaFunction{id}\<%
\end{code}  

\clearpage

\section{Soundness Tests}
\begin{code}%
\>[0]\AgdaSymbol{\{-\#}\AgdaSpace{}%
\AgdaKeyword{OPTIONS}\AgdaSpace{}%
\AgdaPragma{--rewriting}\AgdaSpace{}%
\AgdaPragma{--confluence-check}\AgdaSpace{}%
\AgdaSymbol{\#-\}}\<%
\\
%
\\[\AgdaEmptyExtraSkip]%
\>[0]\AgdaKeyword{open}\AgdaSpace{}%
\AgdaKeyword{import}\AgdaSpace{}%
\AgdaModule{Agda.Builtin.Equality}\<%
\\
\>[0]\AgdaKeyword{open}\AgdaSpace{}%
\AgdaKeyword{import}\AgdaSpace{}%
\AgdaModule{Agda.Builtin.Equality.Rewrite}\<%
\\
%
\\[\AgdaEmptyExtraSkip]%
\>[0]\AgdaKeyword{module}\AgdaSpace{}%
\AgdaModule{ScmQE.Soundness-Tests}\AgdaSpace{}%
\AgdaKeyword{where}\<%
\\
%
\\[\AgdaEmptyExtraSkip]%
\>[0]\AgdaKeyword{open}\AgdaSpace{}%
\AgdaKeyword{import}\AgdaSpace{}%
\AgdaModule{Notation}\<%
\\
\>[0]\AgdaKeyword{open}\AgdaSpace{}%
\AgdaKeyword{import}\AgdaSpace{}%
\AgdaModule{ScmQE.Abstract-Syntax}\<%
\\
\>[0]\AgdaKeyword{open}\AgdaSpace{}%
\AgdaKeyword{import}\AgdaSpace{}%
\AgdaModule{ScmQE.Domain-Equations}\<%
\\
\>[0]\AgdaKeyword{open}\AgdaSpace{}%
\AgdaKeyword{import}\AgdaSpace{}%
\AgdaModule{ScmQE.Auxiliary-Functions}\<%
\\
\>[0]\AgdaKeyword{open}\AgdaSpace{}%
\AgdaKeyword{import}\AgdaSpace{}%
\AgdaModule{ScmQE.Semantic-Functions}\<%
\\
%
\\[\AgdaEmptyExtraSkip]%
\>[0]\AgdaKeyword{open}\AgdaSpace{}%
\AgdaKeyword{import}\AgdaSpace{}%
\AgdaModule{Relation.Binary.PropositionalEquality.Core}\<%
\\
\>[0][@{}l@{\AgdaIndent{0}}]%
\>[2]\AgdaKeyword{using}\AgdaSpace{}%
\AgdaSymbol{(}\AgdaOperator{\AgdaDatatype{\AgdaUnderscore{}≡\AgdaUnderscore{}}}\AgdaSymbol{;}\AgdaSpace{}%
\AgdaInductiveConstructor{refl}\AgdaSymbol{;}\AgdaSpace{}%
\AgdaFunction{cong-app}\AgdaSymbol{)}\<%
\\
%
\\[\AgdaEmptyExtraSkip]%
\>[0]\AgdaKeyword{postulate}\<%
\\
\>[0][@{}l@{\AgdaIndent{0}}]%
\>[2]\AgdaPostulate{fix-fix}\AgdaSpace{}%
\AgdaSymbol{:}\AgdaSpace{}%
\AgdaSymbol{(}\AgdaBound{f}\AgdaSpace{}%
\AgdaSymbol{:}\AgdaSpace{}%
\AgdaGeneralizable{D}\AgdaSpace{}%
\AgdaSymbol{→}\AgdaSpace{}%
\AgdaGeneralizable{D}\AgdaSymbol{)}\AgdaSpace{}%
\AgdaSymbol{→}\AgdaSpace{}%
\AgdaPostulate{fix}\AgdaSpace{}%
\AgdaBound{f}\AgdaSpace{}%
\AgdaOperator{\AgdaDatatype{≡}}\AgdaSpace{}%
\AgdaBound{f}\AgdaSpace{}%
\AgdaSymbol{(}\AgdaPostulate{fix}\AgdaSpace{}%
\AgdaBound{f}\AgdaSymbol{)}\<%
\\
%
\\[\AgdaEmptyExtraSkip]%
\>[0]\AgdaFunction{fix-app}\AgdaSpace{}%
\AgdaSymbol{:}%
\>[39I]\AgdaSymbol{(}\AgdaBound{f}\AgdaSpace{}%
\AgdaSymbol{:}\AgdaSpace{}%
\AgdaSymbol{(}\AgdaGeneralizable{A}\AgdaSpace{}%
\AgdaSymbol{→}\AgdaSpace{}%
\AgdaGeneralizable{D}\AgdaSymbol{)}\AgdaSpace{}%
\AgdaSymbol{→}\AgdaSpace{}%
\AgdaSymbol{(}\AgdaGeneralizable{A}\AgdaSpace{}%
\AgdaSymbol{→}\AgdaSpace{}%
\AgdaGeneralizable{D}\AgdaSymbol{))}\AgdaSpace{}%
\AgdaSymbol{(}\AgdaBound{a}\AgdaSpace{}%
\AgdaSymbol{:}\AgdaSpace{}%
\AgdaGeneralizable{A}\AgdaSymbol{)}\AgdaSpace{}%
\AgdaSymbol{→}\<%
\\
\>[39I][@{}l@{\AgdaIndent{0}}]%
\>[12]\AgdaPostulate{fix}\AgdaSpace{}%
\AgdaBound{f}\AgdaSpace{}%
\AgdaBound{a}\AgdaSpace{}%
\AgdaOperator{\AgdaDatatype{≡}}\AgdaSpace{}%
\AgdaBound{f}\AgdaSpace{}%
\AgdaSymbol{(}\AgdaPostulate{fix}\AgdaSpace{}%
\AgdaBound{f}\AgdaSymbol{)}\AgdaSpace{}%
\AgdaBound{a}\<%
\\
\>[0]\AgdaFunction{fix-app}\AgdaSpace{}%
\AgdaBound{f}\AgdaSpace{}%
\AgdaSymbol{=}\AgdaSpace{}%
\AgdaFunction{cong-app}\AgdaSpace{}%
\AgdaSymbol{(}\AgdaPostulate{fix-fix}\AgdaSpace{}%
\AgdaBound{f}\AgdaSymbol{)}\<%
\\
%
\\[\AgdaEmptyExtraSkip]%
\>[0]\AgdaSymbol{\{-\#}\AgdaSpace{}%
\AgdaKeyword{REWRITE}\AgdaSpace{}%
\AgdaFunction{fix-app}\AgdaSpace{}%
\AgdaSymbol{\#-\}}\<%
\\
%
\\[\AgdaEmptyExtraSkip]%
\>[0]\AgdaFunction{test-1}\AgdaSpace{}%
\AgdaSymbol{:}\AgdaSpace{}%
\AgdaSymbol{∀}\AgdaSpace{}%
\AgdaSymbol{\{}\AgdaBound{K}\AgdaSpace{}%
\AgdaBound{ρ}\AgdaSpace{}%
\AgdaBound{κ}\AgdaSymbol{\}}\AgdaSpace{}%
\AgdaSymbol{→}\<%
\\
\>[0][@{}l@{\AgdaIndent{0}}]%
\>[2]\AgdaOperator{\AgdaFunction{ℰ⟦}}\AgdaSpace{}%
\AgdaInductiveConstructor{con}\AgdaSpace{}%
\AgdaBound{K}\AgdaSpace{}%
\AgdaOperator{\AgdaFunction{⟧}}\AgdaSpace{}%
\AgdaBound{ρ}\AgdaSpace{}%
\AgdaBound{κ}\AgdaSpace{}%
\AgdaOperator{\AgdaDatatype{≡}}\AgdaSpace{}%
\AgdaBound{κ}\AgdaSpace{}%
\AgdaSymbol{(}\AgdaOperator{\AgdaFunction{𝒦⟦}}\AgdaSpace{}%
\AgdaBound{K}\AgdaSpace{}%
\AgdaOperator{\AgdaFunction{⟧}}\AgdaSymbol{)}\<%
\\
\>[0]\AgdaFunction{test-1}\AgdaSpace{}%
\AgdaSymbol{=}\AgdaSpace{}%
\AgdaInductiveConstructor{refl}\<%
\\
%
\\[\AgdaEmptyExtraSkip]%
\>[0]\AgdaFunction{test-2}\AgdaSpace{}%
\AgdaSymbol{:}\AgdaSpace{}%
\AgdaSymbol{∀}\AgdaSpace{}%
\AgdaSymbol{\{}\AgdaBound{ρ}\AgdaSpace{}%
\AgdaBound{κ}\AgdaSymbol{\}}\AgdaSpace{}%
\AgdaSymbol{→}\<%
\\
\>[0][@{}l@{\AgdaIndent{0}}]%
\>[2]\AgdaOperator{\AgdaFunction{ℰ⟦}}\AgdaSpace{}%
\AgdaOperator{\AgdaInductiveConstructor{⦅eval}}\AgdaSpace{}%
\AgdaInductiveConstructor{con}\AgdaSpace{}%
\AgdaInductiveConstructor{\#t}\AgdaSpace{}%
\AgdaOperator{\AgdaInductiveConstructor{⦆}}\AgdaSpace{}%
\AgdaOperator{\AgdaFunction{⟧}}\AgdaSpace{}%
\AgdaBound{ρ}\AgdaSpace{}%
\AgdaBound{κ}\AgdaSpace{}%
\AgdaOperator{\AgdaDatatype{≡}}\<%
\\
\>[2][@{}l@{\AgdaIndent{0}}]%
\>[4]\AgdaFunction{datum}\AgdaSpace{}%
\AgdaSymbol{(}\AgdaPostulate{η}\AgdaSpace{}%
\AgdaInductiveConstructor{true}\AgdaSpace{}%
\AgdaOperator{\AgdaPostulate{𝐓-in-𝐄}}\AgdaSymbol{)}\AgdaSpace{}%
\AgdaSymbol{(λ}\AgdaSpace{}%
\AgdaBound{Δ}\AgdaSpace{}%
\AgdaSymbol{→}\AgdaSpace{}%
\AgdaSymbol{(}\AgdaPostulate{fix}\AgdaSpace{}%
\AgdaOperator{\AgdaFunction{ℱ\AgdaUnderscore{}⟦\AgdaUnderscore{}⟧}}\AgdaSymbol{)}\AgdaSpace{}%
\AgdaOperator{\AgdaFunction{exp⟦}}\AgdaSpace{}%
\AgdaBound{Δ}\AgdaSpace{}%
\AgdaOperator{\AgdaFunction{⟧}}\AgdaSpace{}%
\AgdaPostulate{nullenv}\AgdaSpace{}%
\AgdaBound{κ}\AgdaSymbol{)}\<%
\\
\>[0]\AgdaFunction{test-2}\AgdaSpace{}%
\AgdaSymbol{=}\AgdaSpace{}%
\AgdaInductiveConstructor{refl}\<%
\end{code}
\clearpage
\begin{code}%
\>[0]\AgdaFunction{a}\AgdaSpace{}%
\AgdaFunction{b}\AgdaSpace{}%
\AgdaFunction{c}\AgdaSpace{}%
\AgdaFunction{d}\AgdaSpace{}%
\AgdaFunction{e}\AgdaSpace{}%
\AgdaSymbol{:}\AgdaSpace{}%
\AgdaDatatype{Dat}\<%
\\
\>[0]\AgdaFunction{a}\AgdaSpace{}%
\AgdaSymbol{=}\AgdaSpace{}%
\AgdaInductiveConstructor{ide}\AgdaSpace{}%
\AgdaString{"a"}\<%
\\
\>[0]\AgdaFunction{b}\AgdaSpace{}%
\AgdaSymbol{=}\AgdaSpace{}%
\AgdaInductiveConstructor{ide}\AgdaSpace{}%
\AgdaString{"b"}\<%
\\
\>[0]\AgdaFunction{c}\AgdaSpace{}%
\AgdaSymbol{=}\AgdaSpace{}%
\AgdaInductiveConstructor{ide}\AgdaSpace{}%
\AgdaString{"c"}\<%
\\
\>[0]\AgdaFunction{d}\AgdaSpace{}%
\AgdaSymbol{=}\AgdaSpace{}%
\AgdaInductiveConstructor{ide}\AgdaSpace{}%
\AgdaString{"d"}\<%
\\
\>[0]\AgdaFunction{e}\AgdaSpace{}%
\AgdaSymbol{=}\AgdaSpace{}%
\AgdaInductiveConstructor{ide}\AgdaSpace{}%
\AgdaString{"e"}\<%
\\
%
\\[\AgdaEmptyExtraSkip]%
\>[0]\AgdaComment{--\ R7RS\ §6.4}\<%
\\
%
\\[\AgdaEmptyExtraSkip]%
\>[0]\AgdaComment{--\ (a\ b\ c\ d\ e)\ and\ (a\ .\ (b\ .\ (c\ .\ (d\ .\ (e\ .\ ())))))\ are\ equivalent}\<%
\\
\>[0]\AgdaFunction{test-proper-list}\AgdaSpace{}%
\AgdaSymbol{:}\<%
\\
\>[0][@{}l@{\AgdaIndent{0}}]%
\>[2]\AgdaOperator{\AgdaFunction{𝒟⟦}}\AgdaSpace{}%
\AgdaOperator{\AgdaInductiveConstructor{⦅}}\AgdaSpace{}%
\AgdaFunction{a}\AgdaSpace{}%
\AgdaOperator{\AgdaInductiveConstructor{␣␣}}\AgdaSpace{}%
\AgdaFunction{b}\AgdaSpace{}%
\AgdaOperator{\AgdaInductiveConstructor{␣␣}}\AgdaSpace{}%
\AgdaFunction{c}\AgdaSpace{}%
\AgdaOperator{\AgdaInductiveConstructor{␣␣}}\AgdaSpace{}%
\AgdaFunction{d}\AgdaSpace{}%
\AgdaOperator{\AgdaInductiveConstructor{␣␣}}\AgdaSpace{}%
\AgdaFunction{e}\AgdaSpace{}%
\AgdaOperator{\AgdaInductiveConstructor{␣␣}}\AgdaSpace{}%
\AgdaInductiveConstructor{␣␣␣}\AgdaSpace{}%
\AgdaOperator{\AgdaInductiveConstructor{⦆}}\AgdaSpace{}%
\AgdaOperator{\AgdaFunction{⟧}}\AgdaSpace{}%
\AgdaOperator{\AgdaDatatype{≡}}\<%
\\
%
\>[2]\AgdaOperator{\AgdaFunction{𝒟⟦}}\AgdaSpace{}%
\AgdaOperator{\AgdaInductiveConstructor{⦅}}\AgdaSpace{}%
\AgdaOperator{\AgdaInductiveConstructor{␣␣}}\AgdaSpace{}%
\AgdaFunction{a}\AgdaSpace{}%
\AgdaOperator{\AgdaInductiveConstructor{·}}\AgdaSpace{}%
\AgdaOperator{\AgdaInductiveConstructor{⦅}}\AgdaSpace{}%
\AgdaOperator{\AgdaInductiveConstructor{␣␣}}\AgdaSpace{}%
\AgdaFunction{b}\AgdaSpace{}%
\AgdaOperator{\AgdaInductiveConstructor{·}}\AgdaSpace{}%
\AgdaOperator{\AgdaInductiveConstructor{⦅}}\AgdaSpace{}%
\AgdaOperator{\AgdaInductiveConstructor{␣␣}}\AgdaSpace{}%
\AgdaFunction{c}\AgdaSpace{}%
\AgdaOperator{\AgdaInductiveConstructor{·}}\AgdaSpace{}%
\AgdaOperator{\AgdaInductiveConstructor{⦅}}\AgdaSpace{}%
\AgdaOperator{\AgdaInductiveConstructor{␣␣}}\AgdaSpace{}%
\AgdaFunction{d}\AgdaSpace{}%
\AgdaOperator{\AgdaInductiveConstructor{·}}%
\>[42]\AgdaOperator{\AgdaInductiveConstructor{⦅}}\AgdaSpace{}%
\AgdaOperator{\AgdaInductiveConstructor{␣␣}}\AgdaSpace{}%
\AgdaFunction{e}\AgdaSpace{}%
\AgdaOperator{\AgdaInductiveConstructor{·}}%
\>[52]\AgdaOperator{\AgdaInductiveConstructor{⦅}}\AgdaSpace{}%
\AgdaInductiveConstructor{␣␣␣}\AgdaSpace{}%
\AgdaOperator{\AgdaInductiveConstructor{⦆}}\AgdaSpace{}%
\AgdaOperator{\AgdaInductiveConstructor{⦆}}\AgdaSpace{}%
\AgdaOperator{\AgdaInductiveConstructor{⦆}}\AgdaSpace{}%
\AgdaOperator{\AgdaInductiveConstructor{⦆}}\AgdaSpace{}%
\AgdaOperator{\AgdaInductiveConstructor{⦆}}\AgdaSpace{}%
\AgdaOperator{\AgdaInductiveConstructor{⦆}}\AgdaSpace{}%
\AgdaOperator{\AgdaFunction{⟧}}\<%
\\
\>[0]\AgdaFunction{test-proper-list}\AgdaSpace{}%
\AgdaSymbol{=}\AgdaSpace{}%
\AgdaInductiveConstructor{refl}\<%
\\
%
\\[\AgdaEmptyExtraSkip]%
\>[0]\AgdaComment{--\ (a\ b\ c\ .\ d)\ is\ equivalent\ to\ (a\ .\ (b\ .\ (c\ .\ d)))}\<%
\\
\>[0]\AgdaFunction{test-improper-list}\AgdaSpace{}%
\AgdaSymbol{:}\<%
\\
\>[0][@{}l@{\AgdaIndent{0}}]%
\>[2]\AgdaOperator{\AgdaFunction{𝒟⟦}}\AgdaSpace{}%
\AgdaOperator{\AgdaInductiveConstructor{⦅}}\AgdaSpace{}%
\AgdaSymbol{(((}\AgdaOperator{\AgdaInductiveConstructor{␣␣}}\AgdaSpace{}%
\AgdaFunction{a}\AgdaSymbol{)}\AgdaSpace{}%
\AgdaOperator{\AgdaInductiveConstructor{␣␣}}\AgdaSpace{}%
\AgdaFunction{b}\AgdaSymbol{)}\AgdaSpace{}%
\AgdaOperator{\AgdaInductiveConstructor{␣␣}}\AgdaSpace{}%
\AgdaFunction{c}\AgdaSymbol{)}\AgdaSpace{}%
\AgdaOperator{\AgdaInductiveConstructor{·}}\AgdaSpace{}%
\AgdaFunction{d}\AgdaSpace{}%
\AgdaOperator{\AgdaInductiveConstructor{⦆}}\AgdaSpace{}%
\AgdaOperator{\AgdaFunction{⟧}}\AgdaSpace{}%
\AgdaOperator{\AgdaDatatype{≡}}\<%
\\
%
\>[2]\AgdaOperator{\AgdaFunction{𝒟⟦}}\AgdaSpace{}%
\AgdaOperator{\AgdaInductiveConstructor{⦅}}\AgdaSpace{}%
\AgdaOperator{\AgdaInductiveConstructor{␣␣}}\AgdaSpace{}%
\AgdaFunction{a}\AgdaSpace{}%
\AgdaOperator{\AgdaInductiveConstructor{·}}\AgdaSpace{}%
\AgdaOperator{\AgdaInductiveConstructor{⦅}}\AgdaSpace{}%
\AgdaOperator{\AgdaInductiveConstructor{␣␣}}\AgdaSpace{}%
\AgdaFunction{b}\AgdaSpace{}%
\AgdaOperator{\AgdaInductiveConstructor{·}}\AgdaSpace{}%
\AgdaOperator{\AgdaInductiveConstructor{⦅}}\AgdaSpace{}%
\AgdaOperator{\AgdaInductiveConstructor{␣␣}}\AgdaSpace{}%
\AgdaFunction{c}\AgdaSpace{}%
\AgdaOperator{\AgdaInductiveConstructor{·}}\AgdaSpace{}%
\AgdaFunction{d}\AgdaSpace{}%
\AgdaOperator{\AgdaInductiveConstructor{⦆}}\AgdaSpace{}%
\AgdaOperator{\AgdaInductiveConstructor{⦆}}\AgdaSpace{}%
\AgdaOperator{\AgdaInductiveConstructor{⦆}}\AgdaSpace{}%
\AgdaOperator{\AgdaFunction{⟧}}\<%
\\
\>[0]\AgdaFunction{test-improper-list}\AgdaSpace{}%
\AgdaSymbol{=}\AgdaSpace{}%
\AgdaInductiveConstructor{refl}\<%
\end{code}

\end{document}
\endinput
%%