% This file should be input after the package agda.sty

% It is intended for overriding definitions in agda.sty
% and for loading further packages, defining Unicode charaters, etc.

\AgdaNoSpaceAroundCode{}

% Agda macros

% For drafting inline use of Agda symbols:

%\newcommand{\AgdaInline}[2][]{{\setlength\fboxsep{0pt}\colorbox{mygray}{\strut\AgdaFontStyle{#1}\strut}}}
\renewcommand{\AgdaRef}[2][]{{\setlength\fboxsep{0pt}\colorbox{mygray}{\strut\AgdaFontStyle{#2}\strut}}}
%\renewcommand{\AgdaRef}[2][]{\textcolor{AgdaBound}{\AgdaFontStyle{#2}}}
%\renewenvironment{AgdaAlign}{}{}
%\renewcommand{\AgdaCodeStyle}{\small}


% For suppressing color boxes that indicate warnings and errors:

\renewcommand{\AgdaUnsolvedMeta}      [1]{#1}
%    {\AgdaFontStyle{\colorbox{AgdaUnsolvedMeta}{#1}}}
\renewcommand{\AgdaUnsolvedConstraint}[1]{#1}
%    {\AgdaFontStyle{\colorbox{AgdaUnsolvedConstraint}{#1}}}
\renewcommand{\AgdaTerminationProblem}[1]{#1}
%    {\AgdaFontStyle{\colorbox{AgdaTerminationProblem}{#1}}}
\renewcommand{\AgdaIncompletePattern} [1]{#1}
%	{\colorbox{AgdaIncompletePattern}{#1}}
\renewcommand{\AgdaErrorWarning}      [1]{#1}
%	{\colorbox{AgdaErrorWarning}{#1}}
\renewcommand{\AgdaError}             [1]{#1}
%    {\textcolor{AgdaError}{\AgdaFontStyle{\underline{#1}}}}


% Fonts for math symbols

% \usepackage{alphabeta} 
% \usepackage{sansmathfonts} % requires OT1 fontenc
\usepackage[bb=bboldx]{mathalpha}
\usepackage{bm}
\usepackage{stmaryrd}
% \usepackage[nohelv, nott]{newtx} % for \lambdabar

% Fonts for Agda highlighting

% agda.sty:
% \newcommand{\AgdaFontStyle}[1]{\textsf{#1}}
% \ifthenelse{\equal{\AgdaColourScheme}{bw}}{
%     \newcommand{\AgdaKeywordFontStyle}[1]{\underline{#1}}
% }{
%     \newcommand{\AgdaKeywordFontStyle}[1]{\textsf{#1}}
% }
% \newcommand{\AgdaStringFontStyle}[1]{\texttt{#1}}
% \newcommand{\AgdaCommentFontStyle}[1]{\texttt{#1}}
% \newcommand{\AgdaBoundFontStyle}[1]{\textit{#1}}

\renewcommand{\AgdaStringFontStyle}[1]{\textrm{#1}}
\renewcommand{\AgdaCommentFontStyle}[1]{\texttt{\small#1}}
\renewcommand{\AgdaBoundFontStyle}[1]{\textsf{#1}}


% This handles the translation of unicode to latex:

\usepackage{ucs}
\usepackage{newunicodechar}

% MATH SYMBOLS

\newcommand{\newucsmath}[3]{\newunicodechar{#1}{\ensuremath{#2{#3}}}}

% \newucsmath {→} \mathnormal \rightarrow
\newucsmath {⇒} \mathnormal \Rightarrow
\newucsmath {⟶} \mathnormal \longrightarrow
\newucsmath {⟹} \mathnormal \Longrightarrow
\newucsmath {↔} \mathnormal \leftrightarrow
\newucsmath {▻} \mathnormal \triangleright
\newucsmath {◅} \mathnormal \triangleleft
\newucsmath {∀} \mathnormal \forall
\newucsmath {∧} \mathnormal \wedge
\newucsmath {∪} \mathnormal \cup
\newucsmath {⊃} \mathnormal \supset
\newucsmath {∈} \mathnormal \in
\newucsmath {≤} \mathnormal \leq
\newucsmath {≥} \mathnormal \geq
\newucsmath {≡} \mathnormal \equiv
\newucsmath {≃} \mathnormal \sim
\newucsmath {∎} \mathnormal \square
\newucsmath {⊑} \mathnormal \sqsubseteq
\newucsmath {⟦} \mathnormal \llbracket
\newucsmath {⟧} \mathnormal \rrbracket
\newucsmath {⦅} \mathnormal {\llparenthesis\,}
\newucsmath {⦆} \mathnormal {\,\rrparenthesis}
\newucsmath {⊎} \mathnormal \uplus
\newucsmath {⊕} \mathnormal \oplus
\newucsmath {∘} \mathnormal \circ
\newucsmath {⊥} \mathnormal \bot
\newucsmath {⊤} \mathnormal \top
\newucsmath {′} \mathnormal {'}
\newucsmath {″} \mathnormal {''}
\newucsmath {̇} \mathnormal {\strut^.}

% GREEK LETTERS

\newucsmath {α} \mathnormal \alpha
\newucsmath {γ} \mathnormal \gamma
\newucsmath {δ} \mathnormal \delta
\newucsmath {ϵ} \mathnormal \epsilon
\newucsmath {ε} \mathnormal \varepsilon
\newucsmath {φ} \mathnormal \phi
\newucsmath {ψ} \mathnormal \psi
\newucsmath {θ} \mathnormal \theta
\newucsmath {η} \mathnormal \eta
\newucsmath {ι} \mathnormal \iota
\newucsmath {κ} \mathnormal \kappa
\newucsmath {λ} \mathnormal \lambda
\newucsmath {ƛ} \mathnormal \lambdabar
\newucsmath {μ} \mathnormal \mu
\newucsmath {ν} \mathnormal \nu
\newucsmath {π} \mathnormal \pi
\newucsmath {ρ} \mathnormal \rho
\newucsmath {σ} \mathnormal \sigma
\newucsmath {τ} \mathnormal \tau
\newucsmath {χ} \mathnormal \chi
\newucsmath {ζ} \mathnormal \zeta
\newucsmath {Γ} \mathrm \Gamma
\newucsmath {Δ} \mathrm \Delta
\newucsmath {Σ} \mathrm \Sigma

% BLACKBOARD BOLD

\newucsmath {𝟙} \mathbb 1
\newucsmath {𝕃} \mathbb L
\newucsmath {ℕ} \mathbb N
\newucsmath {𝕋} \mathbb T
\newucsmath {ℤ} \mathbb Z

% CALLIGRAPHIC

\newucsmath {𝒜} \mathcal A
\newucsmath {𝒞} \mathcal C
\newucsmath {𝒟} \mathcal D
\newucsmath {𝓓} \mathcal D
\newucsmath {ℰ} \mathcal E
\newucsmath {𝒦} \mathcal K
\newucsmath {ℒ} \mathcal L
\newucsmath {𝒫} \mathcal P
\newucsmath {𝒬} \mathcal Q
\newucsmath {𝓤} \mathcal U
\newucsmath {𝒱} \mathcal V

% SUBSCRIPTS

\newcommand{\newucssub}[2]{\newunicodechar{#1}{\ensuremath{{}_#2}}}

\newucssub {∞} \infty
\newucssub {₀} 0
\newucssub {₁} 1
\newucssub {₂} 2
\newucssub {ᵢ} {\mathrm i}
\newucssub {ₙ} {\mathrm n}
\newucssub {ₒ} {\mathrm o}

% SUPERSCRIPTS

\newcommand{\newucssup}[2]{\newunicodechar{#1}{\ensuremath{{}^#2}}}

\newucssup {⋆} * % \ast seems to give the same
\newucssup {♯} \sharp
\newucssup {ᵇ} {\mathrm b}
\newucssup {ᶜ} {\mathrm c}
\newucssup {ᵈ} {\mathrm d}
\newucssup {ᶠ} {\mathrm f}
\newucssup {ᵍ} {\mathrm g}
\newucssup {ˡ} {\mathrm l}
\newucssup {ᵖ} {\mathrm p}
\newucssup {ᵒ} {\mathrm o}
\newucssup {ʳ} {\mathrm r}
\newucssup {ᵛ} {\mathrm v}

\newucssup {ᴱ} {\mathrm E}
\newucssup {ᴴ} {\mathrm H}
\newucssup {ᴵ} {\mathrm I}
\newucssup {ᴸ} {\mathrm L}
\newucssup {ᴹ} {\mathrm M}
\newucssup {ᴼ} {\mathrm O}
\newucssup {ᴿ} {\mathrm R}
\newucssup {ᵀ} {\mathrm T}
\newucssup {ⱽ} {\mathrm V}

% TEXT

\newcommand{\newucstext}[3]{\newunicodechar{#1}{#2{#3}}}

\newucstext {∙} \textbf {.}
\newucstext {+} \textbf {+}
% \newucstext {–} \textbf {--}

\newucstext {𝐀} \textbf A
\newucstext {𝐂} \textbf C
\newucstext {𝐃} \textbf D
\newucstext {𝐄} \textbf E
\newucstext {𝐅} \textbf F
\newucstext {𝐇} \textbf H
\newucstext {𝐊} \textbf K
\newucstext {𝐋} \textbf L
\newucstext {𝐌} \textbf M
\newucstext {𝐍} \textbf N
\newucstext {𝐏} \textbf P
\newucstext {𝐐} \textbf Q
\newucstext {𝐑} \textbf R
\newucstext {𝐒} \textbf S
\newucstext {𝐓} \textbf T
\newucstext {𝐔} \textbf U
\newucstext {𝐖} \textbf W
\newucstext {𝐗} \textbf X
\newucstext {𝐩} \textbf p
\newucstext {𝐬} \textbf s
\newucstext {𝐯} \textbf v
\newucstext {𝐱} \textbf x

\newucstext {𝐿} \textit L
\newucstext {𝑁} \textit N
\newucstext {𝑉} \textit V
\newucstext {𝑌} \textit Y
\newucstext {𝑍} \textit Z

% MISC

\newucsmath {ℓ} \mathnormal \ell

\newunicodechar{⍄}{\ensuremath{\mathnormal{\mbox{\,\setlength\fboxsep{1pt}\fbox{$>$}\,}}}}

\newunicodechar{‵}{\ensuremath{\mathnormal{\mkern 2mu}}} % for \/ in code comments: -- send‵′ 